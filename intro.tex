\begin{subsection}{Technical overview}

  Our construction follows the idea of the general transmission form $\Sigma$-protocol to DVNIZK proposed by~\cite{DBLP:conf/pkc/ChaidosG15},
  try using the different RLWE based encryption scheme to instantiate their scheme.

  \paragraph{Ideas}: A first remark is that, such a $\Sigma$-protocol as proposed in Appendix~\ref{sigma} uses only linear operations to prove the statement,
    thus the prover can do all the operations without knowledge of the challenge,
    all he need is only an encryption of the challenge.
    Using the IND-CPA property of the underlying encryption scheme,
    the challenge does not even needed to be generated by the verifier like in the sigma protocol,
    we can generate a random challenge and encrypt it in the setup phase of the protocol.
    This trick remove the interaction of the challenge phase which is the only interaction in the $\Sigma$-protocol.
    The main difficulty for use the same idea in the lattice based cryptography is
    \begin{enumerate}
    \item the underlying encryption scheme should be linearly homomorphic which is not the case in most of the lattice based encryption scheme,
    \item the message space should not be only $\{0,1\}$, otherwise this proof system cannot prove anything,
    \item the bit size of challenge space which is identical with the message space is required at least as large as the security parameter,
    \item we also need to encrypt the randomness used in the schemes, which are usually much larger than the message space in lattice based encryption scheme.
    \end{enumerate}  
  
\end{subsection}




\begin{subsection}{Notations}
  For a distribution $\mathcal{D}$ over a finite set $\mathcal{S}$,
  we denote $d \sample \mathcal{D}$ for the fact that $d$ is chosen from the distribution $\mathcal{D}$ and
  $d \sample \mathcal{S}$ for the fact that $d$ is chosen from the uniform distribution over the finite set $\mathcal{S}$.
  We note also $\mathcal{D}_{\mathcal{S}, r}$ a discrete Gaussian distribution over the set $\mathcal{S}$ with deviation $r$.

  We denote the ring of polynomials over the integers by $\mathbb{Z}[x]$.
  For a degree $n$ irreducible polynomial $f(x)$, we note the quotient ring $\R = \mathbb{Z}[x]/\langle f(x) \rangle$.
  Especially for a prime number $q$, we denote the ring of polynomial over the finite field $\mathbb{Z}_q$ by $\mathbb{Z}_q[x]$
  and the quotient ring by a degree $n$ polynomial $f(x)$ by $\R_q = \mathbb{Z}_q[x]/\langle f(x) \rangle$.
  We denote also $\mathcal{S}_q$ a subset of the finite field $\R_q$ consists of polynomials with coefficients only in $ \{0,1\}$.

  

  \todo{Need several more definition, discrete Gaussian distribution, vector, matrix \etc}
\end{subsection}
