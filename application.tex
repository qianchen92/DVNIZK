\begin{section}{Ring-signature}
  From the PLWE based sigma-protocol and
  the similar techniques used in~\cite{DBLP:conf/eurocrypt/GrothK15}.
  We construct the new ring signature.
  The main idea is to use the $\Sigma$-protocol to prove the fact that
  there exists at least one out of $n$ commitments is a commitment of $0$.
  For a commitment scheme $Com = (\Com, \Verif)$ with his commitment key $\ck$
  and the commitment space $\mathcal{C}_{\ck}$,
  we need to prove the following statement.

  \begin{align*}
    \mathcal{R} &= \{(\ck, (c_1, \dots c_{N-1}), (\ell, r)) | \forall i. c_i \in \mathcal{C}_\ck and c_\ell = \Com(0; r)\}
  \end{align*}


  \subsection{Commitment}
  We present the underlying commitment used in the construction of the $\Sigma$-protocol
  and the ring signature.
  Notice that this is basically the PLWE encryption scheme as we presented in the section~\ref{PLWE}
  \begin{description}
    \item [$\Setup(1^\lambda)$]
  \end{description}

  \begin{subsection}{$\Sigma$-protocol}
  We start with the construction of a such $\Sigma$-protocol,
    we denote $\Sigma_0$ the $\Sigma$-protocol construct in the previous section,
    which allows us to prove that a commitment is committed to $0$ or $1$.
    \begin{description}
      \item[$\Setup$]: where $\lambda$ is the security parameter.
        \begin{enumerate}
          \item $\PPP \gets \Sigma_0.\Setup(1^\lambda)$.
        \end{enumerate}
        \item[$\Prove(\PPP, (c_1, c_2 \cdots c_N), (\ell, r))$]:
        Recall that we need to prove that $\forall i. c_i \in \mathcal{C}_\ck and c_\ell = \Com(0; r)$
        \begin{enumerate}
          \item We denote $\ell_j$ the $j$-th bit $\ell$. We begin to commit $c_{\ell_1}, \dots, c_{\ell_n}$ of $\ell_1, \dots, \ell_n$.
          \item We use $\Sigma_0$ to produce in parallel $\pi_{\ell_1}, \dots, \pi_{\ell_n}$ the proofs of the fact $\ell_1, \dots, \ell_n$ are bits.
          \item For $j = 1 \dots n$
          \begin{enumerate}
          \item
          \end{enumerate}

        \end{enumerate}

    \end{description}
  \end{subsection}
  \begin{description}
    \item[$\Setup(1^{\lambda})$]:
    \begin{enumerate}
      \item
    \end{enumerate}
  \end{description}
\end{section}

\begin{section}{Electronic voting system}
\end{section}

\begin{section}{New Cramer-Shoup type lattice based CCA encryption}
  \todo{this one maybe impossible because this is hash proof system approach which
  use the statistic argument}
\end{section}
