\begin{subsection}{Ring-signature}
  From the PLWE based sigma-protocol and the similar techniques used in~\cite{DBLP:conf/eurocrypt/GrothK15}. We construct the new ring signature.
  The main idea is to usethe $\Sigma$-protocol to prove the fact that there exists at least one out of $n$ commitments is a commitment of $0$.
  For a commitment scheme $Com = (\Com, \Verif)$ with his commitment key $\ck$ and the commitment space $\mathcal{C}_{\ck}$,
  we need to prove the following statement.

  \begin{align*}
    \mathcal{R} &= \{(\ck, (c_1, \dots c_{N-1}), (\ell, r)) | \forall i. c_i \in \}
  \end{align*}
  \begin{description}
  \item[$\Setup(1^{\lambda})$]:
    \begin{enumerate}
    \item 
    \end{enumerate}
  \end{description}
\end{subsection}

\begin{subsection}{Electronic voting system}
\end{subsection}

\begin{subsection}{New Cramer-Shoup type lattice based CCA encryption}

\end{subsection}
