\begin{section}{Application: Ring-signature}
  From the PLWE based sigma-protocol and
  the similar techniques used in~\cite{DBLP:conf/eurocrypt/GrothK15}.
  We construct the new ring signature.
  The main idea is to use the $\Sigma$-protocol to prove the fact that
  there exists at least one out of $n$ commitments is a commitment of $0$.
  For a commitment scheme $Com = (\Com, \Verif)$ with his commitment key $\ck$
  and the commitment space $\mathcal{C}_{\ck}$,
  we need to prove the following statement.

  \begin{align*}
    \mathcal{R} &= \{(\ck, (c_1, \dots c_{N-1}), (\ell, r)) | \forall i. c_i \in \mathcal{C}_\ck \wedge c_\ell = \Com(0; r)\}
  \end{align*}


  \subsection{Commitment}
  We present the underlying commitment used in the construction of the $\Sigma$-protocol
  and the ring signature.
  Notice that this is basically the PLWE encryption scheme as we presented in the section~\ref{PLWE}
  \begin{description}
    \item [$\Setup(1^\lambda) \to \ck$]:
    \begin{enumerate}
      \item Generate two uniformly random polynomials $a_0, b_0 \sample \R_q$.
      \item Output $\ck = \begin{bmatrix} a_0 \\ b_0 \end{bmatrix}$.
    \end{enumerate}
    \item [$\Com(\ck, \mu)$]: The message space of the commitment scheme is $\R_2$
    \begin{enumerate}
      \item Parse the commitment key $\ck = \begin{bmatrix} a_0 \\ b_0 \end{bmatrix}$.
      \item Sample polynomials following the Gaussian distribution
      $v,e' \sample \mathcal{D}_{(\R_q, \sigma_{BV})}$ and $e'' \sample \mathcal{D}_{(\R_q, \sigma'_{BV})}$.
      \item Compute
      \begin{align*}
        a &= a_0 v + 2e' & b &= b_0v +2 e''
      \end{align*}
      \item Set $\com = \begin{bmatrix}  b+ \mu \\  -a \end{bmatrix}$ and $\open = (v, e', e'')$.
    \end{enumerate}
    \item [$\Verif(\ck, \mu, \open)$]:
    \begin{enumerate}
      \item Parse the commitment key $\ck = \begin{bmatrix} a_0 \\ b_0 \end{bmatrix}$.
      \item Parse the commitment $\com = \begin{bmatrix} c_0 \\ c_1 \end{bmatrix}$
      and the opening information $\open = (v, e', e'')$.
      \item Verify that
      \begin{enumerate}
        \item $v \leq B_{BV}$
        \item $e' \leq B_{BV}$
        \item $e'' \leq B'_{BV}$
      \end{enumerate}
      \item Verify that
      \begin{align*}
        c_1 &= a_0 v + 2e' & c_0 &= b_0v +2 e'' + \mu
      \end{align*}
    \end{enumerate}
  \end{description}

  This commitment is perfectly hiding and computational binding.\todo[inline]{need to find a reference of it or a proof}

  \begin{subsection}{$\Sigma$-protocol}
    For the following construction, we consider the public parameter $\PPP$ be the commitment key $\ck$ of the underlying commitment scheme.
    \begin{description}


      \item[$\Prove_1(\PPP, (c_1, \dots, c_N), (l, r))$]:
      Recall that we want prove the fact that
      \begin{align*}
        \forall i. c_i \in \mathcal{C}_\ck \wedge c_\ell = \Com(0; r).
      \end{align*}
      \begin{enumerate}
        \item We denote the $j$-th bit of $\ell$ by $\ell_j$,
        $Random$ the random space of the underlying commitment scheme (in our case it will be $\R_q^3$)
          and $n = \ceil{log~N}$.
          \item For $j \in \{1, \dots, n\}$
          \begin{enumerate}
            \item Generate the randomness for the commitments
            $r_j, r^a_j, r^b_j \sample Random$. Remark that the sampling algorithm use the corresponded Gaussian distribution for each component.
            \item Uniformly generate a random polynomial for the $\Sigma$-protocol $a_j \sample R_2$.
            \item $c_{\ell_j} \gets \Com(\ck, \ell_j; r_j)$.
            \item $c_{a_j} \gets \Com(\ck, a_j; r^a_j)$.
            \item $c_{b_j} \gets \Com(\ck, \ell_j a_j; r^b_j)$.
            \item Set polynomials of $x$, $f_{j,0}(x) = (1 - \ell_j) x + a_j$ and $f_{j,1} = \ell_j x+a_j$
          \end{enumerate}
          \item Note $i_j$ the $j$-th bit of $i$. Compute coefficients $p_{i,k}$ of the polynomial for every $i \in \{1, \dots N\}$
          \begin{align*}
            p_i(x) &= \prod_{j=1}^{n} (f_{j,i_j} \cdot x + a_j)\\
            &= \delta_{l,i} x^n + \sum_{k = 1}^{n-1} p_{i,k} x^k
          \end{align*}

          \item For $k \in \{0, \dots, n-1\}$ generate the randomness $s_k \sample random$
          compute the commitments
          \begin{align*}
            c_{d_k} &= \sum_{i = 0}^{N-1} p_{i,k} \cdot c_i + \com(\ck, 0; s_k)
          \end{align*}
          \item Send $\pi_1 = (\{c_{\ell_j}\}_{j = 1}^n, \{c_{a_j}\}_{j = 1}^n,\{c_{b_j}\}_{j = 1}^n,\{c_{d_k}\}_{k = 0}^{n-1})$ to the verifier.
        \end{enumerate}





        \item[$\Prove_2(\PPP, (c_1, \dots, c_N), (l, r), d)$]: This algorithm is the second part of the prove procedure, the prover received the challenge $d$ from the verifier.
        \begin{enumerate}
          \item For $i \in \{1, \dots, n\}$
          \begin{enumerate}
            \item $f_j = \ell_j \cdot d + a_j$.
            \item $z_{a_j} = r_j \cdot d + r^a_j$. Notice that the operation on the randomness are component wise operation.
            \item $z_{b_j} = r_j \cdot (f_j - d) + r^b_j$.
          \end{enumerate}
          \item Compute $z_d = r \cdot d^n - \sum_{k = 0}^{n-1}s_k x^k$
          \item Send $\pi_2 = (\{f_j\}_{j = 1}^n, \{z_{a_j}\}_{j = 1}^n, \{z_{b_j}\}_{j = 1}^n, z_d)$.
        \end{enumerate}



        \item [$\Verif(\PPP, \pi_1, \pi_2) \to \{\True, \False\}$]:
        \begin{enumerate}
          \item Parse $\pi_1 = (\{c_{\ell_j}\}_{j = 1}^n, \{c_{a_j}\}_{j = 1}^n,\{c_{b_j}\}_{j = 1}^n,\{c_{d_k}\}_{k = 0}^{n-1})$.
          \item Parse $\pi_2 = (\{f_j\}_{j = 1}^n, \{z_{a_j}\}_{j = 1}^n, \{z_{b_j}\}_{j = 1}^n, z_d)$.
        \end{enumerate}
    \end{description}
  \end{subsection}
\end{section}

\begin{section}{Electronic voting system}
\end{section}

\begin{section}{New Cramer-Shoup type lattice based CCA encryption}
  \todo{this one maybe impossible because this is hash proof system approach which
  use the statistic argument}
\end{section}
