\begin{section}{Application: Ring-signature}
  From the PLWE based sigma-protocol and
  the similar techniques used in~\cite{DBLP:conf/eurocrypt/GrothK15}.
  We construct the new ring signature.
  The main idea is to use the $\Sigma$-protocol to prove the fact that
  there exists at least one out of $n$ commitments is a commitment of $0$.
  For a commitment scheme $Com = (\Com, \Verif)$ with his commitment key $\ck$
  and the commitment space $\mathcal{C}_{\ck}$,
  we need to prove the following statement.

  \begin{align*}
    \mathcal{R} &= \{(\ck, (c_1, \dots c_{M-1}), (\ell, r)) | \forall i. c_i \in \mathcal{C}_\ck \wedge c_\ell = \Com(0; r)\}
  \end{align*}


  \subsection{Commitment}
  We present the underlying commitment used in the construction of the $\Sigma$-protocol
  and the ring signature.
  Notice that this is basically the PLWE encryption scheme as we presented in the section~\ref{PLWE}
  \begin{description}
    \item [$\Setup(1^\lambda) \to \ck$]:
    \begin{enumerate}
      \item Generate two uniformly random polynomials $a_0, b_0 \sample \R_q$.
      \item Output $\ck = \begin{bmatrix} a_0 \\ b_0 \end{bmatrix}$.
    \end{enumerate}
    \item [$\Com(\ck, \mu) \to \com$]: The message space of the commitment scheme is $\R_2$
    \begin{enumerate}
      \item Parse the commitment key $\ck = \begin{bmatrix} a_0 \\ b_0 \end{bmatrix}$.
      \item Sample polynomials following the Gaussian distribution
      $v,e' \sample \mathcal{D}_{(\R_q, \sigma_{BV})}$ and $e'' \sample \mathcal{D}_{(\R_q, \sigma'_{BV})}$.
      \item Compute
      \begin{align*}
        a &= a_0 v + 2e' & b &= b_0v +2 e''
      \end{align*}
      \item Set $\com = \begin{bmatrix}  b+ \mu \\  -a \end{bmatrix}$ and $\open = (v, e', e'')$.
    \end{enumerate}
    \item [$\Verif(\ck, \mu, \open) \to \{\True, \False\}$]:
    \begin{enumerate}
      \item Parse the commitment key $\ck = \begin{bmatrix} a_0 \\ b_0 \end{bmatrix}$.
      \item Parse the commitment $\com = \begin{bmatrix} c_0 \\ c_1 \end{bmatrix}$
      and the opening information $\open = (v, e', e'')$.
      \item Verify that
      \begin{enumerate}
        \item $v \leq B_{BV}$
        \item $e' \leq B_{BV}$
        \item $e'' \leq B'_{BV}$
      \end{enumerate}
      \item Verify that
      \begin{align*}
        c_1 &= a_0 v + 2e' & c_0 &= b_0v +2 e'' + \mu
      \end{align*}
    \end{enumerate}
  \end{description}

  This commitment is perfectly hiding and computational binding.\todo[inline]{need to find a reference of it or a proof}

  \begin{subsection}{$\Sigma$-protocol}
    For the following construction, we consider the public parameter $\PPP$ be the commitment key $\ck$ of the underlying commitment scheme.
    \begin{description}


      \item[$\Prove_1(\PPP, (c_1, \dots, c_M), (\ell, r))$]:
      Recall that we want prove the fact that
      \begin{align*}
        \forall i. c_i \in \mathcal{C}_\ck \wedge c_\ell = \Com(0; r).
      \end{align*}
      \begin{enumerate}
        \item We denote the $j$-th bit of $\ell$ by $\ell_j$,
        $Random$ the random space of the underlying commitment scheme (in our case it will be $\R_q^3$)
          and $m = \ceil{log~M}$.
          \item For $j \in \{1, \dots, m\}$
          \begin{enumerate}
            \item Generate the randomness for the commitments
            $r_j, r^a_j, r^b_j \sample Random$. Remark that the sampling algorithm use the corresponded Gaussian distribution for each component.
            \item Uniformly generate a random polynomial for the $\Sigma$-protocol $a_j \sample R_2$.
            \item $c_{\ell_j} \gets \Com(\ck, \ell_j; r_j)$.
            \item $c_{a_j} \gets \Com(\ck, a_j; r^a_j)$.
            \item $c_{b_j} \gets \Com(\ck, \ell_j a_j; r^b_j)$.
            \item Set polynomials of $x$, $f_{j,0}(x) = (1 - \ell_j) x + a_j$ and $f_{j,1} = \ell_j x+a_j$
          \end{enumerate}
          \item Note $i_j$ the $j$-th bit of $i$. Compute coefficients $p_{i,k}$ of the polynomial for every $i \in \{1, \dots M\}$
          \begin{align*}
            p_i(x) &= \prod_{j=1}^{m} (f_{j,i_j} \cdot x + a_j)\\
            &= \delta_{l,i} x^m + \sum_{k = 1}^{m-1} p_{i,k} x^k
          \end{align*}

          \item For $k \in \{0, \dots, n-1\}$ generate the randomness $s_k \sample random$
          compute the commitments
          \begin{align*}
            c_{d_k} &= \sum_{i = 0}^{M-1} p_{i,k} \cdot c_i + \com(\ck, 0; s_k)
          \end{align*}
          \item Send $\pi_1 = (\{c_{\ell_j}\}_{j = 1}^m, \{c_{a_j}\}_{j = 1}^m,\{c_{b_j}\}_{j = 1}^m,\{c_{d_k}\}_{k = 0}^{m-1})$ to the verifier.
        \end{enumerate}





        \item[$\Prove_2(\PPP, (c_1, \dots, c_M), (l, r), d)$]: This algorithm is the second part of the prove procedure, the prover received the challenge $d \in \R_2$ from the verifier.
        \begin{enumerate}
          \item For $i \in \{1, \dots, m\}$
          \begin{enumerate}
            \item $f_j = \ell_j \cdot d + a_j$. Notice that $f_j$ are all elements in $\R_2$.
            \item $z_{a_j} = r_j \cdot d + r^a_j$. Notice that the operation on the randomness are component wise operation.
            \item $z_{b_j} = r_j \cdot (f_j - d) + r^b_j$. Remark that this is operations in $\R_q$.
          \end{enumerate}
          \item Compute $z_d = r \cdot d^m - \sum_{k = 0}^{m-1}s_k x^k$
          \item Send $\pi_2 = (\{f_j\}_{j = 1}^m, \{z_{a_j}\}_{j = 1}^m, \{z_{b_j}\}_{j = 1}^m, z_d)$.
        \end{enumerate}



        \item [$\Verif(\PPP, \pi_1, \pi_2) \to \{\True, \False\}$]:
        \begin{enumerate}
          \item Parse $\pi_1 = (\{c_{\ell_j}\}_{j = 1}^m, \{c_{a_j}\}_{j = 1}^m,\{c_{b_j}\}_{j = 1}^m,\{c_{d_k}\}_{k = 0}^{m-1})$.
          \item Parse $\pi_2 = (\{f_j\}_{j = 1}^m, \{z_{a_j}\}_{j = 1}^m, \{z_{b_j}\}_{j = 1}^m, z_d)$.
          \item Denote $B_{BV}$ and $B_{BV}'$ respectively the $B$-bound of the Gaussian distribution $\mathcal{D}_{(R_q, \sigma_{BV})}$ and $\mathcal{D}_{(R_q, \sigma'_{BV})}$ and $n$ be the degree of the ring $\R_q$.
          \item Recall that $z_{a_j} = (v_{a_j}, e'_{a_j}, e''_{a_j})$ and $z_{b_j} = (v_{b_j}, e'_{b_j}, e''_{b_j})$.
          \item Verify that for all $j \in \{1, \dots m\}$:
          \begin{align*}
            v_{a_j} &\leq (n+1) B & e'_{a_j} &\leq (n+1) B & e''_{a_j} &\leq (n+1) B'\\
            v_{b_j} &\leq (2n+1) B & e'_{b_j} &\leq (2n+1) B & e''_{b_j} &\leq (2n+1) B'
          \end{align*}
          \item Then we prove that every for all $j \in \{1, \dots, m\}$, $c_{\ell_j}$ is committed to one bit:
          \begin{align*}
            d \cdot c_{\ell_j} + c_{a_j} &= \Com(\ck, f_j ; z_{a_j})\\
            (d-f_{j}) \cdot c_{\ell_j} + c_{b_j} &= \Com(\ck, 0; z_{b_j})\\
            \sum_{i = 0}^{M-1}(\prod_{j = 1}^{m} f_{j,i_j} \cdot c_i) - \sum_{k = 0}^{m-1} d^{k} \cdot c_{d_{k}} & = \Com(\ck, 0; z_d)
          \end{align*}
        \end{enumerate}
    \end{description}
  \end{subsection}

  \begin{subsection}{Construction of the ring signature}

    This construction of the ring signature is direct as mentioned in~\cite{DBLP:conf/eurocrypt/GrothK15} using the Fiat-Shamir transformation to obtain a NIZK from the previous $\Sigma$-protocol.
    \begin{description}
      \item [$\Setup(1^\lambda)$]:
      \begin{enumerate}
        \item Generate the commitment key $\ck \gets \Com.\Setup(1^\lambda)$.
        \item Generate the hash function $H \gets \mathcal{H}(1^\lambda)$.
        \item Output $\PPP = (\ck, H)$.
      \end{enumerate}
      \item [$\KeyGen(\PPP)$]:
      \begin{enumerate}
        \item Generate the randomness used in the underlying commitment scheme $r \gets Random$.
        \item Compute the commitment $c = \com(\ck, 0; r)$.
        \item Output $(\vk, \sk) = (c, r)$.
      \end{enumerate}
      \item [$\Sig(\PPP, \sk, \mu, R)$]: The sign algorithm takes the public parameter $\PPP$, the signing key $\sk$, the message $\mu$ and the group ring $R$.
      \begin{enumerate}
        \item Parse $R = (c_0, \dots, c_{M-1})$ and $\sk = r$.
        \item Using the signing key $\sk = r$ to identify a commitment $c_{\ell} = \Com(\ck, 0; r)$ with $\ell \in \{0, \dots, M-1\}$.
        \item Generate the first part of the proof of the $\Sigma$-protocol, $\pi_1 = \Prove(\ck, (c_0, \dots, c_{M-1}), (\ell, \sk))$.
        \item Compute the hash $d = H(\ck, \mu, R, \pi_1) \in \R_2$.
        \item Using the hash $d$ as the challenge of the $\Sigma$-protocol to produce $\pi_2 \gets \Prove_2(\PPP, (c_1, \dots, c_M), (l, r), d)$.
        \item Output $\sigma = (\pi_1, \pi_2)$.
      \end{enumerate}
      \item [$\Verif(\PPP, \mu, R, \sigma)$]:
      \begin{enumerate}
        \item Parse $R = (c_0, dots, c_{M-1})$.
        \item Parse $\sigma = (\pi, \pi_2)$.
        \item Compute the hash $d = H(\ck, \mu, R, \pi_1)$.
        \item Output $\Verif (\ck, (c_0, \dots, c_{M-1}), \pi_1, d, \pi_2)$.
      \end{enumerate}
    \end{description}

  \end{subsection}
\end{section}

\begin{section}{Applicaion: Electronic voting system}
  In this section, we introduce the lattice based electronic voting system which use our DV-NIZK.
  This is an instantiation of the voting scheme introduced in~\cite{DBLP:conf/pkc/ChaidosG15}.

  \begin{definition}{Voting Scheme}
    A voting scheme $\mathcal{V}$ consists of five algorithm $(\mathsf{Setup}, \mathsf{Vote}, \mathsf{SubmitBallot}, \mathsf{ChechBoard}, \mathsf{Tally})$.
    \todo{Add more specific description of the voting scheme}
  \end{definition}


  \begin{description}
    \item[$\mathsf{Setup}(1^\lambda) \to ( \PPP, \SK, \VK)$]:
    \begin{enumerate}
      \item Generate the keys for the underlying PLWE encryption scheme $(\PK_{PLWE}, \SK_{PLWE}) \gets PLWE.\KeyGen(1^\lambda)$.
      \item Generate the keys for the underlying DV-NIZK scheme $(\PK_{DV}, \VK_{DV})$.
      \item Initiate the bulletin board $BB$.
      \item Set
      \begin{align*}
        \PPP &= (\PK_{PLWE}, \PK_{DV}) & \SK &= \SK_{PLWE} & \VK &= \VK_{DV}
      \end{align*}
    \end{enumerate}
    \item[$\mathsf{Vote}(\PPP, m) \to B$]:
    \begin{enumerate}
      \item Parse $\PPP = (\PK_{PLWE}, \PK_{DV})$.
      \item Sample random polynomials $r \gets Random$ \wrt the PLWE encryption scheme.
      \item Compute $c = PLWE.\Enc(\PK_{PLWE}, m; r)$.
      \item Generate a DV-NIZK proof $\pi \gets \Prove(\PK_{DV}, ((\PK_{PLWE}, c), (m, r)))$ of the fact that $m \in \{0,1\}$ and $c = PLWE.\Enc(\PK_{PLWE}, m; r)$.
      \item Output the ballot $B = (c, \pi)$
    \end{enumerate}
    \item[$\mathsf{SubmitBallot}(B. BB)$]: $BB \gets (BB, B)$.
    \item[$\mathsf{CheckBoard}(BB, \VK_{DV})$]:
    \begin{enumerate}
      \item Use the private verification key $\VK_{DV}$ to check all the ballot in the bulletin board $BB$.
      \item Remove all the ballot invalid.
    \end{enumerate}
    \item[$\mathsf{Tally}(BB, \SK)$]:
    \begin{enumerate}
      \item Using the additive homomorphic property of the PLWE encryption. We compute $c_{final} = \sum_{i = 1}^k c_i$.
      \item Submit the final ciphertext $c_{final}$ to the decryption party.
      \item The decryption party use $\SK = \SK_{PLWE}$ to get the result $result = PLWE.\Dec(\SK_{PLWE}, c_{final})$.
    \end{enumerate}
  \end{description}
\end{section}

\begin{section}{New Cramer-Shoup type lattice based CCA encryption}
  \todo{this one maybe impossible because this is hash proof system approach which
  use the statistic argument}
\end{section}
