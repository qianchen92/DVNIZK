\begin{section}{$\Sigma$-protocol to prove an encryption of $0$ or $1$~\cite{DBLP:conf/pkc/ChaidosG15}~\label{sigma}}

  Here we recall the construction proposed by~\cite{DBLP:conf/pkc/ChaidosG15} to prove that an encryption $\Enc(\PK,\mu; r)$ encrypts the message $\mu \in \{0,1\}$.
  We denote $\mathcal{M}$ the plaintext space, $\mathcal{C}$ the challenge space and $\mathcal{Q}$ the randomness space.
  For an encryption scheme $\Pi = (\Setup, \Enc, \Dec)$,
  in order to prove the fact that $c$ is a ciphertext of the plaintext $m$ \wrt the randomness $r \in \mathcal{Q}$ and $m \in \{0,1\}$,
  we use the following $\Sigma$-protocol.

  \begin{description}
    \item[$\Prove_1((\PK, c), (\mu, r))$]:
    \begin{enumerate}
      \item Uniformly choose a message $\mu_a \in \mathcal{M} \{0\}$, randomnesses $r_a, r_b \in \mathcal{Q}$.
      \item Compute the encryptions $c_a \gets \Enc(\PK, \mu_a, r_a)$ and $c_b \gets \Enc(\PK, \mu_b, r_b)$.
      \item Send the ciphertexts $c_a, c_b$ to the verifier.
    \end{enumerate}
    \item[$\Prove_1((\PK, c), (\mu, r), d)$]:
    \begin{enumerate}
      \item The prover receive the challenge $d$ from the verifier.
      \item Compute the ciphertext of $f = d \cdot m + m_a$.
      \item Compute the linear combination of the corresponded randomness
      \begin{align*}
        z_a &= d r + r_a  & z_b &= (f-d) \cdot r + r_b.
      \end{align*}
    \end{enumerate}
  \item[$\Verif(c)$]
    \begin{enumerate}
    \item After receiving $(c_a, c_b)$, randomly sample a challenge $ d \sample \mathcal{C}$ and send $d$ to the prover.
    \item Verify that $z_a, z_b$ are all in the randomness space.
    \item Verify that $f$ is in the message space.
    \item Verify the following two equations:
      \begin{align*}
        f \cdot c + c_a &= \Enc(\PK, f; z_a) &  (f-d) \cdot c + c_b &= \Enc(\PK, 0; z_b)
      \end{align*}
    \end{enumerate}
  \end{description}



\end{section}

\begin{section}{$\Sigma$-protocol to prove an encryption of $0$~\cite{DBLP:conf/pkc/ChaidosG15}~\label{sigma0}}
  Here we want to prove a ciphertext is an encryption of $0$.
  For an encryption scheme $\Pi = (\Setup, \Enc, \Dec)$,
  in order to prove the fact that $c$ is a ciphertext of the plaintext $m$ \wrt the randomness $r \in \mathcal{Q}$ and $m  = 0$,
  we use the following $\Sigma$-protocol.

  \begin{description}
    \item[$\Prove_1((\PK, c), (\mu, r))$]:
    \begin{enumerate}
      \item Uniformly choose a message $\mu_a \in \mathcal{M} \{0\}$, randomnesses $r_a, r_b \in \mathcal{Q}$.
      \item Compute the encryptions $c_a \gets \Enc(\PK, \mu_a, r_a)$ and $c_b \gets \Enc(\PK, \mu_b, r_b)$.
      \item Send the ciphertexts $c_a, c_b$ to the verifier.
    \end{enumerate}
    \item[$\Prove_1((\PK, c), (\mu, r), d)$]:
    \begin{enumerate}
      \item The prover receive the challenge $d$ from the verifier.
      \item Compute the ciphertext of $f = d \cdot \mu + \mu_a$.
      \item Compute the linear combination of the corresponded randomness
      \begin{align*}
        z_a &= d r + r_a  & z_b &= f \cdot r + r_b.
      \end{align*}
    \end{enumerate}
  \item[$\Verif(c)$]
    \begin{enumerate}
    \item After receiving $(c_a, c_b)$, randomly sample a challenge $ d \sample \mathcal{C}$ and send $d$ to the prover.
    \item Verify that $z_a, z_b$ are all in the randomness space.
    \item Verify that $f$ is in the message space.
    \item Verify the following two equations:
      \begin{align*}
        f \cdot c + c_a &= \Enc(\PK, f; z_a) &  f \cdot c + c_b &= \Enc(\PK, 0; z_b)
      \end{align*}
    \end{enumerate}
  \end{description}



\end{section}


\begin{section}{Ring-LWE based ring-signature}

  In this section, we give an application of the


\end{section}
