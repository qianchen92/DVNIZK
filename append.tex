\begin{section}{$\Sigma$-protocol to prove an encryption of $0$ or $1$~\cite{DBLP:conf/pkc/ChaidosG15}~\label{sigma}}

  Here we recall the construction proposed by~\cite{DBLP:conf/pkc/ChaidosG15} to prove that an encryption $\Enc(\PK,\mu; r)$ encrypts the message $\mu \in \{0,1\}$.
  We denote $\mathcal{M}$ the plaintext space, $\mathcal{C}$ the challenge space and $\mathcal{Q}$ the randomness space.
  For an encryption scheme $\Pi = (\Setup, \Enc, \Dec)$,
  in order to prove the fact that $c$ is a ciphertext of the plaintext $m$ \wrt the randomness $r \in \mathcal{Q}$ and $m \in \{0,1\}$,
  we use the following $\Sigma$-protocol.

  \begin{description}
    \item[$\Prove_1((\PK, c), (\mu, r))$]:
    \begin{enumerate}
      \item Uniformly choose a message $\mu_a \in \mathcal{M} \{0\}$, randomnesses $r_a, r_b \in \mathcal{Q}$.
      \item Compute the encryptions $c_a \gets \Enc(\PK, \mu_a, r_a)$ and $c_b \gets \Enc(\PK, \mu_b, r_b)$.
      \item Send the ciphertexts $c_a, c_b$ to the verifier.
    \end{enumerate}
    \item[$\Prove_1((\PK, c), (\mu, r), d)$]:
    \begin{enumerate}
      \item The prover receive the challenge $d$ from the verifier.
      \item Compute the ciphertext of $f = d \cdot m + m_a$.
      \item Compute the linear combination of the corresponded randomness
      \begin{align*}
        z_a &= d r + r_a  & z_b &= (f-d) \cdot r + r_b.
      \end{align*}
    \end{enumerate}
  \item[$\Verif(c)$]
    \begin{enumerate}
    \item After receiving $(c_a, c_b)$, randomly sample a challenge $ d \sample \mathcal{C}$ and send $d$ to the prover.
    \item Verify that $z_a, z_b$ are all in the randomness space.
    \item Verify that $f$ is in the message space.
    \item Verify the following two equations:
      \begin{align*}
        f \cdot c + c_a &= \Enc(\PK, f; z_a) &  (f-d) \cdot c + c_b &= \Enc(\PK, 0; z_b)
      \end{align*}
    \end{enumerate}
  \end{description}



\end{section}

\begin{section}{$\Sigma$-protocol to prove an encryption of $0$~\cite{DBLP:conf/pkc/ChaidosG15}~\label{sigma0}}
  Here we want to prove a ciphertext is an encryption of $0$.
  For an encryption scheme $\Pi = (\Setup, \Enc, \Dec)$,
  in order to prove the fact that $c$ is a ciphertext of the plaintext $m$ \wrt the randomness $r \in \mathcal{Q}$ and $m  = 0$,
  we use the following $\Sigma$-protocol.

  \begin{description}
    \item[$\Prove_1((\PK, c), (\mu, r))$]:
    \begin{enumerate}
      \item Uniformly choose a message $\mu_a \in \mathcal{M} \{0\}$, randomnesses $r_a, r_b \in \mathcal{Q}$.
      \item Compute the encryptions $c_a \gets \Enc(\PK, \mu_a, r_a)$ and $c_b \gets \Enc(\PK, \mu_b, r_b)$.
      \item Send the ciphertexts $c_a, c_b$ to the verifier.
    \end{enumerate}
    \item[$\Prove_1((\PK, c), (\mu, r), d)$]:
    \begin{enumerate}
      \item The prover receive the challenge $d$ from the verifier.
      \item Compute the ciphertext of $f = d \cdot \mu + \mu_a$.
      \item Compute the linear combination of the corresponded randomness
      \begin{align*}
        z_a &= d r + r_a  & z_b &= f \cdot r + r_b.
      \end{align*}
    \end{enumerate}
  \item[$\Verif(c)$]
    \begin{enumerate}
    \item After receiving $(c_a, c_b)$, randomly sample a challenge $ d \sample \mathcal{C}$ and send $d$ to the prover.
    \item Verify that $z_a, z_b$ are all in the randomness space.
    \item Verify that $f$ is in the message space.
    \item Verify the following two equations:
      \begin{align*}
        f \cdot c + c_a &= \Enc(\PK, f; z_a) &  f \cdot c + c_b &= \Enc(\PK, 0; z_b)
      \end{align*}
    \end{enumerate}
  \end{description}



\end{section}



\begin{section}{$\Sigma$-protocol to prove two ciphertexts have the same plaintext}
  We will give a $\Sigma$-protocol in order to prove the two ciphertexts using two
  different keys $(\SK_1, \PK_1)$ and $(\SK_2, \PK_2)$ have the same plaintext.

  For two pairs of keys of the $PLWE$ encryption
  \begin{align*}
    \SK_1 &= s_1, & \PK_1 &= \begin{bmatrix} a_1 \\ b_1\end{bmatrix} & \SK_2 &= s_2,  & \PK_2 &= \begin{bmatrix} a_2 \\ b_2\end{bmatrix},
  \end{align*}
  the following $\Sigma$-protocol prove the fact that $ C_1 = PLWE.\Enc(\PK_1, m; r_1)$ and $C_2 = PLWE.\Enc(\PK_2, m; r_2)$
  for the same message $m$.

  \begin{description}
    \item [$\Setup(1^\lambda)$]:
    \begin{enumerate}
      \item Generate the public parameter for the underlying PLWE and RGSW encryption scheme
      \begin{enumerate}
        \item Sample a polynomial $s_{RGSW} \sample \mathcal{D}_{\R_q, \sigma_{(k,RGSW)}}$
        \item Uniformly sample $a_{RGSW} \sample \R_q$ and a noise polynomial $e_{RGSW} \sample \mathcal{D}_{\R_q, \sigma_{(k, RGSW)}}$.
        \item Compute $b_{RGSW} = s_{RGSW} \cdot a_{RGSW} + e_{RGSW} \in \R_q$
        \item Set secret key and public key of the underlying Ring-GSW encryption scheme:
          \begin{align*}
            \vec{s}_{RGSW} &= \begin{bmatrix} 1 \\ -s_{RGSW} \end{bmatrix}  & \vec{a}_{RGSW} &=  \begin{bmatrix} b_{RGSW} \\ a_{RGSW}\end{bmatrix}
          \end{align*}
        \item Uniformly sample $a_d \sample \R_q$.
        \item Sample from the discrete Gaussian distribution $s_d \sample \mathcal{D}_{R_q, \sigma_{PLWE}}$ and $e_d \sample \mathcal{D}_{R_q, \sigma_{PLWE}}$.
        \item Compute the public key and secret key of the underlying PLWE encryption scheme.
        \begin{align*}
          \SK_{PLWE} &= s_d & \PK_{PLWE} &= (a_d, b_d = a_d s_d + e_d)
        \end{align*}
      \end{enumerate}
      \item Encrypt the message with the two underlying encryption schemes:
        \begin{enumerate}
        \item Sample a $(N \times 1)-$matrix $\mat{R}_{RGSW} \sample (\R_q \cap\{0,1\})^{N \times 1}$.
        \item Sample a $(N \times 2)-$noise matrix $\mat{E}_{RGSW} \sample \mathcal{D}_{(\R_q^{N \times 2}, \sigma_{(c, RGSW)})}$.
        \item Sample polynomials $v,e' \sample \mathcal{D}_{R_q, \sigma_{BV}}$ and $e'' \sample  \mathcal{D}_{R_q, \sigma_{BV}'}$.
        \item Compute the following ciphertexts
          \begin{align*}
            \mat{C}_d^{RGSW} &= \vec{a}_{RGSW} \cdot \mat{R}_{RGSW} + \mat{E}_{RSGW} +  d \cdot \mat{G} &
            \vec{c}_d &= \begin{bmatrix} b_d v + 2e'' + \mu \\ a_d v + 2e' \end{bmatrix}
          \end{align*}
        \end{enumerate}
        \item Output the public parameter and the secrete verification key of the DVNIZK proof system
        \begin{align*}
          \PPP &= (\vec{a}_{RGSW}, a_d, \vec{c}_d, \vec{C}_d^{RGSW}) & \SK &= (\vec{s}_{RGSW}, s_d)
        \end{align*}
      \end{enumerate}
      \item [$\Prove(\PPP, \vec{C}_1, \vec{C_2}, \vec{r}_1, \vec{r}_2, m, \PK_1, \PK_2)$]:
      This algorithm proves that $\vec{C}_1$ and $\vec{C}_2$ are PLWE ciphertext of the same message $m$
      \wrt the different public keys:
      \begin{align*}
        \vec{C}_1 &= PLWE.\Enc(\PK_1, m; \vec{r}_1) &  \vec{C}_2 &= PLWE.\Enc(\PK_2, m; \vec{r}_2)
      \end{align*}
      \begin{enumerate}
        \item Parse the public keys and the randomness used in the PLWE ciphertext.
        \begin{align*}
          \PK_1 &= (a_1, b_1) \in \R_q^2 & \PK_2 &= (a_2, b_2) \in \R_q^2 \\
          \vec{r}_1 &= (v_1, e_1', e_1'') \in R_q^3 & \vec{r}_2 &= (v_2, e_2', e_2'') \in \R_q^3
        \end{align*}
        \item niformly sample a random message $\mu_a \sample \R_2$.
        \item Sample the randomnesses
        \begin{align*}
          (v_{a,1}, e_{a,1}', e_{a,1}'') &\sample \mathcal{D}^2_{\R_q, \sigma_{DV'}} \times \mathcal{D}_{\R_q, \sigma_{DV''}}\\
          (v_{a,2}, e_{a,2}', e_{a,2}'') &\sample \mathcal{D}_{\R_q, \sigma_{DV'}}^2 \times \mathcal{D}_{\R_q, \sigma_{DV''}}
        \end{align*}
        \item Encrypt $\mu_a$ using the public key $\PK_1$ and $\PK_2$
        \begin{align*}
          \vec{C}_{a}^{1} &=
          \begin{bmatrix} b_1 \cdot v_{a,1} + 2 e_{a,1}''+ \mu_a  \\
            a_1 \cdot v_{a,1} + 2 \cdot e_{a,1}' \end{bmatrix} &
            \vec{C}_{a}^{2} &= \begin{bmatrix} b_2 \cdot v_{a,2} + 2 e_{a,2}''+ \mu_a \\
            a_2 \cdot v_{a,2} + 2 \cdot e_{a,2}' \end{bmatrix}
        \end{align*}
        \item Homomorphically compute the following ciphertexts
        \begin{align*}
          \vec{c}^1_f =  
        \end{align*}
      \end{enumerate}
  \end{description}
\end{section}

\begin{section}{Ring-LWE based ring-signature}

  In this section, we give an application of the


\end{section}
