\begin{theorem}
  The algorithm $(\Setup, \Prove, \Verif)$ specified in the previous section is a Designed-Verifier Non-Interactive Zero-Knowledge proof system
  which verifies adaptive culpable soundness \wrt the guilt relation $\mathcal{R}_{guilt}$
  \begin{align*}
    \mathcal{R}_{guilt} &= \{(\vec{c}, \PK) | \exists \SK~\st~\Dec(\SK, \vec{c}) \not \in \{0,1\}  \wedge {\mathsf{VerifyKey}}(\PK, \SK)  = \True\}.
  \end{align*}
\end{theorem}

\begin{proof}
  We will prove the correctness and the zero-knowledge of a such protocol.
  \begin{description}
  \item[\textsf{Correctness}]: The correctness of this proof is omitted for the moment.
    \todo{Give a formal proof, i know it's really easy}
  \item[\textsf{Zero-Knowledge}]:
    The simulator $\mathcal{S}_{\Setup}$ does exactly the same as the setup algorithm $\Setup$ except that $\mathcal{S}_{\Setup}$ also output $d$ as the simulation key $\SK_{sim}$.

    The simulator of the proof generation algorithm $\mathcal{S}_{\Prove}$ works as follows:
    \begin{description}
    \item[$\mathcal{S}_{\Prove}(\PPP, \vec{c}, \SK_{sim})$]:
      \begin{enumerate}
      \item Parse $\SK_{sim} = d$.
      \item Sample random polynomials $f \sample \R_2$, $v,v_a, v_b, e, e_a, e_b \sample \mathcal{D}_{(\R_q, \sigma_{BV}')}$ and $e'', e_a'', e_b'' \sample \mathcal{D}_{(\R_q, \sigma_{BV}'')}$.
      \item Compute
        \begin{align*}
          v_f &= v \cdot d + v_a & e_f' &= e' \cdot d + e_a' & e_f'' &= e'' \cdot d + e_a''\\
          v_0 &= v \cdot d + v_b & e_0' &= e' \cdot d + e_b' & e_0'' &= e'' \cdot d + e_b''\\
          \vec{c}_a &= PLWE.\Enc(f, (v_f, e_f', e_f'')) - \vec{c} \cdot d \\
          \vec{c}_b &= PLWE.\Enc(0, (v_0, e_0', e_0'')) - \vec{c} \cdot (f - d)
        \end{align*}
      \item Similarly compute the ciphertexts for the randomness
        \begin{align*}
          \mat{C}_{r_f^0} &= \mat{C}_d^{RGSW} \cdot \mat{G}^{-1}(v \cdot \mat{G}) + v_a \cdot \mat{G} \\
          \mat{C}_{r_f^1} &= \mat{C}_d^{RGSW} \cdot \mat{G}^{-1}(e' \cdot \mat{G}) + e_a' \cdot \mat{G} \\
          \mat{C}_{r_f^2} &= \mat{C}_d^{RGSW} \cdot \mat{G}^{-1}(e'' \cdot \mat{G}) + e_a'' \cdot \mat{G} \\
          \mat{C}_{r_0^0} &= (\mat{C}_d^{RGSW} \cdot \mat{G}^{-1}(\mu \cdot \mat{G}) + \mu_a \cdot \mat{G} - \mat{C}_d^{RGSW}) \cdot \mat{G}^{-1}(v \cdot \mat{G}) + v_b \cdot \mat{G} \\
          \mat{C}_{r_0^1} &= (\mat{C}_d^{RGSW} \cdot \mat{G}^{-1}(\mu \cdot \mat{G}) + \mu_a \cdot \mat{G} - \mat{C}_d^{RGSW}) \cdot \mat{G}^{-1}(e' \cdot \mat{G}) + e_b' \cdot \mat{G} \\
          \mat{C}_{r_0^2} &= (\mat{C}_d^{RGSW} \cdot \mat{G}^{-1}(\mu \cdot \mat{G}) + \mu_a \cdot \mat{G} - \mat{C}_d^{RGSW}) \cdot \mat{G}^{-1}(e'' \cdot \mat{G}) + e_b'' \cdot \mat{G}
        \end{align*}
      \item Output the simulated proof $\pi_{sim} = (\vec{c}_a, \vec{c}_b, \vec{c}_f, \mat{C}_{r_f^0}, \mat{C}_{r_f^1}, \mat{C}_{r_f^2}, \mat{C}_{r_0^0}, \mat{C}_{r_0^1}, \mat{C}_{r_0^2})$
      \end{enumerate}
    \end{description}

    Since for an honestly generated proof $f = d \cdot \mu + \mu_a$ where $\mu, \mu_a \sample \R_2$, $f$ is uniformly distributed in $\R_2$,
    the $f$ in the simulated proof has the same distribution and the randoness used in the encryptions are sampled from exactly the same distribution as in the original proof.
    Thus the distributions of $\pi_{sim}$ and an honestly generated proof are perfectly indistinguishable and such a proof simulated will be accepted by the verifier.
    
  \item[\textsf{Adaptive culpable soundness}]:
    We can prove our construction is culpable sound \wrt the relation $\mathcal{R}_{guilt}$ as follows
    \begin{align*}
      \mathcal{R}_{guilt} &= \{(\vec{c}, \PK) | \exists \SK~\st~\Dec(\SK, \vec{c}) \not \in \{0,1\}  \wedge {\mathsf{VerifyKey}}(\PK, \SK)  = \True\}
    \end{align*}

    We intantiate the definition of the culpable soundness of the construction:
    \begin{align*}
      \Pr[(\PPP, \SK) \gets \Setup(1^{\lambda}); (\vec{c}, \pi) \gets \Adv(\PPP) | (\PK, \vec{c}) \in \mathcal{R}_{guilt} \wedge \Verif(\SK, x, \pi) = \True]
    \end{align*}

    We provide a game based proof:
    \begin{description}
    \item[\textsf{Game} $0$]: The security in this game is the original security game of the definition of culpable soundness.
      \begin{enumerate}
      \item Parse $\pi$ as $(\vec{c}_a, \vec{c}_b, \vec{c}_f, \mat{C}_{r_f^0}, \mat{C}_{r_f^1}, \mat{C}_{r_f^2}, \mat{C}_{r_0^0}, \mat{C}_{r_0^1}, \mat{C}_{r_0^2})$.
      \item If the proof can verifies all the equations, it means that if we note
        \begin{align*}
          \mu_a &= PLWE.\Dec(\SK_{PLWE}, \vec{c}_a) & \mu_b &= PLWE.\Dec(\SK_{PLWE}, \vec{c}_b) \\
          f &= PLWE.\Dec(\SK_{PLWE}, \vec{c}_f)
        \end{align*}
        the underlying plaintext verifies the following equations:
        \begin{align*}
          f &= d \cdot \mu + \mu_a & (f - d) \cdot \mu + \mu_b = 0
        \end{align*}
      \item Thus we have $\mu_b = - d \cdot \mu^2 + d \cdot \mu - \mu_a \cdot \mu$.
      \item We note $Adv_0(\Adv)$ the advantage of adversary in this game.
      \end{enumerate}

    \item[\textsf{Game} $1$]: In this game, we do the same things as in the previous game except that:
      \begin{enumerate}
      \item We uniformly sample $\R_2$-polynomial $\tilde{d} \sample \R_2$.
      \item In the $\Setup$ algorithm, we replace $\vec{c}_d$ and $\mat{C}_d^{RGSW}$ by the encryptions $\vec{c}_{\tilde{d}}$ and $\mat{C}_{\tilde{d}}$ of $\tilde{d}$.
        This change is indistinguishable for the adversary since the distributions of $d$ and $\tilde{d}$ are exactly the same.
      \item As pointed out in the previous game, we have similarly $\mu_b = - \tilde{d} \cdot \mu^2 + \tilde{d} \cdot \mu - \mu_a \cdot \mu$.
      \item Since the only difference from the previous game is the change of the plaintext inside of an encryption IND-CPA.
        The advantage of $Adv_1(\Adv)$ adversary in \textsf{Game} $1$ verifies:
        \begin{align*}
          Adv_0(\Adv) &\leq Adv_1(\Adv) + \varepsilon_{IND-CPA}
        \end{align*}
      \end{enumerate}
    \end{description}

    \begin{lemma}
      The advantage $Adv_1(\Adv)$ in \textsf{Game} $1$ is at most $2^{-n}$.
    \end{lemma}
    \begin{proof}
      As showed in \textsf{Game} $0$, in order to verifier all the equations we have
      \begin{align*}
        \mu_b &= - d \cdot \mu^2 + d \cdot \mu - \mu_a \cdot \mu & \mu \not \in \{0,1\}
      \end{align*}
      This relation should remains true for $\tilde{d}$ in \textsf{Game} $1$ if the adversary win in \textsf{Game} 1:
      \begin{align*}
        \mu_b &= - \tilde{d} \cdot \mu^2 + \tilde{d} \cdot \mu - \mu_a \cdot \mu &\mu \not \in \{0,1\}
      \end{align*}
      which means $d = \tilde{d}$.
      But $\tilde{d}$ is uniformly chosen from $\R_2$, thus the advantage of the adversary is bounded by:
      \begin{align*}
        Adv_1(\Adv) &\leq \frac{1}{|\R_2|}\\
        &= 2^{-n}
      \end{align*}
      
    \end{proof}

    
  \end{description}

\end{proof}

