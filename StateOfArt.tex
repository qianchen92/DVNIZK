\paragraph{DV-NIZK.}
From the introduction of non-interactive zero-knowledge proof system by Blum, Feldman and Mical~\cite{DBLP:conf/stoc/BlumFM88},
many efforts had been investigated into its efficiency.
The breakthrough of pairing based efficient NIZK~\cite{DBLP:conf/eurocrypt/GrothS08} bring this primitive to the practical construction in the pairing based protocols.

Despite their efficiency, GS proof system can only prove pairing based system, it is very hard to construct NIZK based on other assumptions.
Existing NIZK proof systems rely on the Fiat-Shamir transformation with random oracle model which is not very efficient.
On the other hand, when we consider the interactive proof system,
there are many very efficient $\Sigma$-protocol (a particular type of 3-move honest verifier zero-knowledge proof) based on various assumptions~\cite{DBLP:journals/iacr/BaumDOP16}\cite{DBLP:conf/crypto/BaumDLN16}.

In a special family of NIZK, designated-verifier non-interactive zero-knowledge proof system is a proof system which can only be verified with a secret verification key
and the soundness of the proof system holds only for the adversary who doesn't have the secret verification key.
DV-NIZK has many applications in the construction of more complex cryptographic protocols such as
tightly CCA encryption schemes~\cite{DBLP:conf/eurocrypt/GayHKW16} and in the electronic voting system~\cite{DBLP:conf/pkc/ChaidosG15}.


Damg\r{a}rd, Fazio and Nicolosi (DFN)~\cite{DBLP:conf/tcc/DamgardFN06} firstly established the connection between $\Sigma$-protocol and DV-NIZK via homomorphic encryption schemes,
this technique is later on developed by Chaidos and Groth~\cite{DBLP:conf/pkc/ChaidosG15}.
They use this idea to construct an efficient designated-verifier zero-knowledge proof system based on Okamoto-Uchiyama encryption scheme~\cite{DBLP:conf/eurocrypt/OkamotoU98}.
The main open problem was how to extend their idea to construct more DV-NIZK based on other assumptions.
