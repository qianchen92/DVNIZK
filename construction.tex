%\begin{subsection}{DV-NIZK of an encryption of $0$ or $1$ under LWE}
%    
%    \begin{description}
%    \item[$\KeyGen(1^\lambda)$]:
%      \begin{enumerate}
%      \item Set the LWE parameters:
%        \begin{enumerate}
%        \item Choose a modulus $q$ prime of $\kappa = \kappa(\lambda)$ bits, the lattice dimension parameter $n = n(\lambda)$ and the error distribution $\chi = \chi(\lambda)$ to ensure at least $\lambda$ bits security against the LWE problem.
%        \item Compute $m = m(\lambda) = O(n~log~q)$, $\ell = \floor{q}+1$ and $N = (n+1) \cdot \ell$.
%        \end{enumerate}
%      \item Generate keys for the underlying GSW-FHE scheme:
%        \begin{enumerate}
%        \item Sample $\vec{t}_V \sample \mathbb{Z}_q^m$ and set $\vec{s}_V = (1, -t_1, \dots, -t_n) \in \mathbb{Z}_q^{n+1}$.
%        \item Sample a uniformaly random matrix $\mat{B}_V \gets \mathbb{Z}_q^{n \times m}$ and a noise vector $\vec{e}_V \gets \chi^m$.
%        \item Compute $\vec{b}_V = \vec{t}_V \cdot \mat{B}_V  + \vec{e}_V$.
%        \item Set the matrix $\mat{A}_V = \begin{bmatrix}\vec{b}_V \\ \mat{B}_V \end{bmatrix} \in \mathbb{Z}_q^{(n+1) \times m}$, remark that $\vec{s}_V \cdot \mat{A}_V = \vec{e}_V$.
%        \end{enumerate}
%      \item Generate keys for the underlying Matrix-FHE scheme:
%        \begin{enumerate}
%        \item Choose a noise matrix $\mat{E}_{MFHE} \sample \chi^{r \times m}$.
%        \item Let $\mat{S}_{MFHE}:= [\mat{I}_r || - \mat{S}_{MFHE}'] \in \mathbb{Z}_q^{r \times (n+r)}$.
%        \item Set the matrix:
%          \begin{align*}
%            \mat{A}_{MFHE} &:= \begin{bmatrix}\mat{S}_{MFHE}'\mat{B}_{MFHE} + \mat{E}_{MFHE} \\ \mat{B}_{MFHE}\end{bmatrix} \in \mathbb{Z}_q^{(n+r) \times m}
%          \end{align*}
%        \item Denote $\mat{E}_{i,j} \in \{0, 1\}^{r\times r}$ with $(i,j) \in \{1, \dots ,r\}^2$.
%        \item Sample $\mat{R}_{i,j} \sample \{0,1\}^{m \times N}$ and
%          \begin{align*}
%            \mat{P}_{i,j} &:= \mat{A}_{MFHE} \mat{R}_{i,j} + \begin{bmatrix}\mat{M}_{i,j}\mat{S}_{MFHE}\\ \mat{0}\end{bmatrix}\cdot  \mat{G}_{MFHE} \in \mathbb{Z}_q^{(n+r) \times N}.
%          \end{align*}
%        \item Set $\PK_{MFHE} := (\{\mat{P}_{i,j \in [r]}, \mat{A}_{MFHE}\})$ and $\SK_{MFHE} = \mat{S}_{MFHE}$.
%        \end{enumerate}
%      \item Generate the encryption of the challenge:
%        \begin{enumerate}
%        \item Choose a random element $d \in \mathbb{Z}_{q}$ and set $\mat{D} = d \cdot \mat{I}_r$.
%        \item Draw random matrix $\mat{R}_d \sample \{0,1\}^{m \times N}$ and $\mat{R}_{\mat{D}} \sample \{m \times N\}$.
%        \item Compute
%          \begin{align*}
%            \mat{C}_d &= \mat{A}_V \mat{R}_d + d \cdot \mat{G}\\
%            \mat{C}_{\mat{D}} &= \mat{A}_{MFHE} \mat{R}_{\mat{D}} + \sum_{i,j \in [r]: D[i,j] = 1} \mat{P}_{i,j} \in \mathbb{Z}_q^{(n+1) \times N}
%          \end{align*}
%        \end{enumerate}
%      \item Output $\PPP = (n, q, \chi, m)$, $\SK = \vec{s} \in \mathbb{Z}_q^{n+1}$ and $\PK = \mat{A} \in \mathbb{Z}_q^{m \times (n+1)}$.
%      \end{enumerate}
% 
% 
% 
%    \item[$\Prove((\PK, \mat{C}), (m, \mat{R}))$]:
%      We want to prove the following fact:
%      \begin{enumerate}
%      \item $\PK = \mat{A}_p \in \mathbb{Z}_q^{}$
%      \end{enumerate}
%      \begin{enumerate}
%      \item Choose $m_a \gets \{1\}||\{0,1\}^{\ell-2} \cap \mathbb{Z}_q$, $\mat{R}_a \gets \{0,1\}^{m \times N}$ and $\mat{R}_b \gets \{0,1\}^{m \times N}$.
%      \item Compute
%        \begin{align*}
%          \mat{C}_a &= \mat{A}_P \cdot \mat{R}_a + m_a \cdot \mat{G} \in \mathbb{Z}_q^{n \times N}\\
%          \mat{C}_b &= \mat{A}_P \cdot \mat{R}_b - \mu \cdot m_a \cdot \mat{G} \in \mathbb{Z}_q^{n \times N}          
%        \end{align*}
%      \item Sample $\mat{R}_{\mat{Z}_a}, \mat{R}_{\mat{Z}_b}  \sample \{0,1\}^{N\times N}$ 
%      \item Compute
%        \begin{align*}
%          \mat{C}_f &= \mat{C}_d \cdot \mat{G}^{-1}(m \cdot \mat{G}) + \mat{G}^{-1}(m_a \cdot \mat{G})\\
%          \mat{Z}_a &= (\mat{A}_{MFHE} \mat{R}_{\mat{Z}_a}  + \sum_{i,j\in [r]: \mat{R}[i,j] = 1} \mat{P}_{(i,j)}) \cdot \mat{G}^{-1}(\mat{C}_{\mat{D}}) \\
%          &+ \sum_{i,j \in [r]: \mat{R}_a[i,j]= 1}\mat{P}_{(i,j)}\\
%          \mat{Z}_b &= (\mat{A}_{MFHE} \mat{R}_{\mat{Z}_b}  + \sum_{i,j\in [r]: \mat{R}[i,j] = 1} \mat{P}_{(i,j)}) \cdot \mat{G}^{-1}(\sum_{i,j\in [r]: \mat{G}^{-1}(d \cdot \mat{G})[i,j] = 1} \mat{P}_{(i,j)})\\
%          &- \sum_{i,j\in [r]: \mat{R}[i,j] = 1} \mat{P}_{(i,j)} \cdot \mat{G}^{-1}(\mat{C}_{\mat{D}}) \\
%          &+ \sum_{i,j \in [r]: \mat{R}_b[i,j]= 1}\mat{P}_{(i,j)}\\
%        \end{align*}
%      \item Output $(\mat{C}_f, \mat{Z}_a, \mat{Z}_b)$.
%      \end{enumerate}
%      
%    \item[$\Verif(\PK, \SK_{V}, \SK_{MFHE}, \mat{C}, \mat{C}_a, \mat{Z}_a, \mat{C}_b, \mat{Z}_b, \mat{C}_f)$]:
%      \begin{enumerate}
%      \item Verify that $(\mat{C}_a, \mat{C}_b, \mat{C}_f, \mat{Z}_a, \mat{Z}_b) \in (\mathbb{Z}_q^{n \cdot N})^3 \times (\mathbb{Z}_q^{(n+N)\cdot N})^2$.
%      \item Decrypt
%        \begin{align*}
%          \mat{z}_a &\gets \Dec_{MFHE}(\SK_{MFHE}, \mat{Z}_a) & \mat{z}_b &\gets \Dec_{MFHE}(\SK_{MFHE}, \mat{Z}_b) & e &\gets Dec(\SK_{V}, C_e)
%        \end{align*}
%        
%      \item Verify that
%        \begin{align*}
%          {\mathsf{MultConst}}(C, e)+C_a &= FHE.\Enc(f; \mat{z}_a) \\
%          {\mathsf{MultConst}}(C, f) - {\mathsf{MultConst}}(C, e) + C_b &= FHE.\Enc(0; \mat{z}_b)
%        \end{align*}
%      \end{enumerate}
%    \end{description}
%\end{subsection}

Before giving the Ring-LWE version of the DV-NIZK construction.
We need to give a less efficient variant of Ring-GSW~\cite{DBLP:journals/tc/KhedrGV16}.

\begin{subsection}{Regev-like Ring-GSW}
  In this subsection, we note for the security parameter $n$,
  \begin{enumerate}
  \item prime number $q$,
  \item polynomial ring $\R$ of degree $n$ and its quotient ring $\R_q = \R/q\R$, 
  \item $\ell = \ceil{log~q}$,
  \item number $N = 2 \ell$
  \item matrix dimension $m$ \st $ m > (1 + \varepsilon) \cdot 2n \ell$ for some constant $\varepsilon$ and
  \item $\sigma_k$ standard deviation of Gaussian distribution in the key space. 
  \end{enumerate}
  \begin{paragraph}{Ring-GSW}
    \begin{description}
    \item[$\Setup(n)$]:
      \begin{enumerate}
      \item Uniformaly sample a vector of polynomials $\vec{a} = \begin{bmatrix} a_1 \\ \vdots \\ a_m \end{bmatrix} \sample R_q^m$.
      \item Sample a vector of polynomials $\vec{e} = \begin{bmatrix} e_1 \\ \vdots \\ e_m \end{bmatrix} \sample \chi_{R_q, \sigma_k}^m$ from a discret Gaussin distribution.
      \item Uniformaly sample a random polynomial $t \sample \R_q$.
      \item Compute $\vec{b} = t \cdot \vec{a} + \vec{e}$.
      \item Output:
        \begin{align*}
          \pk &= \mat{A} = \begin{bmatrix} \vec{b}^T \\ \vec{a}^T \end{bmatrix} & \sk &= \begin{bmatrix} 1 \\ -t\end{bmatrix}
        \end{align*}
      \end{enumerate}
    \item[$\Enc(\pk, \mu)$]: To encrypt the message $\mu$ using the public key $\pk = \mat{A}$
      \begin{enumerate}
      \item Uniformly sample a matrix $\mat{R} \sample \{0,1\}^{m \times N}$.
      \item Compute
        \begin{align*}
          \mat{C} &= \mat{A} \mat{R} + \mu \cdot \mat{G}_{2\times N}
        \end{align*}
      \end{enumerate}
    \item[$\Dec(\sk, \mat{C})$]:
      \begin{enumerate}
      \item Parse $\sk$ as $\sk = \vec{s}^T$. Remark that we have
        \begin{align*}
          \vec{s}^T \cdot \mat{A} &= \vec{e}^T
        \end{align*}
      \item Compute
        \begin{align*}
          \vec{s}^T \cdot \mat{C} &= \vec{s}^T \cdot (\mat{A} \mat{R} + \mu \cdot \mat{G}_{2\times m})\\
          &=\vec{e}^T \cdot \mat{R} + \mu \cdot \vec{s}^T \cdot \mat{G}_{2 \times m}\\
          &= \mu \cdot \vec{s}^T \cdot \mat{G}_{2 \times m} + error
        \end{align*}
      \end{enumerate}
    \end{description}

    \begin{lemma}{(\textbf{security})}
      For any $\varepsilon > 0$ and $m > (1+\epsilon)(n+1) log(q)$,
      if there exists a probabilistic polynomial time adversary $\Adv$ who can distinguish the ciphertext $\mat{C} \in \R_q^{2 \times m}$
      from a uniformly sampled matrix $\tilde{\mat{C}} \sample \R_q^{m \times N}$,
      then there exists a distinguisher $\Adv_{RLWE}$ against the $RLWE_{\chi, n}$.
    \end{lemma}
    \begin{proof}
      In order to prove the IND-CPA property of the schem, we give a hybrid argument,
      \begin{description}
      \item[$\textbf{Game} 0$]:
        The initial game is the classical IND-CPA security game\todo{more precisions}.
      \item[$\textbf{Game} 1$]:
        In this game, we sample the public key $\mat{A}$ from uniformly random distribution. This game is indistinguishable from $\textbf{Game} 0$ with $RLWE_{R_q, \sigma_k}$ assumption.
      \item[$\textbf{Game} 2$]:
        Instead of encrypt message by computing $\mat{C} = \mat{A} \mat{R} + m \cdot \mat{G}_{2\times m}$, we sampel $\mat{B} \sample \R_q^{2 \times m}$ from uniformly random distribution.
        The indistinguishability of $\textbf{Game} 2$ from $\textbf{Game} 1$ can be proved using a variant of leftover hash lemma given in lemma~\ref{LHL}.
      \end{description}
      \begin{lemma}\label{LHL}{(\textbf{Leftover Hash Lemma}~\cite{DBLP:conf/stoc/ImpagliazzoLL89}\cite{DBLP:journals/jacm/Regev09})}
      \end{lemma}
    \end{proof}
  \end{paragraph}
  
\end{subsection}

\begin{subsection}{Batched matrix Ring-GSW}
  Following the same idea from~\cite{DBLP:conf/pkc/HiromasaAO15}. We give a batched version of Ring-GSW.
  \begin{description}
  \item[$\KeyGen(n)$]:
    \begin{enumerate}
    \item Set a uniformly random matrix $\mat{B} \sample \mathbb{Z}_q^{n \times m}$ and $\mat{S'} \sample \chi^{r \times n}$.
    \item Choose a noise matrix $\mat{E} \sample \chi^{r \times m}$.
    \item Let $\mat{S}:= [\mat{I}_r || - \mat{S}'] \in \mathbb{Z}_q^{r \times (n+r)}$.
    \item Set the matrix:
      \begin{align*}
        \mat{A} &:= (\frac{\mat{S}'\mat{B} + \mat{E}}{\mat{B}}) \in \mathbb{Z}_q^{(n+r) \times m}
      \end{align*}
    \item Denote $\mat{E}_{i,j} \in \{0, 1\}^{r\times r}$ with $(i,j) \in \{1, \dots ,r\}^2$.
    \item Sample $\mat{R}_{i,j} \sample \{0,1\}^{m \times N}$ and 
      \begin{align*}
        \mat{P}_{i,j} &:= \mat{A} \mat{R}_{i,j} + (\frac{\mat{M}_{i,j}\mat{S}}{\mat{0}}) \mat{G} \in \mathbb{Z}_q^{(n+r) \times N}.
      \end{align*}
    \item Output $\PK := (\{\mat{P}_{i,j \in [r]}, \mat{A}\})$ and $\SK = \mat{S}$.
    \end{enumerate}
    
  \item[$\Enc(\PK, \mat{M} \in \{0,1\}^{r \times r})$] :
    \begin{enumerate}
    \item Sampl a random matrix $\mat{R} \sample \{0,1\}^{m \times N}$.
    \item Compute the ciphertext:
      \begin{align*}
        \mat{C} &:= \mat{A} \mat{R} + \sum_{i,j \in [r] : \mat{M}[i,j] = 1} \mat{P}_{(i,j)} \in \mathbb{Z}_q^{(n+r) \times N}
      \end{align*}
    \end{enumerate}
    
  \item[$\Dec(\mat{C}, \SK)$]:
    \begin{enumerate}
    \item Compute
      \begin{align*}
        \mat{M} &=(\round{\langle \vec{s}_i, \vec{c}_{j\ell-1} \rangle}_2)_{i,j \in [r]} \in \{0,1\}^{r \times r}
      \end{align*}
    \end{enumerate}
  \end{description}
\end{subsection}
