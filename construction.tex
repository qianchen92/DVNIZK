
Before giving the Ring-LWE version of the DV-NIZK construction.
We need to give a less efficient variant of Ring-GSW~\cite{DBLP:journals/tc/KhedrGV16}.

\begin{subsection}{Regev-like Ring-GSW}
  In this subsection, we note for the security parameter $\lambda$,
  \begin{enumerate}
  \item prime number $q$,
  \item polynomial ring $\R$ of degree $n$ and its quotient ring $\R_q = \R/q\R$, 
  \item $\ell = \ceil{log~q}$,
  \item number $N = 2 \ell$
  \item matrix dimension $m$ \st $ m > (1 + \varepsilon) \cdot 2n \ell$ for some constant $\varepsilon$ and~\todo{This is not correct for now but a similar proof with another parameter should be fine}
  \item $\sigma_k$ standard deviation of Gaussian distribution in the key space. 
  \end{enumerate}
  \begin{paragraph}{Ring-GSW}
    \begin{description}
    \item[$\Setup(n)$]:
      \begin{enumerate}
      \item Uniformaly sample a vector of polynomials $\vec{a} = \begin{bmatrix} a_1 \\ \vdots \\ a_m \end{bmatrix} \sample R_q^m$.
      \item Sample a vector of polynomials $\vec{e} = \begin{bmatrix} e_1 \\ \vdots \\ e_m \end{bmatrix} \sample \chi_{R_q, \sigma_k}^m$ from a discret Gaussin distribution.
      \item Uniformaly sample a random polynomial $t \sample \R_q$.
      \item Compute $\vec{b} = t \cdot \vec{a} + \vec{e}$.
      \item Output:
        \begin{align*}
          \pk &= \mat{A} = \begin{bmatrix} \vec{b}^T \\ \vec{a}^T \end{bmatrix} & \sk &= \begin{bmatrix} 1 \\ -t\end{bmatrix}
        \end{align*}
      \end{enumerate}
    \item[$\Enc(\pk, \mu)$]: To encrypt the message $\mu$ using the public key $\pk = \mat{A}$
      \begin{enumerate}
      \item Uniformly sample a matrix $\mat{R} \sample \{0,1\}^{m \times N}$.
      \item Compute
        \begin{align*}
          \mat{C} &= \mat{A} \mat{R} + \mu \cdot \mat{G}_{2\times N}
        \end{align*}
      \end{enumerate}
    \item[$\Dec(\sk, \mat{C})$]:
      \begin{enumerate}
      \item Parse $\sk$ as $\sk = \vec{s}^T$. Remark that we have
        \begin{align*}
          \vec{s}^T \cdot \mat{A} &= \vec{e}^T
        \end{align*}
      \item Compute
        \begin{align*}
          \vec{s}^T \cdot \mat{C} &= \vec{s}^T \cdot (\mat{A} \mat{R} + \mu \cdot \mat{G}_{2\times m})\\
          &=\vec{e}^T \cdot \mat{R} + \mu \cdot \vec{s}^T \cdot \mat{G}_{2 \times m}\\
          &= \mu \cdot \vec{s}^T \cdot \mat{G}_{2 \times m} + error
        \end{align*}
      \end{enumerate}
    \end{description}

    \begin{lemma}{(\textbf{security})}
      For any $\varepsilon > 0$ and $m > (1+\epsilon)(n+1) log(q)$,
      if there exists a probabilistic polynomial time adversary $\Adv$ who can distinguish the ciphertext $\mat{C} \in \R_q^{2 \times m}$
      from a uniformly sampled matrix $\tilde{\mat{C}} \sample \R_q^{m \times N}$,
      then there exists a distinguisher $\Adv_{RLWE}$ against the $RLWE_{\chi, n}$.
    \end{lemma}
    \begin{proof}
      In order to prove the IND-CPA property of the schem, we give a hybrid argument,
      \begin{description}
      \item[$\textbf{Game} 0$]:
        The initial game is the classical IND-CPA security game\todo{more precisions}.
      \item[$\textbf{Game} 1$]:
        In this game, we sample the public key $\mat{A}$ from uniformly random distribution. This game is indistinguishable from $\textbf{Game} 0$ with $RLWE_{R_q, \sigma_k}$ assumption.
      \item[$\textbf{Game} 2$]:
        Instead of encrypt message by computing $\mat{C} = \mat{A} \mat{R} + m \cdot \mat{G}_{2\times m}$, we sampel $\mat{B} \sample \R_q^{2 \times m}$ from uniformly random distribution.
        The indistinguishability of $\textbf{Game} 2$ from $\textbf{Game} 1$ can be proved using a variant of leftover hash lemma given in lemma~\ref{LHL}.
      \end{description}

      \begin{definition}{Generalized Knapsack Function Family}{\cite{DBLP:conf/focs/Micciancio02}}
        For any ring $\mathcal{R}$, its subset $\mathcal{S} \subset \mathcal{R}$ and integer $m \leq 1$, for all $\vec{a} \in \mathcal{R}^m$ and $\vec{x} \in \mathcal{S}^m$, we define the generalized knapsack function family $\mathcal{H}(\mathcal{R}, \mathcal{S}, m) = \{f_{\vec{a}} : \mathcal{S}^m \to \mathcal{R}\}$ is defined by
        \begin{align*}
          f_{\vec{a}}(\vec{x}) &= \sum_{i=1}^m x_i \cdot a_i
        \end{align*}
      \end{definition}
      
      \begin{lemma}\label{LHL}{(\textbf{Leftover Hash Lemma}~\cite{DBLP:conf/stoc/ImpagliazzoLL89}\cite{DBLP:conf/focs/Micciancio02})}
        For a finite field $\mathcal{F}$, its subset $\mathcal{S} \in \mathcal{F}$ and two integers $n, m$, the hash function family $\mathcal{H}(\mathcal{F}^n, \mathcal{S}^n, m) = \{f_{\vec{x}} : (\mathcal{S}^n)^m \to \}$ defined as
        
      \end{lemma}
    \end{proof}
    {\todo[inline]{We cannot directly use the lemma in the original paper.... so need a sketch proof for it.}}
  \end{paragraph}
  
\end{subsection}

\begin{subsection}{Batched matrix Ring-GSW}
  Following the same idea from~\cite{DBLP:conf/pkc/HiromasaAO15}. We give a batched version of Ring-GSW.
  We need to encryption the matrix of size $N \times N$.
  We show in the following construction how to encryption a matrix of size $r \times r$.
  \begin{description}
  \item[$\KeyGen(n)$]:
    \begin{enumerate}
    \item Set a uniformly random matrix $\mat{B} \sample \R_q^{2 \times m}$ and $\mat{S'} \sample \chi^{r \times 2}$.
    \item Choose a discrete Gaussian noise matrix $\mat{E} \sample \chi^{r \times m}$.
    \item Let $\mat{S}:= [\mat{I}_r || - \mat{S}'] \in \R_q^{r \times (2+r)}$.
    \item Set the matrix:
      \begin{align*}
        \mat{A} &:= (\frac{\mat{S}'\mat{B} + \mat{E}}{\mat{B}}) \in \R_q^{(2+r) \times m}
      \end{align*}
    \item Denote $\mat{E}_{i,j} \in \{0, 1\}^{r\times r}$ with $(i,j) \in \{1, \dots ,r\}^2$.
    \item Sample $\mat{R}_{i,j} \sample (\R_q \cap \{0,1\})^{m \times N}$ and compute 
      \begin{align*}
        \mat{P}_{i,j} &:= \mat{A} \mat{R}_{i,j} + (\frac{\mat{M}_{i,j}\mat{S}}{\mat{0}}) \mat{G} \in \R_q^{(2+r) \times N}.
      \end{align*}
    \item Output $\PK := (\{\mat{P}_{i,j \in [r]}, \mat{A}\})$ and $\SK = \mat{S}$.
    \end{enumerate}
    
  \item[$\Enc(\PK, \mat{M} \in \{0,1\}^{r \times r})$] :
    \begin{enumerate}
    \item Sampl a random matrix $\mat{R} \sample \{0,1\}^{m \times N}$.
    \item Compute the ciphertext:
      \begin{align*}
        \mat{C} &:= \mat{A} \mat{R} + \sum_{i,j \in [r] : \mat{M}[i,j] = 1} \mat{P}_{(i,j)} \in \mathbb{Z}_q^{(n+r) \times N}
      \end{align*}
    \end{enumerate}
    
  \item[$\Dec(\mat{C}, \SK)$]:
    \begin{enumerate}
    \item Compute
      \begin{align*}
        \mat{M} &=(\round{\langle \vec{s}_i, \vec{c}_{j\ell-1} \rangle}_2)_{i,j \in [r]} \in \{0,1\}^{r \times r}
      \end{align*}
    \end{enumerate}
  \end{description}
\end{subsection}
