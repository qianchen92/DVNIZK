In this section, we will present two constructions of encryption scheme based on Ring-LWE assumption.


\begin{subsection}{Ring-GSW encryption scheme~\cite{DBLP:journals/tc/KhedrGV16}}
  \begin{description}
  \item[$\Setup(1^\lambda)$]: where $\lambda$ is the security parameter of the encryption scheme
    \begin{enumerate}
    \item Sample a polynomial $t_{RGSW} \sample \mathcal{D}_{\R_q, \sigma_{(k,RGSW)}}$
      \todo{Precisely define somewhere the deviation of the Gaussian distribution used here and same for the other deviation}
    \item Uniformly sample $a_{RGSW} \sample \R_q$ and a noise polynomial $e_{RGSW} \sample \mathcal{D}_{\R_q, \sigma_{(k, RGSW)}}$.
    \item Compute $b_{RGSW} = s_{RGSW} \cdot a_{RGSW} + e_{RGSW} \in \R_q$
    \item Set secret key and public key of the encryption scheme:
      \begin{align*}
        \SK &= \vec{S}_{RGSW} =\begin{bmatrix} 1 \\ -s_{RGSW} \end{bmatrix}  & \PK &= \vec{a}_{RGSW} =  \begin{bmatrix} b_{RGSW} \\ a_{RGSW}\end{bmatrix}
      \end{align*}
      Note that we have the following relation:
      \begin{align*}
        \vec{s}_{RGSW}^T \cdot \vec{a}_{RGSW} &= \begin{bmatrix} 1 & -s_{RGSW} \end{bmatrix} \cdot \begin{bmatrix} b_{RGSW} \\ a_{RGSW}\end{bmatrix} = \begin{bmatrix} e_{RGSW} \end{bmatrix}
      \end{align*}
    \end{enumerate}
  \item[$\Enc(\PK, \mu)$]: The message $\mu$ is in the message space $\R_q$
    \begin{enumerate}
    \item Parse the public key $\PK = \vec{a}_{RGSW}$
    \item Sample a $(N \times 1)-$matrix $\mat{R}_{RGSW} \sample (\R_q \cap\{0,1\})^{N \times 1}$.
    \item Sample a $(N \times 2)-$noise matrix $\mat{E}_{RGSW} \sample (\mathcal{D}_{\R_q^{N \times 2}, \sigma_{c, RGSW}})$.
    \item Compute $C_{RGSW}$ the ciphertext of $\mu \in \R_q$:
      \begin{align*}
        \mat{C}_{RGSW} &= \mu \cdot \mat{G} + \vec{a}_{RGSW} \cdot \mat{R}_{RGSW} + \mat{E}_{RGSW}
      \end{align*}
    \end{enumerate}
  \item[$\Dec(\SK, \mat{C}_{RGSW})$]:
    \begin{enumerate}
    \item Parse the secret key $\SK = \mat{S}_{RGSW}$.
    \item Compute the plaintext as follows:
      \begin{align*}
        \vec{s}_{GSW}^T \cdot \mat{C}_{GSW} &= \mu \cdot \vec{s}_{RGSW}^T \cdot \mat{G} + error 
      \end{align*}
    
    \end{enumerate}
  \end{description}
  
\end{subsection}

\begin{subsection}{Ring-LWE encryption scheme~\cite{DBLP:conf/eurocrypt/LyubashevskyPR13}}

  Here we will present compact encryption scheme based on Ring-LWE assumption~\cite{DBLP:conf/eurocrypt/LyubashevskyPR13}.
  It is a linearly homomorphic encryption scheme.
  
  \begin{description}
  \item[$\Setup(1^\lambda)$]: where $\lambda$ is the security parameter.
    \begin{enumerate}
    \item Sample a polynomial $s \sample \mathcal{D}_{(\R_q, \sigma_{BV})}$, $a_0 \sample \R_q$ and $e_0 \sample \mathcal{D}_{(\R_q, \sigma_{BV})}$.
    \item Set $b_0 = a_0 s + 2 e_0$.
    \item Set secret key and public key
      \begin{align*}
        \SK &= \begin{bmatrix} 1 \\ -s\end{bmatrix} & \PK &= \begin{bmatrix} b_0 \\ a_0\end{bmatrix}
      \end{align*}
    \end{enumerate}
  \item[$\Enc(\PK, \mu)$]: where the message $\mu$ is a polynomial in $\R_2$.
    \begin{enumerate}
    \item Sample polynomials $v,e' \sample \mathcal{D}_{(R_q, \sigma_{BV})}$ and $e'' \sample  \mathcal{D}_{(R_q, \sigma_{BV}')}$.
    \item Compute two polynomials $(a, b) \in \R_q^2$
      \begin{align*}
        a &= a_0 v + 2 e' & b &= b_0 v +2 e'' 
      \end{align*}
    \item Compute the ciphertext $\vec{c} = \begin{bmatrix} c_0 \\ c_1\end{bmatrix}$
      \begin{align*}
        c_0 &= b + \mu & c_1 &= -a
      \end{align*}
    \end{enumerate}
  \item[$\Dec(\SK, \vec{c})$]:
    \begin{enumerate}
    \item Parse the secret key $\SK = \vec{s}$
    \item Compute the plaintext $\mu = \round{\langle \vec{s}, \vec{c}\rangle}_2$
    \end{enumerate}
  \end{description}
\end{subsection}



\begin{subsection}{Designed-Verifier Non-Interactive Zero-Knowledge proof (DV-NIZK)}

  \begin{definition}{(DV-NIZK)} A Designed-Verifier Non-Interactive Zero-Knowledge proof system is a triple of algorithm $(\Setup, \Prove, \Verif)$ which verify the following specifications:
    \begin{description}
    \item[$\Setup(1^\lambda) \to (\PPP, \SK)$]: The $\Setup$ algorithm takes the security parameter $\lambda$ and outputs the public parameter $\PPP$ and a secret verification key.
    \item[$\Prove(\PPP, (x, w)) \to \pi$]: The prover use the public parameter $\PPP$ to prove the fact that
      \begin{align*}
        x &\in_w \mathcal{L}
      \end{align*}
      which means $w$ is the witness of the fact that $x$ is in the language $\mathcal{L}$.
    \item[$\Verif(\PPP, \SK, x, \pi) \to \{\True, \False\}$]: The verifier use the secret verification key $\SK$ and the proof $\pi$ to verify that $x \in \mathcal{L}$. 
    \end{description}
  \end{definition}

  \begin{definition}{(Zero-Knowledge)}
    A DV-NIZK proof system $(\Setup, \Prove, \Verif)$ is zero-knowledge if there exists two probabilistic polynomial time (PPT) algorithms $\mathcal{S}_{\Setup}, \mathcal{S}_{\Prove}$ such that for any PPT adversary $\Adv$:
    \begin{align*}
      &\Pr[(x, w) \gets \Adv; (\PPP,\SK) \gets \Setup(1^\lambda); \pi \gets \Prove(\PPP, (x,w)) | \Adv(\pi, \SK) = 1]\\
      \approx &\Pr[(x, w) \gets \Adv; (\PPP,\SK, \SK_{sim}) \gets \mathcal{S}_{\Setup}(1^\lambda); \pi \gets \mathcal{S}_{\Prove}(\PPP, x, \SK_{sim}) | \Adv(\pi, \SK) = 1]
    \end{align*}
  \end{definition}



  \begin{definition}{(Adaptive culpable soundness)}
    A DV-NIZK proof system $(\Setup, \Prove, \Verif)$ is culpably sound with respect to the relation $\mathcal{R}_{guilt}$ if for all PPT adversary $\Adv$:
    \begin{align*}
      \Pr[(\PPP, \SK) \gets \Setup(1^{\lambda}); (x, \pi, w_{guilt}) \gets \Adv(\PPP) | (x, w_{guilt}) \in \mathcal{R}_{guilt} \wedge \Verif(\SK, x, \pi) = \True] \approx 0
    \end{align*}
  \end{definition}

  
\end{subsection}
