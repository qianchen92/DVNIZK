\documentclass[10pt]{llncs}

%\input{preamble.tex}

%%%%%%%%%%%%%%%%%% macros %%%%%%%%%%%%%%%%%%

\usepackage{cite}
\usepackage{amssymb}
%\bibliography{main}

%\usepackage{fullpage}


%\usepackage[top=1.5cm, bottom=1.5cm, left=1.5cm, right=1.5cm]{geometry}
\usepackage[english]{babel}
\usepackage[utf8]{inputenc}
  
%\usepackage[T1]{fontenc}
\usepackage{lmodern}
\usepackage[onelanguage,boxed]{algorithm2e}
\usepackage{graphicx}
\usepackage{float}
\usepackage{amsmath}
\usepackage{amsfonts}
%\usepackage{amsthm}
\usepackage{color}
\usepackage[usenames,dvipsnames]{xcolor}
\usepackage[toc,page]{appendix}
\usepackage{listings}
\usepackage{tikz}
\usepackage{subfigure}
\usepackage{wrapfig}
\usepackage{hyperref}
\hypersetup{colorlinks,linkcolor=black,urlcolor=blue}
\usepackage[font=small,labelfont=bf]{caption}
%\newtheorem {myDef} {Definition}
%\newtheorem {myTh}{Theorem}

\lstset
{
  emph={Stack},
  emphstyle={\color{DarkOrchid}}
}


\newcommand{\OTS}{\mathcal{OTS}}
\newcommand{\TC}{\mathcal{TC}}
\newcommand{\Adv}{\mathbf{Adv}}

\newcommand{\A}{\mathcal{A}}
\newcommand{\B}{\mathcal{B}}
\newcommand{\D}{\mathcal{D}}
\newcommand{\G}{\mathbb{G}}
\newcommand{\Z}{\mathbb{Z}}
\newcommand{\PR}{\operatorname{Pr}}
\newcommand{\PP}{\mathsf{P}}  
\newcommand{\VV}{\mathsf{V}}  
\newcommand{\K}{\mathsf{K}}  
\newcommand{\SIM}{\mathsf{S}}  
\newcommand{\lbl}{\mathsf{lbl}} 
\newcommand{\PPE}{\mathrm{PPE}} 
\newcommand{\SK}{\mathsf{SK}}
\newcommand{\PK}{\mathsf{PK}}
\newcommand{\VK}{\mathsf{VK}}
\newcommand{\SSK}{\mathsf{SSK}}
\newcommand{\SVK}{\mathsf{SVK}}
\newcommand{\sk}{\mathsf{sk}}
\newcommand{\ck}{\mathsf{ck}}
\newcommand{\tk}{\mathsf{tk}}
\newcommand{\msk}{\mathsf{msk}}
\newcommand{\vk}{\mathsf{vk}}
\newcommand{\ovk}{\mathsf{ovk}}
\newcommand{\pk}{\mathsf{pk}}
\newcommand{\opk}{\mathsf{opk}}
\newcommand{\osk}{\mathsf{osk}}
\newcommand{\com}{\hat{\mathsf{com}}}
\newcommand{\open}{\mathsf{open}}
\newcommand{\True}{\mathsf{True}}
\newcommand{\False}{\mathsf{False}}
\newcommand{\BF}{\mathbf}
\newcommand{\sample}{\stackrel{{\scriptscriptstyle \mkern4mu R}}{\gets}}
\newcommand{\etal}{\textit{el. al.}}
\newcommand{\eg}{\textrm{e.g.} }
\newcommand{\ie}{\textrm{i.e.} }
\newcommand{\wrt}{\textrm{w.r.t.} }
\newcommand{\st}{\textrm{s.t.} }
\newcommand{\resp}{\textrm{resp.} }
\providecommand{\tprod}{{\textstyle\prod}}
\newcommand{\Setup}{{\mathsf{Setup}}}
\newcommand{\KeyGen}{{\mathsf{KeyGen}}}
\newcommand{\OKeyGen}{{\mathsf{OKeyGen}}}
\newcommand{\Enc}{{\mathsf{Encrypt}}}
\newcommand{\Dec}{{\mathsf{Decrypt}}}
\newcommand{\Rerand}{{\mathsf{ReRand}}}
\newcommand{\Sig}{{\mathsf{Sign}}}
\newcommand{\Verif}{{\mathsf{Verify}}}
\newcommand{\Prove}{{\mathsf{Prove}}}
\newcommand{\Com}{{\mathsf{Commit}}}
\newcommand{\PPP}{\mathsf{PP}}
\newcommand{\Forge}{\mathsf{Forge}}
%\newcommand{\Adv}{\mathcal{A}}


%%--- Page settings
\makeatletter
\renewcommand{\paragraph}{%
  \@startsection {paragraph}{4}{0pt}{-12pt \@plus -4pt \@minus -4pt}%
  {-0.5em \@plus -0.22em \@minus -0.1em}{\normalfont\normalsize\scshape }}
\makeatother
% \usepackage{enumitem} 
% \setlist[description]{itemsep=2pt}
 




%%%%%%%%%%%%%%%%%%%%%%%%%%%%%%%%%%%%%%%%%%%%%



\usepackage{comment}




\begin{document}

\pagestyle{plain}

%\title{Shorter Publicly Verifiable Ciphertexts in Structure-Preserving Chosen-Ciphertext-Secure Public-Key Encryption}


\title{Structure-Preserving Chosen-Ciphertext Security With   Shorter Verifiable Ciphertexts}

\author{}
 
\institute{}


\maketitle

\begin{abstract} 
 
 
Structure-preserving cryptography is a world where messages, signatures, ciphertexts and public keys are entirely made of 
elements of a group over which a bilinear map is efficiently computable. While structure-preserving signatures have received 
much attention   the last $6$ years, structure-preserving encryption schemes have undergone slower development. 
In particular, the best known  structure-preserving cryptosystems with chosen-ciphertext (IND-CCA2) security either rely on 
symmetric pairings or require 
long ciphertexts comprised of hundred of group elements or do not provide publicly verifiable ciphertexts. We provide a publicly verifiable construction based on the SXDH assumption in asymmetric bilinear groups
 $e : \G \times \hat{\G} \rightarrow \G_T$, 
which features relatively short ciphertexts.   For typical parameters, our ciphertext size amounts to less than $40$ elements of $\G$. 
As a second contribution, we provide a structure-preserving encryption scheme with perfectly randomizable ciphertexts and replayable 
chosen-ciphertext security. Our new RCCA-secure system significantly improves upon the best known system featuring similar properties in terms 
of ciphertext size. \smallskip \smallskip 

\noindent \textbf{Keywords.} Structure-preserving encryption,   chosen-ciphertext security, RCCA security, public ciphertext verifiability.    
 

\end{abstract}



%%%%%%%%%%%%%%%%%%%%%%%%%%%%%%%%%%%%%%%%%%%%%%%%%%%%%%%%%%%%%%%%%%%%%%%%%%%%%%%
\section{Introduction}
%%%%%%%%%%%%%%%%%%%%%%%%%%%%%%%%%%%%%%%%%%%%%%%%%%%%%%%%%%%%%%%%%%%%%%%%%%%%%%%

 \begin{subsection}{Notations}
  For a distribution $\mathcal{D}$ over a finite set $\mathcal{S}$,
  we denote $d \sample \mathcal{D}$ for the fact that $d$ is chosen from the distribution $\mathcal{D}$ and
  $d \sample \mathcal{S}$ for the fact that $d$ is chosen from the uniform distribution over the finite set $\mathcal{S}$.\

  We denote the ring of polynomials over the integers by $\mathbb{Z}[x]$.
  For a degree $n$ polynomial $f(x)$, we note the quotient ring $\R = \mathbb{Z}[x]/\langle f(x) \rangle$.
  Especially for a prime number $q$, we denote the ring of polynomila over the finite field $\mathbb{Z}_q$ by $\mathbb{Z}_q[x]$
  and the quotient ring by a degree $n$ polynomial $f(x)$ by $\R_q = \mathbb{Z}_q[x]/\langle f(x) \rangle$.

  \todo{Need several more definition, discret Gaussian distribution, vector, matrix \etc}
\end{subsection}
 

%%%%%%%%%%%%%%%%%%%%%%%%%%%%%%%%%%%%%%%%%%%%%%%%%%%%%%%%%%%%%%%%%%%%%%%%%%%%%%%
\section{Background and Definitions} \label{background}
%%%%%%%%%%%%%%%%%%%%%%%%%%%%%%%%%%%%%%%%%%%%%%%%%%%%%%%%%%%%%%%%%%%%%%%%%%%%%%%

%In this section, we will present two constructions of homomorphic encryption based on Ring-LWE assumption.


\begin{subsection}{Ring-GSW encryption scheme~\cite{DBLP:journals/tc/KhedrGV16}}
  \begin{description}
  \item[$\Setup(1^\lambda)$]: where $\lambda$ is the security parameter of the encryption scheme
    \begin{enumerate}
    \item Sample a polynomial $t_{RGSW} \sample \mathcal{D}_{\R_q, \sigma_{(k,RGSW)}}$
      \todo{Precisely define somewhere the deviation of the Gaussian distribution used here and same for the other deviation}
    \item Uniformly sample $a_{RGSW} \sample \R_q$ and a noise polynomial $e_{RGSW} \sample \mathcal{D}_{\R_q, \sigma_{(k, RGSW)}}$.
    \item Compute $b_{RGSW} = t_{RGSW} \cdot a_{RGSW} + e_{RGSW} \in \R_q$
    \item Set secret key and public key of the encryption scheme:
      \begin{align*}
        \SK &= \mat{S}_{RGSW} =\begin{bmatrix} 1 & -t \end{bmatrix}  & \PK &= \mat{A}_{RGSW} =  \begin{bmatrix} b_{RGSW} \\ a_{RGSW}\end{bmatrix}
      \end{align*}
      Note that we have the following relation:
      \begin{align*}
        \mat{S}_{RGSW} \cdot \mat{A}_{RGSW} &= \begin{bmatrix} 1 & -t_{RGSW} \end{bmatrix} \cdot \begin{bmatrix} b_{RGSW} \\ a_{RGSW}\end{bmatrix} = \begin{bmatrix} e_{RGSW} \end{bmatrix}
      \end{align*}
    \end{enumerate}
  \item[$\Enc(\PK, \mu)$]: The message $\mu$ is in the message space $\R_q$
    \begin{enumerate}
    \item Parse the public key $\PK = \mat{A}_{RGSW}$
    \item Sample a $(N \times 1)-$matrix $\mat{R}_{RGSW} \sample (\R_q \cap\{0,1\})^{N \times 1}$.
    \item Sample a $(N \times 2)-$noise matrix $\mat{E}_{RGSW} \sample (\mathcal{D}_{\R_q^{N \times 2}, \sigma_{c, RGSW}})$.
    \item Compute $C_{RGSW}$ the ciphertext of $\mu \in \R_q$:
      \begin{align*}
        \mat{C}_{RGSW} &= \mu \cdot \mat{G} + \mat{A}_{RGSW} \cdot \mat{R}_{RGSW} + \mat{E}_{RGSW}
      \end{align*}
    \end{enumerate}
  \item[$\Dec(\SK, \mat{C}_{RGSW})$]:
    \begin{enumerate}
    \item Parse the secret key $\SK = \mat{S}_{RGSW}$.
    \item Compute the plain text as follows:
      \begin{align*}
        \mat{S}_{GSW} \cdot \mat{C}_{GSW} &= \mu \cdot \mat{S}_{RGSW} \cdot \mat{G} + error 
      \end{align*}
    
    \end{enumerate}
  \end{description}
  
\end{subsection}

\begin{subsection}{BV-SHE based on Ring-LWE~\cite{DBLP:conf/crypto/BrakerskiV11}}

  Here we will present a somewhat homomorphic encryption BV-SHE proposed by~\cite{DBLP:conf/crypto/BrakerskiV11}.
  In this BV-SHE, we need to give the multiplication level in the setup procedure.
  It is sufficiant in our construction.
  In fact we only need its linearly homomorphism.
  
  \begin{description}
  \item[$\Setup(1^\lambda, D)$]: where $\lambda$ is the security parameter and $D$ is the multiplication level predefined for the somewhat encryption scheme.
    \begin{enumerate}
    \item Sample a polynomail $s \sample \mathcal{D}_{(\R_q, \sigma_{BV})}$, $a_0 \sample \R_q$ and $e_0 \sample \mathcal{D}_{(\R_q, \sigma_{BV})}$.
    \item Set $b_0 = a_0 s + 2 e_0$.
    \item Set secret key and public key
      \begin{align*}
        \SK &= \begin{bmatrix} 1 & s & s^2 & \dots & s^D\end{bmatrix}^T & \PK &= \begin{bmatrix} a_0 & b_0\end{bmatrix}^T
      \end{align*}
    \end{enumerate}
  \item[$\Enc(\PK, \mu)$]: where the message $\mu$ is a polynomial in $\R_2$.
    \begin{enumerate}
    \item Sample polynomails $v,e' \sample \mathcal{D}_{(R_q, \sigma_{BV})}$ and $e'' \sample  \mathcal{D}_{(R_q, \sigma_{BV}')}$.
    \item Compute two polynomials $(a, b) \in \R_q^2$
      \begin{align*}
        a &= a_0 v + 2 e' & b &= b_0 v +2 e'' 
      \end{align*}
    \item Compute the ciphertext $\vec{c} = \begin{bmatrix} c_0 \\ c_1\end{bmatrix}$
      \begin{align*}
        c_0 &= b + \mu & c_1 &= -a
      \end{align*}
    \end{enumerate}
  \item[$\Dec(\SK, \vec{c})$]:
    \begin{enumerate}
    \item Parse the secret key $\SK = \vec{s}$
    \item Compute the plaintext $\mu = \round{\langle \vec{s}, \vec{c}\rangle}_2$
    \end{enumerate}
  \end{description}
\end{subsection}



\begin{subsection}{Designed-Verifier Non-Interactive Zero-Knowledge proof (DV-NIZK)}

  \begin{definition}{(DV-NIZK)} A Designed-Verifier Non-Interactive Zero-Knowledge proof system is a triple of algorithm $(\Setup, \Prove, \Verif)$ which verify the following specifications:
    \begin{description}
    \item[$\Setup(1^\lambda) \to (\PPP, \SK)$]: The $\Setup$ algorithm takes the security parameter $\lambda$ and outputs the public parameter $\PPP$ and a secret verification key.
    \item[$\Prove(\PPP, (x, w)) \to \pi$]: The prover use the public parameter $\PPP$ to prove the fact that
      \begin{align*}
        x &\in_w \mathcal{L}
      \end{align*}
      which means $w$ is the witness of the fact that $x$ is in the language $\mathcal{L}$.
    \item[$\Verif(\PPP, \SK, x, \pi) \to \{\True, \False\}$]: The verifier use the secret verification key $\SK$ to verify that $x \in \mathcal{L}$. 
    \end{description}
  \end{definition}

  \begin{definition}{(Zero-Knowledge)}
    A DV-NIZK proof system $(\Setup, \Prove, \Verif)$ is zero-knowledge if there exists two probabilistic polynomial time (PPT) algorithms $\mathcal{S}_{\Setup}, \mathcal{S}_{\Prove}$ such that for any PPT adversary $\Adv$:
    \begin{align*}
      &\Pr[(x, w) \gets \Adv; (\PPP,\SK) \gets \Setup(1^\lambda); \pi \gets \Prove(\PPP, (x,w)) | \Adv(\pi, \SK) = 1]\\
      \approx &\Pr[(x, w) \gets \Adv; (\PPP,\SK, \SK_{sim}) \gets \mathcal{S}_{\Setup}(1^\lambda); \pi \gets \mathcal{S}_{\Prove}(\PPP, x, \SK_{sim}) | \Adv(\pi, \SK) = 1]
    \end{align*}
  \end{definition}



  \begin{definition}{(Adaptive culpable soundness)}
    A DV-NIZK proof system $(\Setup, \Prove, \Verif)$ is culpably sound with respect to the relation $\mathcal{R}_{guilt}$ if for all PPT adversary $\Adv$:
    \begin{align*}
      \Pr[(\PPP, \SK) \gets \Setup(1^{\lambda}), (x, \pi, w_{guilt}) \gets \Adv(\PPP) | (x, w_{guilt}) \in \mathcal{R}_{guilt} \wedge \Verif(\SK, x, \pi) = \True] \approx 0
    \end{align*}
  \end{definition}

  
\end{subsection}


\subsection{Hardness Assumptions}
%~~~~~~~~~~~~~~~~~~~~~~~~~~~~~~~~~~ 
We consider groups $(\G,\hat{\G},\G_T)$ of prime-order $p$ endowed
with a bilinear map $e:\G \times \hat{\G} \to \G_T$. 

\begin{comment}
In this setting,
we rely on the standard Decision Linear assumption, which is a special case of the $K$-linear assumption (see Definition \ref{k-lin-def} in Appendix \ref{k-lin-ext}))
 for $K=2$.
  	
\begin{definition}[\cite{DBLP:conf/crypto/BonehBS04}] \label{DLIN-def} The \emph{Decision
    Linear Problem} (DLIN) in $\G$, is to distinguish the
  distributions $(g,g^{a},g^b,g^{ac},g^{bd},g^{c+d})$ and
  $(g,g^{a},g^b,g^{ac},g^{bd},g^{z})$, with $a,b,c,d \sample \Z_p$,
  $z\sample \Z_p $.  The Decision Linear assumption asserts the
  intractability of DLIN for any PPT distinguisher.
\end{definition}  
 
It will sometimes be convenient to use the following assumption, which
is weaker than DLIN. As noted in \cite{DBLP:conf/asiacrypt/CathaloLY09}, any algorithm solving
SDP immediately yields a DLIN distinguisher.
      
\begin{definition} \label{SDP-def} The \emph{Simultaneous Double
    Pairing problem} (SDP) in $(\G,\hat{\G}, \G_T)$ is, given a tuple of group
  elements $( \hat{g}_z, \hat{g}_r, \hat{h}_{z}, \hat{h}_{u}) \in
  \hat{\G}^4 $, to find a non-trivial triple $(z,r,u) \in \G^3
  \backslash \{(1_{\G},1_{\G},1_{\G})\}$ such that $e( z, \hat{g_z})
  \cdot e(r,\hat{g_{r}})=1_{\G_T}$ and $e(z,\hat{h}_z ) \cdot e(u,
  \hat{h}_{u})=1_{\G_T}$.
\end{definition}

\end{comment}

\begin{definition} \label{DDH-def} The {\bf Diffie-Hellman problem} (DDH) in $\G$,
 is to distinguish the distributions $(g, g^a, g^b, g^{ab})$ and $(g, g^a, g^b, g^c)$ with $a, b, c \sample \Z_p$.
 The Diffie-Hellman assumption asserts the intractability of DDH for any PPT distinguisher.

 In the asymmetric setting $(\G,\hat{\G}, \G_T)$, we consider the SXDH  assumption,
 which posits that the DDH assumption holds in both $\G$ and $\hat{\G}$.
\end{definition}

\begin{definition} \label{DP-def} The {\bf Double Pairing problem} (DP) in $(\G, \hat{\G}, \G_T)$ is,
  given a pair of group elements $(\hat{g}_z, \hat{g_r}) \in \hat{\G}^2$,
  to find a non-trivial triple $(z, r) \in \G^2\backslash \{(1_{\G}, 1_{\G})\}$ such that $e(z, \hat{g}_z) \cdot e(r, \hat{g}_r) = 1_{\G_T}$.
\end{definition}


It is known \cite{DBLP:journals/iacr/AbeHO10} that the DP assumption is   implied by the   DDH assumption in $\G$. By exchanging the roles of $\G$ and 
$\hat{\G}$ 
in the definition of DP, we obtain a variant of the assumption which implies the hardness of DDH in $\hat{\G}$. 


\subsection{One-Time Structure-Preserving Signatures}\label{ot-sig}    
%%%%%%%%%%%%%%%%%%%%%%%%%%%%%%%%%%%%%%%%%%%%%%%%%%%%%%%%%%%%%%%%%%%%%%%%%%%%%%% 

Structure-preserving signatures (SPS) \cite{DBLP:conf/crypto/AbeFGHO10,DBLP:journals/iacr/AbeHO10} are signature
schemes where messages and public keys all consist of elements of a group
over which a bilinear map $e: \G \times \hat{\G} \to \G_T$ is
efficiently computable. Constructions based on simple assumptions were
put forth in \cite{DBLP:conf/asiacrypt/AbeCDKNO12,DBLP:conf/pkc/AbeDKNO13}. 

In the forthcoming sections, we will rely on  one-time SPS schemes.
\begin{definition}
  A one-time signature scheme is a tuple of efficient algorithms $\OTS = (\Setup, \KeyGen, \Sig, \Verif)$ where:
  \begin{description}
  \item[\textsf{Setup}$({\lambda}) :$] This algorithm takes as input a security parameter $\lambda$ and generates the public parameters $\PPP$ for the scheme.
 \item[\textsf{KeyGen} $(\PPP):$] This algorithm takes as input $\PPP$ and generates a one-time secret key $\osk$ and a one-time verification key $\ovk$.
  \item[\textsf{Sign}$(\PPP, \osk, \vec{M})  :$] Given  as input $(\PPP, \osk)$ and a message $\vec{M}$, this algorithm produces a signature $\sigma$ for  $\vec{M}$.
  \item[\textsf{Verify}$(\PPP, \ovk, \vec{M}, \sigma) :$] The verification algorithm takes $(\PPP, \ovk, \vec{M}, \sigma)$ and returns $1$ or $0$.
  \end{description}
\end{definition}

Correctness  mandates that, for any $\lambda \in \mathbb{N}$,  any $\PPP \leftarrow \mathsf{Setup}(\lambda)$,  any pair $(\osk,\ovk) \leftarrow \mathsf{KeyGen}(\PPP)$, we have $\Verif(\PPP, \ovk, \vec{M}, \Sig(\PPP, \osk, \vec{M})) =1$ for any message $\vec{M}$.\\
\indent In addition, a one-time signature is said {\it structure-preserving} if the components of $\ovk$, $\vec{M}$ and $\sigma$ all live in 
the source groups $(\G,\hat{\G})$ of a configuration $(\G,\hat{\G},\G_T)$ of bilinear groups. 

 
\begin{definition}
  A one-time signature scheme  $\OTS = (\Setup, \KeyGen, \Sig, \Verif)$ is strongly unforgeable against  chosen
	message attack $(SUF\textrm{-}CMA)$ if the advantage
  \begin{align*}
    \Adv_{OTS,\A}^{SUF\textrm{-}CMA} &= \Pr \left[
			\begin{array}{l}
      (m^\star, \sigma^\star) \not \in Q_{\Sig^{OT}} \wedge\\
      \Verif(\ovk, m^\star, \sigma^\star) =1
    \end{array}
				    \middle| 
	                              \begin{array}{l}
                                \PPP \gets \Setup(1^{\lambda})\\
                                (\ovk,\osk) \gets \KeyGen(\PPP)\\
                                (m^\star,\sigma^\star) \gets \A^{\Sig^{OT}_{\osk}(\cdot)}(\ovk)
                              \end{array}  
    \right]
  \end{align*}
  is negligible against any PPT adversary $\mathcal{A}$. Here, $\Sig^{OT}_{\osk}(\cdot)$ is a signing oracle which  allows the adversary to obtain a  signature $\sigma_m$ of only one message $m$ for which $(m, \sigma_m)$ is stored in $Q_{\Sig^{OT}}$. 
\end{definition}



We recall a construction of the one-time Structure-Preserving Signature scheme which was proposed in~\cite{DBLP:conf/pkc/AbeDKNO13}.


\begin{description}
\item[\textsf{Setup}$(\lambda):$]  Choose asymmetric bilinear groups $(\G, \hat{\G}, \G_T)$ of prime order $p>2^\lambda $ and output 
$\PPP =(\G, \hat{\G}, \G_T)$.
\item[\textsf{KeyGen}$(\PPP):$]   Generates the signing key $\osk$ and the verification key $\ovk$ using the security parameter $\lambda$ and  the number $n$ of messages to be signed.
  \begin{enumerate}
  \item Choose $\hat{g}_z, \hat{g}_r, g \sample \hat{\G}$.
  \item For $i = 1$ to $n$, pick $(\chi_i, \gamma_i) \sample \mathbb{Z}_p^2$
    and compute $\hat g_i=\hat{g}_z^{\chi_i}\hat{g}_r^{\gamma_i}$.
  \item Pick $(\zeta, \rho) \sample \mathbb{Z}_p^2$ and compute $\hat{A} = g_z^{\zeta} \cdot g_r^{\rho}$.
  \item Set $\osk = (\{(\chi_i, \gamma_i)\}_{i = 1}^n, \zeta, \rho) \in \G^{2n+2}$ and 
	$$\ovk = (\hat{g}_z, \hat{g}_r, \{\hat{g}_i\}_{i=1}^n, \hat{A}) \in \hat{\G}^{n+3}.$$
  \end{enumerate}
\item[\textsf{Sign}$(\osk, \vec{M} = (M_1, \dots, M_n)):$] In order to sign   $\vec{M} = (M_1, \dots, M_n) \in \G^n$, 
    compute $z = g^{\zeta}\prod_{i = 1}^n M_i^{\chi_i}$ and $r = g^{\rho}\prod_{i=1}^n M_i^{\gamma_i}$.
Output the signature $\sigma = (z,r)$.
\item[\textsf{Verify}$(\ovk, \vec{M} = (M_1, \dots, M_n), \sigma = (z, r)):$] Return $1$ if and only if the following equations are satisfied:
  \begin{align}\label{OT-validity}
    e(z, \hat{g}_z) \cdot e(r, \hat{g}_r) &= e(g, \hat{A}) \cdot \prod_{i=1}^ne(M_i, \hat{g}_i)
  \end{align}
  %This algorithm returns $\True$ if and only if $\vec{M} \neq (1,\dots, 1)$ and the previous equation is verified.
\end{description}

	
	%%%%%%%%%%%%%%%%%%%%%%%%%%%%%%%%%%%%%%%%%%%%%%%%%%%%%%%%%%%%%%%%%%%%%%%%%%%%%%%
\subsection{Partial  One-time Signature}\label{pots}    
%%%%%%%%%%%%%%%%%%%%%%%%%%%%%%%%%%%%%%%%%%%%%%%%%%%%%%%%%%%%%%%%%%%%%%%%%%%%%%%
A special case of the one-time signature presented in   Section~\ref{ot-sig} 
is called Partial One-Time Signature (POTS)~\cite{DBLP:conf/pkc/BellareS07}.
In a such scheme,  part of the verification key can be re-used in multiple signatures  and 
 the remaining part  must be refreshed at  every  signature generation.

\begin{definition}
  A partial one-time signature (POTS) scheme is a tuple of algorithms $POTS = (\mathsf{Setup}, \mathsf{KeyGen}, \mathsf{OKeyGen}, \mathsf{Sign}, \mathsf{Verify})$.
  \begin{description}
  \item[\boldmath$\mathsf{Setup}(\lambda):$] 
    The setup algorithm takes as input a security parameter $\lambda$ and generates the public parameters $\PPP$ for the scheme.
  \item[\boldmath$\mathsf{KeyGen}(\PPP) :$]
    The key generation algorithm takes the public parameter $\PPP$ and generates the long-term signing key $\sk$ and long-term verification key $\vk$.
  \item[\boldmath$\mathsf{OKeyGen}(\PPP) :$] 
    The key generation algorithm takes $\PPP$ and generates the one-time signing key $\osk$ and the one-time verification key $\ovk$.
  \item[\boldmath$\mathsf{Sign}(\PPP, \sk, \osk, \vec{M}) :$] 
    The signature algorithm uses the $(\PPP, \osk)$ to produce a valid signature $\sigma$ for the message vector $\vec{M}$.
  \item[\boldmath$\mathsf{Verify}(\PPP, \vk, \ovk, \vec{M}, \sigma) :$] 
    The verification algorithm takes $(\PPP, \vk, \ovk, \vec{M}, \sigma)$ and returns $1$ or $0$.
  \end{description}
\end{definition}

Correctness requires that, for any $\PPP \gets \mathsf{Setup}(\lambda)$, $(\sk, \vk) \gets \mathsf{KeyGen}(\PPP)$ and $(\osk, \ovk) \gets \mathsf{OKeyGen}(\PPP)$,
the partial one-time signature scheme is correct if and only if  $\Verif(\PPP, \vk, \ovk, \vec{M}, \mathsf{Sign}(\PPP, \sk, \osk, \vec{M})) = 1$.
%\begin{align*}
   %\Verif(\PPP, \vk, \ovk, \vec{M}, \mathsf{Sign}(\PPP, \sk, \osk, \vec{M})) &= 1
%\end{align*}

We focus on the strong unforgeability against one-time chosen-message attack of our POTS.
\begin{definition}
  A POTS scheme $POTS = (\mathsf{Setup}, \mathsf{KeyGen}, \mathsf{OKeyGen}, \mathsf{Sign}, \mathsf{Verify})$ is strongly unforgeable against   one-time chosen-message attack (or OT-CMA secure) if:
  \begin{multline*}
    \Adv_{POTS, \A}^{OT\textrm{-}SU\textrm{-}CMA}(\lambda) \\ = \Pr \left[
                                            \begin{array}{l}
                                              \exists ~(m',\sigma') ~\mathrm{s.t.}~ (\ovk^\star, \sigma', m')  \in Q \\ \wedge ~ (\ovk^\star, \sigma^\star, m^\star) \not \in Q \\
                                           \wedge  ~    \Verif(\vk, \ovk^\star, m^\star, \sigma^\star) = 1
                                            \end{array}
    \middle|
    \begin{array}{l}
      \PPP \gets \Setup(1^\lambda)\\
      (\vk, \sk) \gets \KeyGen(\PPP)\\
      (\ovk^\star, \sigma^\star, m^\star) \gets \A^{\mathcal{O}_{\sk}} (\PPP,\vk)
    \end{array}
    \right]
  \end{multline*}
  is negligible for any PPT adversary $\A$. Here, the signing oracle inputs a message $m$, 
  generates $(\ovk, \osk) \gets \mathsf{OKeyGen}(\PPP)$, $\sigma \gets \mathsf{Sign}(\sk, \osk, m)$.
  Then, it records $(\ovk, m)$ to $Q$ and returns $(\sigma, \ovk)$.
  
\end{definition}
	
	
Here, we recall an instantiation of the POTS scheme~\cite{DBLP:conf/asiacrypt/AbeCDKNO12},
which is strongly unforgeable against the one-time chosen-message attack (SU-OTCMA) under the DP assumption.  % (Definition \ref{DP-def}) 
%in $\G$.
%
\begin{description}
\item[\boldmath$\Setup(\lambda, \ell):$] On input of a security parameter $\lambda$ and an integer $\ell \in poly(\lambda)$, 
  the setup algorithm chooses a large prime $p > 2^\lambda$, 
  asymmetric groups $(\G, \hat{\G}, \G_T)$ of prime order $p$, with a bilinear map $e : \G \times \hat{\G} \to \G_T$
  and the corresponding  generators $(g, \hat{g} ) \in \G \times \hat{\G}$.
  The algorithm outputs
  \begin{align*}
  \PPP & = (p, \G, \hat{\G}, \G_T, e, g, \hat{g}, \ell).
  \end{align*}

\item[\boldmath$\KeyGen(\PPP):$] Parse $\PPP$ as 
  $(p, \G, \hat{\G}, \G_T, e, g, \hat{g}, \ell)$. 
  Choose $w_z \sample \mathbb{Z}_p^*$ and compute $g_z \gets g^{w_z}$.
  For $i \in \{1, \dots, \ell\}$, choose $\chi_i \sample \mathbb{Z}_p$ 
  and compute $g_i \gets g^{\chi_i}$.
 Return 
  \begin{align*}
    \vk &= (g_z, g_1, \dots, g_\ell) \in \G^{\ell+1} &
    \sk &= (w_z, \chi_1, \dots, \chi_\ell) \in\mathbb{Z}_p^{\ell+1}
  \end{align*}
\item[\boldmath$\OKeyGen(\PPP):$] Parse $\PPP$, choose $a \gets \mathbb{Z}_p$, 
  compute $A \gets g^a$ and output
  \begin{align*}
    \ovk &= A & \osk &= a
  \end{align*}
\item[\boldmath$\mathsf{Sign}(\sk, \osk, \vec{\hat{M}}):$] Parse $\vec{\hat{M}}$ as  
  $(\hat{M}_1, \dots, \hat{M}_{\ell}) \in \hat{\G}^{\ell}$.
  Parse $\sk$ and $\osk$, choose $\zeta \sample \mathbb{Z}_p^*$,then 
  compute and output
  \begin{align*}
    \hat{Z} &= \hat{g}^\zeta & 
    \hat{R} &= \hat{g}^{a - \zeta w_z} 
               \mbox{$\prod_{i=1}^\ell$} \hat{M_i}^{-\chi_i}.
  \end{align*}
\item[\boldmath$\mathsf{Verify}(\vk, \ovk, \vec{\hat{M}}, \sigma):$] 
  Parse $\sigma$ as $(\hat{Z}, \hat{R}) \in \hat{\G}^2$, 
  $\vec{\hat{M}}$ as $(\hat{M}_1, \dots, \hat{M}_\ell) \in \hat{\G}^\ell$ 
  and $\ovk$ as $A \in \G$.
  The algorithm returns $1$ if the following equation holds:
  \begin{align*}
    e(A, \hat{g}) &= e(g_z, \hat{Z}) \cdot e(g, \hat{R}) 
                     \cdot \prod_{i = 1}^{\ell} e(g_i, \hat{M}_i)
  \end{align*}
  otherwise the algorithm returns $0$.

\end{description}
	
	
\subsection{One-Time Linearly Homomorphic Structure-Preserving Signatures}
\label{H-SPS}  
%~~~~~~~~~~~~~~~~~~~~~~~~~~~~~~~~~~~~~~~~~~~~~~~~~~~~~~~~~~~~~~~~~~~~~~~~~~
 
Libert \etal\  \cite{DBLP:conf/crypto/LibertPJY13} considered structure-preserving with
linear homomorphic properties (see Appendix \ref{hom-sig-sec-defs} for formal definitions). This section recalls the one-time
 linearly  homomorphic structure-preserving signature (LHSPS) of \cite{DBLP:conf/crypto/LibertPJY13}.
  
	\begin{comment}
	
\begin{description}
\item[\boldmath$\mathsf{Keygen}(\lambda,n)$:] Given a security
  parameter $\lambda$ and the dimension $n \in \mathbb{N}$ of the
  subspace to be signed, choose bilinear group $(\G,\hat{\G},\G_T)$ of
  prime order $p$. % >2^{\lambda}$.
  Then, choose $\hat{g_z},\hat{g_r},\hat{h_z},\hat{h_u} \sample
  \hat{\G}$.  For $i=1$ to $n$, pick $\chi_i,\gamma_i,\delta_i \sample
  \Z_p$ and compute $\hat{g}_i=\hat{g_z}^{\chi_i}
  \hat{g_r}^{\gamma_i}$, $\hat{h}_i=\hat{h_z}^{\chi_i}
  \hat{h_u}^{\delta_i}$.  The private key is defined to be $\sk = \{ (\chi_i,
  \gamma_i,\delta_i ) \}_{i=1}^n $ while the public key is $ \pk=\big(
  \hat{g_z},~\hat{g_r},~\hat{h_z},~ \hat{h_u},~\{ (\hat{g}_i,\hat{h}_i
  ) \}_{i=1}^n \big) \in \hat{\G}^{2n+4}$.
 
\item[\boldmath$\mathsf{Sign}(\sk, (M_1,\dotsc,M_n))$:] To sign a
  vector $(M_1,\dotsc,M_n) \in \G^n$ using $\sk= \{ (\chi_i,
  \gamma_i,\delta_i ) \}_{i=1}^n $, output $\sigma=(z,r,u ) \in \G^3
  $, where $z = \prod_{i=1}^n M_i^{\chi_i} $, $r = \prod_{i=1}^n,
  M_i^{\gamma_i} $ and $ u = \prod_{i=1}^n M_i^{\delta_i}
  $.
  % The signature consists of $\sigma=(z,r,u,v) \in \G^4 $.

\item[\boldmath$\mathsf{SignDerive}(\pk, \{(\omega_i,
  \sigma^{(i)})\}_{i=1}^\ell)$:] Given   $\pk$ as well as
  $\ell$ tuples $(\omega_i,\sigma^{(i)}) $, parse $\sigma^{(i)}$ as
  $\sigma^{(i)}=\big( z_i,r_i,u_i \big)  $ for $i=1$ to
  $\ell$.  Compute and return $\sigma=(z,r,u )$, where $z =
  \prod_{i=1}^\ell z_{i}^{\omega_i}$, $r=\prod_{i=1}^{\ell}
  r_i^{\omega_i}$, $u=\prod_{i=1}^{\ell} u_i^{\omega_i} $.

\item[\boldmath$\mathsf{Verify}(\pk,\sigma, (M_1,\dotsc,M_n))$:] Given
  a signature $\sigma=(z,r,u ) \in \G^3$ and a vector
  $(M_1,\dotsc,M_n)$, return $1$ if and only if $(M_1,\dotsc,M_n)\neq
  (1_{\G},\dotsc,1_{\G})$ and $(z,r,u )$ satisfy
  \begin{align*}
    e(z,\hat{g}_z) \cdot e(r,\hat{g}_r) & = \prod_{i=1}^n e(M_i,\hat{g}_i) \thinspace, & 
		e(z,\hat{h}_z) \cdot e(u,\hat{h}_u) & = \prod_{i=1}^n e(M_i,\hat{h}_i) \enspace.
  \end{align*}
\end{description}

\end{comment}




%Under the SDP assumption in asymmetric pairings, the scheme can be simplified as follows.


\begin{description}

\item[\boldmath$\mathsf{Keygen}(\lambda,n)$:] Given a security
  parameter $\lambda$ and the dimension $n \in \mathbb{N}$ of the
  subspace to be signed, choose bilinear group $(\G,\hat{\G},\G_T)$ of
  prime order $p$. % >2^{\lambda}$.
  Then, choose $\hat{g_z},\hat{g_r}  \sample
  \hat{\G}$.  For $i=1$ to $n$, pick $\chi_i,\gamma_i  \sample
  \Z_p$ and compute $\hat{g}_i=\hat{g_z}^{\chi_i}
  \hat{g_r}^{\gamma_i}$.  The private key is defined to be $\sk = \{ (\chi_i,
  \gamma_i  ) \}_{i=1}^n $ while the public key is $ \pk=\big(
  \hat{g_z},~\hat{g_r}, ~\{  \hat{g}_i 
    \}_{i=1}^n \big) \in \hat{\G}^{ n+2}$.
 
\item[\boldmath$\mathsf{Sign}(\sk, (M_1,\dotsc,M_n))$:] To sign a
   $(M_1,\dotsc,M_n) \in \G^n$ using $\sk= \{ (\chi_i,
  \gamma_i  ) \}_{i=1}^n $, output $\sigma=(z,r  ) \in \G^2
  $, where $z = \prod_{i=1}^n M_i^{\chi_i} $, $r = \prod_{i=1}^n,
  M_i^{\gamma_i} $.
  % The signature consists of $\sigma=(z,r,u,v) \in \G^4 $.

\item[\boldmath$\mathsf{SignDerive}(\pk, \{(\omega_i,
  \sigma^{(i)})\}_{i=1}^\ell)$:] given   $\pk$ as well as
  $\ell$ tuples $(\omega_i,\sigma^{(i)}) $, parse $\sigma^{(i)}$ as
  $\sigma^{(i)}=\big( z_i,r_i \big)  $ for $i=1$ to
  $\ell$.  Compute and return $\sigma=(z,r  )$, where $z =
  \prod_{i=1}^\ell z_{i}^{\omega_i}$, $r=\prod_{i=1}^{\ell}
  r_i^{\omega_i}$.

\item[\boldmath$\mathsf{Verify}(\pk,\sigma, (M_1,\dotsc,M_n))$:] Given
  a signature $\sigma=(z,r  ) \in \G^2$ and a vector
  $(M_1,\dotsc,M_n)$, return $1$ if and only if $(M_1,\dotsc,M_n)\neq
  (1_{\G},\dotsc,1_{\G})$ and $(z,r  )$ satisfy
  \begin{align*}
    e(z,\hat{g_z}) \cdot e(r,\hat{g_r}) & = \prod_{i=1}^n e(M_i,\hat{g}_i) .
  \end{align*}
	
\end{description}



The one-time security of the scheme (in the sense of Definition
\ref{hncs-def} in Appendix \ref{hom-sig-sec-defs}) was proved
\cite{DBLP:conf/crypto/LibertPJY13} under the DP assumption. In
short, the security notion implies the infeasibility of deriving a
signature on a vector outside the subspace spanned by the vectors
authenticated by the signer.  Here, ``one-time'' security means that a
given public key allows signing only one subspace.




We remark that the one-time structure-preserving signature of Section \ref{ot-sig} can be seen  as a special case of the above LHSPS scheme,
in which we fix the first element of the vector to be signed.
The one-time security of this signature scheme can be directly deduced from the security of the LHSPS scheme.


%%%%%%%%%%%%%%%%%%%%%%%%%%%%%%%%%%%%%%%%%%%%%%%%%%%%%%%%%%%%%%%%%%%%%%%%%%%%%%%
%\section{Structure-Preserving Replayable Chosen-Ciphertext Attack secure Encryption} \label{RCCA}
%%%%%%%%%%%%%%%%%%%%%%%%%%%%%%%%%%%%%%%%%%%%%%%%%%%%%%%%%%%%%%%%%%%%%%%%%%%%%%%

%\input{RCCA}

%%%%%%%%%%%%%%%%%%%%%%%%%%%%%%%%%%%%%%%%%%%%%%%%%%%%%%%%%%%%%%%%%%%%%%%%%%%%%%%
\subsection{Strictly Structure-Preserving Trapdoor Commitments} \label{TC}
%%%%%%%%%%%%%%%%%%%%%%%%%%%%%%%%%%%%%%%%%%%%%%%%%%%%%%%%%%%%%%%%%%%%%%%%%%%%%%%
%\input{TC}

In this section, we recall the notion of  Chosen-Message Target Collision Trapdoor Commitment as it was defined by 
Abe~\etal \cite{DBLP:conf/eurocrypt/AbeKOT15}.

\begin{definition}
  A trapdoor commitment scheme $TC$ is a tuple of polynomial-time algorithms $TC = \{\Setup, \KeyGen, \Com, \Verif\}$ that:
  \begin{description}
  \item[\textsf{Setup}$({\lambda}) $:] The parameter generation algorithm takes the security parameter $\lambda$ and outputs a public parameter $\PPP$.
  \item[\textsf{KeyGen}$(\PPP)$:] The key generation algorithm takes $\PPP$ and outputs the commitment key $\ck$.
  \item[\textsf{Com}$(\PPP, \ck, m) $:] The commitment algorithm takes $(\PPP, \ck)$ and a message $m$, then it outputs a commitment $\mathsf{com}$ and an opening information $\open$.
  \item[\textsf{Verify}$(\PPP, \mathsf{com}, m, \open):$] The verification algorithm takes $(\PPP, \mathsf{com}, m, \open)$ and outputs $1$ or $0$.
  \end{description}
\end{definition}

In our construction, we need a trapdoor commitment scheme which satisfies a stronger notion of  Chosen-Message Target Collision Resistance (CM-TCR) than the one considered in \cite[Definition 10]{DBLP:conf/eurocrypt/AbeKOT15}.
%
\begin{definition} \label{ecm-tcr-def}
  A Trapdoor Commitment Scheme $TC$ provides {\bf enhanced chosen-message 
  target collision-resistance} (ECM-TCR) if the advantage
  \begin{multline*}
    \Adv_{TC,\A}^{ECM \textrm{-}TCR}(\lambda) \\ = \Pr \left[                                  
    \begin{array}{l}
      \exists (m^\dagger,\open^\dagger) ~\mathrm{ s.t. }~ (\com^\star, m^\dagger, \open^\dagger) \in Q  \\
    ~ \wedge~ (\mathsf{com}^\star, m^\star, \open^\star) \not \in Q   \\
     ~ \wedge~ \mathsf{Verify}(\ck, \mathsf{com}^\star, m^\star, \open^\star) = 1
    \end{array}
		 \middle|
		  \begin{array}{l}
                                        \PPP \gets \Setup(1^\lambda)\\
                                        \ck \gets \KeyGen(\PPP)\\
                                        (\mathsf{com}^\star, m^\star, \open^\star) \gets \A^{\mathcal{O}_{\ck}  } (\ck) 
                                      \end{array}   
    \right]
  \end{multline*}
  is negligible for any PPT adversary $\A$. 
  Here, $\mathcal{O}_{\ck}$ is an oracle that, 
  given a message $m$,  executes $(\mathsf{com}, \open) \gets \mathsf{Com}(\PPP,\ck, m)$, 
  records $(\mathsf{com}, m, \open)$ in $Q$ and returns $(\mathsf{com}, \open)$.
\end{definition}
We note that Definition \ref{ecm-tcr-def} captures a stronger requirement than the original definition 
\cite[Definition 10]{DBLP:conf/eurocrypt/AbeKOT15} in that the latter only requires that the adversary be unable to open a target 
commitment $\mathsf{com}^\star$ to a different message than the one queried to the oracle $\mathcal{O}_{\ck}$. Here, the adversary is 
also considered successful if it provides a different opening  $\open^\star \neq \open'$ of $\mathsf{com}^\star$ to the same message $m^\star=m^\dagger$ 
as the one queried to $\mathcal{O}_{\ck}$. 
 

We now recall  the Strictly Structure-Preserving  Trapdoor Commitment of Abe~\etal \cite{DBLP:conf/eurocrypt/AbeKOT15} and show that it 
actually satisfies our stronger notion of ECM-TCR security. 

\begin{description}
\item[\boldmath{$TC.\Setup({\lambda},\ell)$}]: On input of a security parameter $\lambda$ and an integer $\ell \in \mathsf{poly}(\lambda)$, 
  the public parameters are generated by choosing    a large prime $p> 2^{\lambda}$,  asymmetric  groups $(\G, \hat{\G}, \G_T)$ of prime order $p$, with a bilinear map $e : \G \times \hat{\G} \to \G_T$ and  group generators $(g, \hat{g}) \in \G \times \hat{\G}$. The algorithm outputs 	
   $$\PPP = (p, \G, \hat{\G}, \G_T, e, g, \hat{g},\ell).$$
  

\item[\boldmath{$TC.\KeyGen(\PPP)$}]:  For $i = 1, \ldots, \ell+2$, choose  $\rho_i \sample \mathbb{Z}_p^*$ and compute 
$$\hat{X}_i \gets \hat{g}^{\rho_i} \qquad \forall i \in \{1,\ldots,\ell+2 \}.$$
   Define the commitment key  $\ck := \{\hat{X}_i\}_{i = 1}^{\ell+2}$ and the trapdoor $tk := \{\rho_i\}_i^{\ell+2}$.
\smallskip   


\item[\boldmath{$TC.\Com(\PPP, \ck, \vec{M})$}]: To commit to  $\vec{\hat{M}} = (\hat{M}_1, \dots, \hat{M}_\ell) \in \hat{\G}^\ell$, conduct the 
following step. \smallskip  \smallskip 
  \begin{enumerate}
  \item[1.] Choose  $w_z \in \mathbb{Z}_p^*$ and compute $g_z = g^{w_z}$.
  \item[2.] For $i = 1, \dots, \ell$, pick $\chi_i \sample \mathbb{Z}_p$ and compute $g_i = g^{\chi_i}$.
  \item[3.] Generate a key pair $(\vk_{pots},\sk_{pots})$ for the partial one-time signature of Section \ref{pots} . Namely, choose 
	$\sk_{pots} \sample (w_z,\chi_1, \dots, \chi_\ell) \in \Z_p^{\ell+1}$ and set 
 	$$\vk_{pots} = (g_z, g_1, \dots, g_\ell) =( g^{w_z},g^{\chi_1},\ldots, g^{\chi_{\ell}} ) \in \G^{\ell+1}.$$
  \item[4.] Choose  $a \sample \mathbb{Z}_p$ and compute $\ovk_{pots} =A = g^{a}$ and $\osk_{pots} = a$.
  \item[5.] Using $\sk_{pots}$,  generate a partial one-time signature  on the message $\vec{\hat{M}}$ \wrt to the one-time  secret key $\osk_{pots}$.
	 To this end, \smallskip 
	    \begin{enumerate}
    \item[a.] Pick $\zeta_1 \in \mathbb{Z}_p$.
    \item[b.] Compute  $(\hat{Z}, \hat{R}) \in \hat{\G}^2$ as a partial one-time signature of $\vec{\hat{M}}$ as
      \begin{align*}
	\hat{Z} &= \hat{g}^{\zeta_1} & \hat{R} = \hat{g}^{a-\zeta_1 w_z}\prod_{i=1}^{\ell} \hat{M}_i^{\chi_i} 
      \end{align*}
    \end{enumerate}
  \item[6.] Generate a commitment to the message.
    \begin{enumerate}
    \item[a.] Set $(m_1, \dots, m_{\ell+2}) \gets (\chi_1, \dots, \chi_\ell, w_z, a)$ 
    \item[b.] Parse $\vec{\ck}$ as $(\hat{X}_1, \dots, \hat{X}_{\ell+2})$.
    \item[c.] Generate a random value $\zeta_2 \gets \mathbb{Z}_p^*$ and compute:
      \begin{align*}
	\hat{C} &= \hat{g}^{\zeta_2}\prod_{i = 1}^{\ell+2}\hat{X}_i^{m_i} & D &= g^{\zeta_2}
      \end{align*}
    \end{enumerate}
  \item[7.] Output the commitment  $\com  = \hat{C}$ as well as the opening information
	\begin{eqnarray} \label{open}
	\open = \big( D, g_z, g_1, \dots, g_\ell, A = g^a, \hat{Z}, \hat{R} \big) \in \G^{\ell+3} \times \hat{\G}^{2}.   
	\end{eqnarray}
	\smallskip 
  \end{enumerate}
  
\item[\boldmath{$TC.\Verif(\ck, \com, \vec{\hat{M}}, \open)$}:] Given  $\com = \hat{C} \in \hat{\G}$, parse 
  $\vec{\hat{M}}$ as $(\hat{M}_1, \dots, \hat{M}_\ell)$ and $\open$ as in (\ref{open}). \smallskip \smallskip   
  \begin{enumerate}
	%$(D, g_z, g_1, \dots, g_\ell, \ovk_{pots} = g^a, \hat{Z}, \hat{R})$.
  \item Set $\vec{N} = (N_1, \dots, N_{\ell+2}) = (g_1, \dots, g_\ell, g_z, A )$
  \item Using $\ovk_{pots} = A \in \G$, return $1$ if the following equalities hold:
    \begin{align} \label{ver-eq-spc}
      e(g, \hat{C}) &= e(D, \hat{g}) \cdot \prod_{i = 1}^{\ell+2} e(N_i, \hat{X}_i) \\ \nonumber e(A, \hat{g}) &= e(g_z, \hat{Z}) \cdot e(g, \hat{R}) \cdot \prod_{i = 1}^\ell e(g_i, \hat{M}_i) .
    \end{align}
		Otherwise, return $0$. 
  \end{enumerate}
\end{description}

We now prove that the above commitment does not only provide CM-TCR security as defined in  \cite{DBLP:conf/eurocrypt/AbeKOT15}, but also  ECM-TCR security. The proof builds on the same ideas as that of  \cite{DBLP:conf/eurocrypt/AbeKOT15} but also  takes advantage of the strong 
unforgeability\footnote{Note that, while \cite{DBLP:conf/asiacrypt/AbeCDKNO12}  only considered the standard notion of unforgeability, it is 
straightforward that their scheme also provides strong unforgeability.}  
of the underlying 
partial one-time signature. 

\begin{theorem} \label{ecm-ctr-sec}
The scheme provides ECM-CTR security under the SXDH assumption.
\end{theorem}
\begin{proof}
For the sake of contradiction, let us assume that a PPT adversary $\A$ can win the game Definition \ref{ecm-tcr-def} with noticeable 
probability. 
We observe that the adversary can only wins in in two mutually exclusive cases.
\begin{itemize}
\item[I.] $\A$ outputs a commitment $\hat{C}^\star \in \hat{\G}$  for which it provides an opening information   
\begin{eqnarray*}
\vec{M}^\star &=& (M_1^\star, \ldots , M_n^\star) \\ 
\open^\star  &=& \big( D^\star, g_z^\star, g_1^\star, \ldots, g_\ell^\star, A^\star , \hat{Z}^\star , \hat{R}^\star \big) ,
\end{eqnarray*}
 where $(D^\star, g_z^\star, g_1^\star, \ldots, g_\ell^\star, A^\star )$ differs from  
the tuple   $(D^\dagger, g_z^\dagger, g_1^\dagger, \ldots, g_\ell^\dagger, A^\dagger )$ returned by $\mathcal{O}_{\ck}$ as part of the opening 
\begin{eqnarray*}
\open^\dagger  &=& \big( D^\dagger, g_z^\dagger, g_1^\dagger, \ldots, g_\ell^\dagger, A^\dagger , \hat{Z}^\dagger , \hat{R}^\dagger \big) ,
\end{eqnarray*}
of $\hat{C}^\star  $ when $\A$ queried $\mathcal{O}_{\ck}$   to obtain a commitment to  $\hat{\vec{M}}^\dagger=(\hat{M}_1^\dagger,\ldots,\hat{M}_{\ell}^\dagger) $.  

\item[II.] $\A$ outputs a commitment $\hat{C}^\star \in \hat{\G}$    which it opens by revealing a pair   
\begin{eqnarray*}
\vec{M}^\star &=& (M_1^\star, \ldots , M_n^\star) \\ 
\open^\star  &=& \big( D^\star, g_z^\star, g_1^\star, \ldots, g_\ell^\star, A^\star , \hat{Z}^\star , \hat{R}^\star \big) ,
\end{eqnarray*}
such that 
 $(D^\dagger, g_z^\dagger, g_1^\dagger, \ldots, g_\ell^\dagger, A^\dagger )=  (D^\star, g_z^\star, g_1^\star, \ldots, g_\ell^\star, A^\star ) $.
In this case, we must have either $\vec{M}^\star \neq \vec{M}^\dagger$ or $(\hat{Z}^\star,\hat{R}^\star) \neq (\hat{Z}^\dagger,\hat{R}^\dagger)$.   
\end{itemize}
\medskip 

Let us first assuming that situation I occurs with noticeable probability. We show that $\A$ can be turned into an algorithm $\B_I$ that breaks the DDH 
assumption in $\hat{G}$ by finding a pair $(Z,R)$ such that $e(Z,\hat{g}) \cdot e(R,\hat{h}) = 1_{\G_T}$ for a given pair 
$(\hat{g},\hat{h}) \in \hat{\G}^2$.  This algorithm $\B_I$ proceeds in the same way as in 
\cite{DBLP:conf/eurocrypt/AbeKOT15}. Namely, it creates the commitment key $\ck$ by choosing $\rho_i,\theta_i \sample \Z_p$ 
and setting $\hat{X}_i=\hat{g}^{\rho_i} \cdot \hat{h}^{\theta_i}$ for each $i \in \{1,\ldots , \ell+2 \}$.  It faithfully answers all queries 
made by $\A$ to $\mathcal{O}_{\ck}$. By hypothesis, $\A$ outputs a commitment $\hat{C}^\star \in \hat{\G}$ as well as an opening $(\vec{M}^\star,\open^\star)$ 
which satisfy the conditions of situation I.  In particular, $\open^\star  = \big( D^\star, g_z^\star, g_1^\star, \ldots, g_\ell^\star, A^\star , \hat{Z}^\star , \hat{R}^\star \big)$ satisfies 
\begin{eqnarray} \label{eq-un}
   e(g, \hat{C}^\star)  = e(D^\star, \hat{g}) \cdot \prod_{i = 1}^{\ell} e(g_i^\star, \hat{X}_i) \cdot 
e(g_z^\star, \hat{X}_{\ell+1}) \cdot e(A^\star ,\hat{X}_{\ell+2}) 
\end{eqnarray}
and the set $Q$ must contain $\open^\dagger  = \big( D^\dagger, g_z^\dagger, g_1^\dagger, \ldots, g_\ell^\dagger, A^\dagger , \hat{Z}^\dagger, \hat{R}^\dagger \big)$ such that 
\begin{eqnarray} \label{eq-deux}
   e(g, \hat{C}^\star)  = e(D^\dagger, \hat{g}) \cdot \prod_{i = 1}^{\ell} e(g_i^\dagger, \hat{X}_i) \cdot 
e(g_z^\dagger, \hat{X}_{\ell+1}) \cdot e(A^\dagger ,\hat{X}_{\ell+2}) .
\end{eqnarray}
Dividing (\ref{eq-deux}) from (\ref{eq-un}), we find that the pair 
\begin{eqnarray*}
Z &=& \Bigl( \frac{D^\star }{D^\dagger}   \Bigr) \cdot 
 \Bigl( \frac{g_{z}^\star }{g_{z}^\dagger } \Bigr)^{ \rho_{\ell+1}} \cdot   \Bigl( \frac{A^\star }{A^\dagger } \Bigr)^{ \rho_{\ell+2}} \cdot \prod_{i=1}^\ell \Bigl( \frac{g_i^\star }{g_i^\dagger} \Bigr)^{ \rho_i}  \\
R &=& \Bigl( \frac{D^\star }{D^\dagger}   \Bigr) \cdot 
 \Bigl( \frac{g_{z}^\star }{g_{z}^\dagger } \Bigr)^{ \theta_{\ell+1}} \cdot   \Bigl( \frac{A^\star }{A^\dagger } \Bigr)^{ \theta_{\ell+2}} \cdot \prod_{i=1}^\ell \Bigl( \frac{g_i^\star }{g_i^\dagger} \Bigr)^{ \theta_i}  \\
\end{eqnarray*}
satisfies  $e(Z,\hat{g}) \cdot e(R,\hat{h}) = 1_{\G_T}$. Moreover, we  have $Z \neq 1_\G$ with all but negligible probability since 
$\{\rho_i\}_{i=1}^\ell$ are completely independent of $\A$'s view.  \medskip 

We now turn to situation II and show that it implies an algorithm $\B_{II}$ that defeats  the strong unforgeability of the partial one-time signature scheme. Algorithm $\B_{II}$ takes as input a POTS verification key $\vk_{pots}=(g_z^\dagger,g_1^\dagger,\ldots,g_{\ell}^\dagger)$ supplied by its own challenger 
in the POTS security game. It generates $\ck = \{ \hat{X}_i \}_{i=1}^{\ell+2}$ by picking $\rho_i \sample \Z_p$ and 
defining $\hat{X}_i=\hat{g}^{\rho_i}$  for each $i \in \{1,\ldots, \ell+2\}$. Letting $Q_c \in \mathsf{poly}(\lambda)$ denote the number of 
queries made by $\A$ to $\mathcal{O}_{\ck}$, $\B_{II}$ draws a random 
index $k^\star \sample \{1,\ldots, Q_c \}$ as a guess that $\A$ will choose to equivocate the commitment $\hat{C}^\dagger$ returned as the 
output of the $k^\star$-th query. It answers all queries to $\mathcal{O}_\ck$ as follows. For each $k \in \{1,\ldots ,Q_c \} \backslash 
\{k^\star \}$, the $k$-th query is answered by faithfully running the commitment algorithm. When the $k^\star$-th query occurs, 
$\B_{II}$ embeds $\vk_{pots}=(g_z^\dagger,g_1^\dagger,\ldots,g_{\ell}^\dagger)$ into the opening of the $k^\star$-th commitment. To this end, it chooses 
 $\zeta \sample \Z_p$ and 
computes 
\begin{eqnarray*}
\hat{C}^\dagger &=& \hat{g}^{\zeta} 
\end{eqnarray*}
Next, $\B_{II}$ queries its own POTS challenger to obtain a signature $(A^\dagger,(\hat{Z},\hat{R}))$ on  the message 
$\hat{\vec{M}}=(\hat{M}_1,\ldots,\hat{M}_{\ell})
 \in \hat{\G}^{\ell}$ queried by $\A$ at this $k^\star$-th query. Upon receiving a partial one-time signature 
$(A^\dagger,(\hat{Z}^\dagger,\hat{R}^\dagger))$ from its POTS challenger, $\B_{II}$ defines
 $(N_1,\ldots,N_{\ell},N_{\ell+1},N_{\ell+2})=(g_1^\dagger,\ldots, g_{\ell}^\dagger ,g_z^\dagger , A^\dagger)$ 
and computes 
\begin{eqnarray*}
 D^\dagger = g^\zeta \cdot \prod_{i=1}^{\ell+2} N_i^{-\rho_i} \in \G,
\end{eqnarray*}
which satisfies $e(g,\hat{C}^\dagger) = e(D^\dagger,\hat{g}) \cdot \prod_{i=1}^{\ell+2} e(N_i,\hat{X}_i) $.  Given that $(A^\dagger,(\hat{Z}^\dagger,\hat{R}^\dagger))$ satisfies 
the second verification equation of (\ref{ver-eq-spc}) by construction, we observe that
$$ \open^\dagger =   \big( D^\dagger, g_z^\dagger , g_1^\dagger, \ldots, g_\ell^\dagger , A^\dagger , \hat{Z}^\dagger , \hat{R}^\dagger \big) $$
forms a valid opening of $\hat{C}^\dagger$. When $\A$ halts, we know that, with probability $1/Q_c$, it chooses to output  a pair 
$(\vec{M}^\star,\open^\star)$ which   opens $\hat{C}^\star = \hat{C}^\dagger$. Given that   
 $(D^\star, g_z^\star, g_1^\star, \ldots, g_\ell^\star, A^\star )=(D^\dagger, g_z^\dagger, g_1^\dagger, \ldots, g_\ell^\dagger, A^\dagger )$ 
  and since we must have
$(\vec{M}^\star,\open^\star) \neq (\vec{M}^\dagger , \open^\dagger) $ by the definition of ECM-TCR security,  we know that  $(\vec{M}^\star,(\hat{Z}^\star,\hat{R}^\star)) \neq 
(\vec{M}^\dagger,(\hat{Z}^\dagger,\hat{R}^\dagger)) $. This  means that $\B_{II}$ can win the game against its POTS challenger by outputting 
  $(\vec{M}^\star,(A^\star,\hat{Z}^\star,\hat{R}^\star)) $. In turn, the result of \cite{DBLP:conf/asiacrypt/AbeCDKNO12} implies that 
	$\B_{II}$ would contradict the DDH assumption in $\G$. \qed 
\end{proof}


%%%%%%%%%%%%%%%%%%%%%%%%%%%%%%%%%%%%%%%%%%%%%%%%%%%%%%%%%%%%%%%%%%%%%%%%%%%%%%%
\section{A  Structure-Preserving CCA2-Secure Public-Key  Cryptosystem With Shorter Publicly Verifiable Ciphertexts } \label{SPCCA}
%%%%%%%%%%%%%%%%%%%%%%%%%%%%%%%%%%%%%%%%%%%%%%%%%%%%%%%%%%%%%%%%%%%%%%%%%%%%%%%
%\input{SPCCA}



In this section, we use the all-but-one hash proof techniques of~\cite{DBLP:conf/tcc/LibertY12} and combine them with the   structure-preserving commitment scheme of Section \ref{TC} and a strongly unforgeable signature scheme. We show that the ECMTCR property of the commitment scheme suffices
to construct the sought-after 
 CCA-2 structure preserving   encryption scheme with publicly verifiable ciphertexts.

\begin{description}

\item[\boldmath{$\KeyGen(\lambda)$}]: \smallskip 
  \begin{enumerate}
  %\item Choose an asymmetric pairing group system $(\G, \hat{\G}, \G_T)$, groups of prime order $p > 2^\lambda$.
  \item %Set $\PPP$ as $(\G, \hat{\G}, \G_T)$.
	 Run the setup algorithm of the commitment scheme in Section \ref{TC} 
	 to obtain $\PPP_{TC} = (p, \G, \hat{\G}, \G_T, e, g, \hat{g},\ell=6) 
	 \leftarrow TC.\Setup({\lambda},6) $, which will be used to commit to messages in $\hat{\G}^6$. 
  \item Generate $(\ck,\tk)\gets TC.\KeyGen(\PPP)$, where 
    ${\ck} \in \hat{\G}^8$ is the commitment key and 
    ${\tk} \in \mathbb{Z}_p^8$ is the trapdoor key which can be erased.
  \item \label{it:SPCCA-key} Choose also group generators $g_1, g_2 \sample \G$ and random values $x_1, x_2 \sample \mathbb{Z}_p$ and set $X = g_1^{x_1}g_2^{x_2}$.
  \item Choose random  $\rho_u,\rho_u' \sample \Z_p$ and a random  $\hat{h} \sample \hat{\G}^2$.
  \item Define $(\hat{\vec{u}}_1, \hat{\vec{u}}_2) $ with $\hat{\vec{u}}_1 = (\hat{g}, \hat{h}) \in \hat{\G}^2$ and $\hat{\vec{u}}_2 =  (\hat{g}^{\rho_u}, \hat{h}^{\rho_u'}) \in \hat{\G}^2$. 
    Note that $\hat{\vec{u}}_1$ and $\hat{\vec{u}}_2$ are linearly independent with overwhelming probability.
  \item \label{it:SPCCA-end} Define $\SK = (x_1, x_2)$ and $$\PK = (g_1, g_2, \hat{\vec{u}}_1, \hat{\vec{u}}_2, X, \PPP_{TC}, {\ck}).$$ 
  \end{enumerate}
  
\item[\boldmath{$\Enc(M,\PK)$}]: To encrypt $M \in \G$, 
conduct the following steps. \smallskip  \smallskip 
  \begin{enumerate}
  \item Generate a  key pair $(\SSK, \SVK) \gets OT1.\KeyGen(\PPP,5)$ 
    for the one-time SPS of Section \ref{ot-sig} so as to sign messages 
    in $\G^5$. Let the resulting key pair consist of $\SSK = \big( \{\chi_i, 
    \gamma_i\}_{i=1}^5, \zeta, \rho \big) \in \mathbb{Z}_p^{14}$ 
	and $ \SVK = \big( \{\hat{g}_i\}_{i =1}^5, \hat{A}  \big) \in \hat{\G}^6,$
    where  $\hat{g}_i=\hat{g}_z^{\chi_i} \cdot \hat{g}_r^{\delta_i}$ and 
    $\hat{A}=\hat{g}_z^{\zeta} \cdot \hat{g}_r^{\rho}$.
  \item Choose $\theta \sample \mathbb{Z}_p$ and compute
    \begin{align*}
      C_0 &= M\cdot X^{\theta}, & C_1 &= g_1^{\theta}, & C_2 &= g_2^{\theta}.
    \end{align*}
  \item \label{it:SPCCA-com} Generate a commitment to 
    $\SVK = (\{\hat{g}_i\}_{i =1}^5, \hat{A})$ and let $$(\com, \open) \gets 
    TC.\Com(\PPP_{TC}, {ck}, \SVK) \in \hat{\G} \times (\G^9 \times \hat{\G}^2)$$
    be the resulting commitment/opening pair.
  \item Define vector $\hat{\vec{u}}_{\com} = \hat{\vec{u}}_2\cdot (1, \com) 
    \in \hat{\G}^2$ as well as the Groth-Sahai CRS $\hat{\mathbf{u}}_{\com}
    =(\hat{\vec{u}}_{\com},\hat{\vec{u}}_1) \in \hat{\G}^2$. 
  \item Pick $r \sample \mathbb{Z}_p$. Compute $\hat{\vec{C}}_{\theta} 
    = \hat{\vec{u}}_{\com}^{\theta} \cdot \hat{\vec{u}}_1^r$.
  \item Using the randomness of the commitment $\vec{C}_{\theta}$, generate  
    proof elements $\vec{\pi}=(\pi_1,\pi_2)=(g_1^r,g_2^r) \in \G^2$ showing 
    that the committed $\theta \in \Z_p$ satisfies the multi-exponentiation 
    equations
    \begin{align*}
      C_1 &= g_1^{\theta} & C_2 &= g_2^{\theta}
    \end{align*}
  \item Output the ciphertext
    \begin{align} \label{SPS-CT} 
      \vec{C} = (\SVK, \com, \open, C_0, C_1, C_2, \hat{\vec{C}}_{\theta}, 
                \vec{\pi}, \vec{\sigma}) \in \G^{16} \times \hat{\G}^{11}
    \end{align}
    where $\vec{\sigma} \leftarrow OT1.\Sig(\SSK, (C_0, C_1, C_2, {\pi}_1,\pi_2)) 
    \in \G^2$.  \medskip  \smallskip

%     Notice that we don't sign the commitments because in the Groth-Sahai proof system and in this very special case, there is only one valid 
% commitment for given proofs.
    
  \end{enumerate}
  
\item[\boldmath{$\Dec(\PK, \vec{C}, \SK)$}]: Parse the ciphertext $\vec{C}$ as in (\ref{SPS-CT}). Then, conduct the following steps. \smallskip  \smallskip   
  \begin{enumerate}
  \item Parse $\PK$ as $({g}_1, {g}_2, X, \PPP_{TC}, \ck)$ and $\SK$ as $(x_1, x_2)$.
 % \item Parse $\vec{C}$ as $ (\SVK, \com, \open, C_0, C_1, C_2, \hat{\vec{C}}_{\theta}, \vec{\pi}, \vec{\sigma})$.
  \item Return $\perp$ if $OT1.\Verif(\PPP,  (C_0, C_1, C_2, \pi_1,\pi_2), \sigma) = 0$.
  \item Return $\perp$ if  $TC.\Verif(\ck, \com, \SVK, \open) = 0$.
  \item Verify that $\vec{\pi}=(\pi_1,\pi_2)$ is a valid Groth-Sahai proof \wrt  $(C_1, C_2, \vec{C}_{\theta}, \com)$. Namely, 
    it should satisfy 
    \begin{eqnarray} \label{ver-eq} 
      E(g_1,\hat{\vec{C}}_{\theta}) &=& E(C_1 , \hat{\vec{u}}_{\com}) \cdot E(\pi_1,\hat{\vec{u}}_1) \\ \nonumber
      E(g_2,\hat{\vec{C}}_{\theta}) &=& E(C_2 , \hat{\vec{u}}_{\com}) \cdot E(\pi_2,\hat{\vec{u}}_1)
    \end{eqnarray}
  \item If  the above verifications all succeed, output $M=C_0/(C_1^{x_1}\cdot C_2^{x_2})$. \medskip \smallskip \smallskip 
  \end{enumerate}
  
\end{description}

\medskip 

\begin{theorem} \label{thm-SXDH}
  The scheme provides IND-CCA2 security under the SXDH assumption.
  More precisely,
  $\Adv^{\mathsf{CCA}}(\lambda)\leq 4\times\Adv^{\mathsf{SXDH}}(\lambda)
  + q_d \times 2^{-\lambda}$.
\end{theorem}

\begin{proof}
  The proof proceeds with a sequence of games that begins with the real game 
  and ends with a game where no advantage is left to the adversary whatsoever. 
  In each game, we call $W_i$ the event that the experiment outputs $1$. 
  The security parameter $\lambda$ is implicitly given in all the games.
  Let $q_d$ denotes the number of decryption queries made by the adversary.

  \begin{description}
  \item[\textsf{Game} $0$:] This is the real game. The adversary is given the public key $\PK$ which contains 
    vectors 
    $(\hat{\vec{u}}_1, \hat{\vec{u}}_2)$ such that
    \begin{align} \label{vec-PK} 
      \hat{\vec{u}}_1 &= (\hat{g}, \hat{h}) \in \hat{\G}^2  &
      \hat{\vec{u}}_2 &=  (\hat{g}^{\rho_u}, \hat{h}^{\rho_u'}) \in \hat{\G}^2,
    \end{align}
    where $\hat{g},\hat{h} \sample \hat{\G}$,  $\rho_u,\rho_u' \sample \Z_p$.
    In the challenge phase, 
    it chooses two messages $M_0,M_1 \in \G$ and obtains a challenge ciphertexts 
    $$  \vec{C}^\star = (\SVK^\star, \com^\star, \open^\star, C_0^\star, C_1^\star, C_2^\star, \hat{\vec{C}}_{\theta}^\star, \vec{\pi}^\star, \vec{\sigma}^\star)  $$
    where, for some random bit $\beta \sample \{0,1\}$,
    \begin{align*}
      C_0^\star &= M_{\beta} \cdot X^{\theta^\star}, & C_1^\star &= g_1^{\theta^\star}, & C_2^\star &= g_2^{\theta^\star},
    \end{align*}
     as well as $(\com, \open) \gets TC.\Com(\PPP_{TC}, {ck}, \SVK)$,
     $\hat{\vec{C}}_{\theta}^\star=\hat{\vec{u}}_{\com^\star}^{\theta^\star}
     \cdot \hat{\vec{u}}_1^{r^\star}$
     and $\vec{\pi}^\star=(\pi_1^\star,\pi_2^\star) =(g_1^{r^\star},g_2^{r^\star})$,
    %
%    \begin{eqnarray*}
%%      \com^\star  &=& \hat{C}^\star  = \hat{g}^{\zeta_2^\star} \cdot \prod_{i = 1}^{\ell }\hat{X}_i^{\chi_i^\star } \cdot \hat{X}_{\ell+1}^{w_z^\star} \cdot \hat{X}_{\ell+2}^{a^\star}  \\
%%      \open^\star &=& \big( D^\star , g_z^\star, g_1^\star, \dots, g_\ell^\star, A^\star  , \hat{Z}^\star, \hat{R}^\star \big)  \\ 
%%                  & =& \big(  g^{\zeta_2^\star },~ g^{w_z^\star}, ~ g^{\chi_1^\star}, ~\ldots , ~ g^{\chi_\ell^\star} , ~ g^{\chi_z^\star} , ~ g^{a^\star}, ~ 		 
%%                       \hat{g}^{\zeta_1^\star} ,  ~\hat{g}^{a^\star-\zeta_1^\star w_z^\star }  \cdot  \prod_{i=1}^{6} {\hat{M}_i^\star~ }^{\chi_i^\star} 
%%                       \big), \\
%      \hat{\vec{C}}_{\theta}^\star &=& \hat{\vec{u}}_{\com^\star}^{\theta^\star} \cdot \hat{\vec{u}}_1^{r^\star}\\
%      \vec{\pi}^\star &=& (\pi_1^\star,\pi_2^\star) =(g_1^{r^\star},g_2^{r^\star}) 
%    \end{eqnarray*}
%    %
%    with $(\hat{M}_1^\star,\ldots,\hat{M}_\ell^\star)=(\hat{g}_1^\star,\ldots,\hat{g}_5^\star,\hat{A}^\star) $
     where 
    $\hat{\vec{u}}_{\com^\star} = \hat{\vec{u}}_2\cdot (1, \com^\star)$.
   
   The adversary's decryption queries are always faithfully answered by the challenger. When the adversary halts, it outputs   
    $\beta' \in \{0,1\}$ and wins if $\beta' =\beta$. In this case, the experiment outputs $1$. Otherwise, it outputs $0$.  
    The adversary's advantage is thus $|\Pr[W_0]-1/2|$. \smallskip \smallskip 

  \item[\textsf{Game} $1$:] In this game, we modify the generation of the public key and define
    \begin{eqnarray} \label{vec-PK-sim} 
      \hat{\vec{u}}_1 &=& (\hat{g}, \hat{h}) \in \hat{\G}^2  \\ \nonumber 
      \hat{\vec{u}}_2 &=&  (\hat{g}^{\rho_u}, \hat{h}^{\rho_u'}) \cdot (1,({\hat{C}^\star})^{-1})  \in \hat{\G}^2.
    \end{eqnarray}
    instead of computing $(\hat{\vec{u}}_1,\hat{\vec{u}}_2)$ as in (\ref{vec-PK}) (note that we may assume w.l.o.g. that $\SVK^\star$ and $\com^\star=\hat{C}^\star$ are 
    generated 
    at the outset of the game).  However, this modification does not affect the adversary's view since $\hat{\vec{u}}_2$ remains uniformly distributed 
    over $\hat{\G}^2$. We have $\Pr[W_1]=\Pr[W_0]$.   \smallskip \smallskip 

  \item[\textsf{Game} $2$:] This game is like Game $1$ except that, if the adversary makes a pre-challenge decryption query 
    $ \vec{C} = (\SVK, \com, \open, C_0, C_1, C_2, \hat{\vec{C}}_{\theta}, \vec{\pi}, \vec{\sigma})  $ such that $\com=\com^\star$, the experiment halts and 
    outputs a random bit. Since Game $2$ is identical to Game $1$ until this event $F_2$ occurs, we have the inequality 
    $|\Pr[W_2]-\Pr[W_1]| \leq \Pr[F_2]$. Moreover, since $\com^\star$ was chosen uniformly in $\hat{\G}$ and remains independent of $\A$'s view until 
    the challenge phase, we have $ |\Pr[W_2]-\Pr[W_1]| \leq \Pr[F_2] \leq q_d/p$. \smallskip \smallskip 


  \item[\textsf{Game} $3$:] This game is like Game $2$ but we modify the decryption oracle. Namely, if the adversary makes a post-challenge decryption query 
    for a valid ciphertext
    $ \vec{C} = (\SVK, \com, \open, C_0, C_1, C_2, \hat{\vec{C}}_{\theta}, \vec{\pi}, \vec{\sigma})  $
    such that $\com=\com^\star$ but $(\SVK,\open) \neq (\SVK^\star,\open^\star)$, the experiment halts and outputs a random bit. If we call $F_3$ the latter event, 
    we have $|\Pr[W_3]-\Pr[W_2]| \leq \Pr[F_3]$. As shown by Lemma  \ref{lemma-game3},   event $F_3$   implies an adversary $\B_3$ against the ECM-TCR property (as formalized 
		by Definition \ref{ecm-tcr-def}) of the 
     trapdoor commitment in Section \ref{TC}, which contradicts the SXDH assumption. We thus have    
    $|\Pr[W_3]-\Pr[W_2]| 
      \leq \mathbf{Adv}_{TC,\B_3}^{\mathsf{ECM}\textsf{-}\mathsf{TCR}}(\lambda) 
      \leq   \mathbf{Adv}_{\B_3}^{\mathrm{SXDH}}(\lambda)$.   
    \smallskip \smallskip 


  \item[\textsf{Game} $4$:] We modify again the decryption oracle in  post-challenge decryption queries. After the challenge phase, if the adversary $\A$  queries the decryption of a ciphertext 
    $ \vec{C} = (\SVK, \com, \open, C_0, C_1, C_2, \hat{\vec{C}}_{\theta}, \vec{\pi}, \vec{\sigma})  $ such that we have $(\com,\open)=(\com^\star,\open^\star)$
    but $(C_0,C_1,C_2,\pi_1,\pi_2) \neq (C_0^\star,C_1^\star,C_2^\star,\pi_1^\star,\pi_2^\star)$, the experiment halts and outputs a random 
    bit. If we call $F_4$ this event, we have the inequality  $|\Pr[W_4]-\Pr[W_3]| \leq \Pr[F_4]$ since Game $4$ is identical to Game $3$ until $F_4$ occurs.
    Moreover,   $F_4$ would contradict the strong unforgeability of the one-time structure-preserving signature  and thus the DP assumption. This implies  
    $|\Pr[W_4]-\Pr[W_3]| \leq \mathbf{Adv}_\B^{\mathsf{SUF}\textsf{-}\mathsf{OTS}}(\lambda)   \leq   \mathbf{Adv}_\B^{\mathrm{DP}}(\lambda)$.
    \smallskip 

  \item[\textsf{Game} $5$:] We introduce another modification in the decryption oracle.  We  reject all ciphertexts $ \vec{C} = (\SVK, \com, \open, C_0, C_1, C_2, \hat{\vec{C}}_{\theta}, \vec{\pi}, \vec{\sigma})  $ 
    such that
    \begin{multline} \label{event-F5}
      (\com,\open)=(\com^\star,\open^\star)  \quad \wedge \quad \\ (C_0,C_1,C_2,\pi_1,\pi_2) = (C_0^\star,C_1^\star,C_2^\star,\pi_1^\star,\pi_2^\star) 
      \quad  \wedge \quad \hat{\vec{C}}_{\theta} \neq \hat{\vec{C}}_{\theta}^\star .
    \end{multline}  
    Let $F_5$ be the event that the decryption oracle rejects a ciphertext that would not have been rejected in Game $4$.   
    We argue that $\Pr[W_5] = \Pr[W_4]$ since Game $5$ is identical to Game $4$ until event $F_5$ occurs and we have $\Pr[F_5]=0$. 
    Indeed, 
    % since the vectors $\vec{u}_{\com^\star}$ and $\vec{u}_1$ are linearly independent, they span the entire vector space $\hat{\G}^2$ which 
    % means that, 
    for a given $(C_1^\star,C_2^\star,\pi_1^\star,\pi_2^\star) \in \G^4$, there exists only one commitment $\hat{\vec{C}}_{\theta}^\star \in \hat{\G}^2$ that satisfies the equalities (\ref{ver-eq}). This follows from the fact that, since  
    $(C_1^\star,C_2^\star,\pi_1^\star,\pi_2^\star)=(g_1^{\theta^\star},g_2^{\theta^\star},g_1^{r^\star},g_2^{r^\star})$, relations 
    (\ref{ver-eq}) can be written
    \begin{eqnarray*}  
      E(g_1,\hat{\vec{C}}_{\theta}^\star) &=& E(g_1^{\theta^\star} , \hat{\vec{u}}_{\com}) \cdot E(g_1^{r^\star},\hat{\vec{u}}_1) = E(g_1 , \hat{\vec{u}}_{\com}^{\theta^\star}) \cdot E(g_1,\hat{\vec{u}}_1^{r^\star}) \\ \nonumber
      E(g_2,\hat{\vec{C}}_{\theta}^\star) &=& E(g_2^{\theta^\star} , \hat{\vec{u}}_{\com}) \cdot E(g_2^{r^\star},\hat{\vec{u}}_1) =E(g_2 , \hat{\vec{u}}_{\com}^{\theta^\star}) \cdot E(g_2, \hat{\vec{u}}_1^{r^\star})
    \end{eqnarray*}
    which uniquely determines the  only commitment $\hat{\vec{C}}_{\theta}^\star= \hat{\vec{u}}_{\com}^{\theta^\star} \cdot \hat{\vec{u}}_1^{r^\star} \in \hat{\G}^2$ that satisfies (\ref{ver-eq}). 
    This shows that $\Pr[F_5] = 0$, as claimed. 
    \smallskip \smallskip 





  \item[\textsf{Game} $6$:] In this game, we modify the distribution of the public key.  Namely, instead of generating 
    the vectors $(\hat{\vec{u}}_1, \hat{\vec{u}}_2)$ as in (\ref{vec-PK-sim}), we 
    set 
    \begin{align} \label{vec-PK-sim-bis} 
      \hat{\vec{u}}_1 & = (\hat{g}, \hat{h}) \in \hat{\G}^2  &
      \hat{\vec{u}}_2 & = (\hat{g}^{\rho_u}, \hat{h}^{\rho_u}) 
                          \cdot (1,{\hat{C}^{\star{-1}}})  \in \hat{\G}^2.
    \end{align}
    Said otherwise,  $\hat{\vec{u}}_2$ is now the product of two terms, the first one of which lives in the 
    one-dimensional subspace spanned by $\hat{\vec{u}}_1$. Under the DDH assumption in $\hat{\G}$, this modified  
    distribution of $\PK$ should have not noticeable impact on the adversary's behavior.  
    A straightforward reduction shows 
    that $|\Pr[W_6]-\Pr[W_5] | \leq \mathbf{Adv}^{\mathrm{DDH}}_\B (\lambda)$. Note that, although the vectors $(\hat{\vec{u}}_{\com^\star},\hat{\vec{u}}_1) \in \hat{\G}^2$ are 
    no longer linearly independent, $\hat{\vec{C}}_{\theta}^\star = \hat{\vec{u}}_1^{\rho_u \cdot \theta^\star +r^\star}$ remains the only commitment 
    that satisfies the verification equations  for a given tuple $(C_1^\star,C_2^\star,\pi_1^\star,\pi_2^\star)$.
    

    \smallskip \smallskip 

  \item[\textsf{Game} $7$:] In this game, we  modify the challenge ciphertext and replace the NIZK proof $\vec{\pi}^\star=(\pi_1^\star,\pi_2^\star) \in \G^2$ by a simulated proof which is produced 
    using $\rho_u \in \Z_p$ as a simulation trapdoor. Namely, $(\hat{\vec{C}}_\theta^\star,\vec{\pi}^\star)$ is obtained by picking $r \sample \Z_p$ and  computing
    \begin{eqnarray*}
      \hat{\vec{C}}_{\theta}^\star &=& \vec{u}_1^{r},   \qquad \qquad \quad
                                 \pi_1^\star  =  g_1^{r} \cdot {C_1^\star }^{-\rho_u} , \qquad \qquad \quad 
                                 \pi_2^\star  =  g_2^{r} \cdot {C_2^\star }^{-\rho_u}
    \end{eqnarray*}
    Observe that, although $(\hat{\vec{C}}_\theta^\star,\pi_1^\star,\pi_2^\star)$ are generated without using the 
		witness $\theta^\star = \log_{g_1}(C_1^\star) =
    \log_{g_2}(C_2^\star)$,  the NIZK property of 
    GS proofs ensures that 
    their distribution remains exactly as in Game $6$: indeed, if we define $\tilde{r} =r -\rho_u \cdot \theta^\star$, we have
    \begin{eqnarray*}
      \hat{\vec{C}}_{\theta}^\star &=& \hat{\vec{u}}_{\com^\star}^{\theta^\star} \cdot \hat{\vec{u}}_1^{\tilde{r}},   \qquad \qquad \quad
                                 \pi_1^\star  =  g_1^{\tilde{r}} , \qquad \qquad \quad 
                                 \pi_2^\star  =  g_2^{\tilde{r}} ,
    \end{eqnarray*}
    which implies $\Pr[W_7]=\Pr[W_6]$.  
    \smallskip \smallskip


  \item[\textsf{Game} $8$:]  We modify the generation of the challenge ciphertext, which is generated using the private key $\SK=(x_1,x_2)$ instead
    of the public key: Namely, the challenger computes 
    \begin{align*}
      C_1^\star &= g_1^{\theta^\star}, & C_2^\star &= g_2^{\theta^\star},    &  C_0^\star &= M_{\beta} \cdot {C_1^\star}^{x_1} \cdot  {C_2^\star}^{x_2} , 
    \end{align*} 
    while $(\hat{\vec{C}}_\theta^\star,\pi_1^\star,\pi_2^\star)$ are computed using the NIZK simulation trapdoor $\rho_u \in \Z_p$ as in Game $7$. 
    This modification does not affect the adversary's view since the ciphertext retains exactly the same distribution as in Game $7$. 
    We have $\Pr[W_8]=\Pr[W_7]$.  \smallskip \smallskip

  \item[\textsf{Game} $9$:] We modify again the distribution of the challenge ciphertext which is obtained as 
    \begin{align*}
      C_1^\star &= g_1^{\theta_1^\star}, & C_2^\star &= g_2^{\theta_2^\star},    &  C_0^\star &= M_{\beta} \cdot {C_1^\star}^{x_1} \cdot  {C_2^\star}^{x_2} , 
    \end{align*} 
    for random and independent $\theta_1^\star,\theta_2^\star \sample \Z_p$, 
    while the NIZK proof $(\hat{\vec{C}}_\theta^\star,\pi_1^\star,\pi_2^\star)$ is simulated using $\rho_u \in \Z_p$ as in Game $8$.  Since 
    the witness $\theta^\star \in \Z_p$ was not used anymore in Game $8$, a straightforward reduction shows that  any noticeable change in $\A$'s output distribution implies a DDH distinguisher in $\G$. We have 
    $|\Pr[W_9]-\Pr[W_8]| \leq   \mathbf{Adv}_{\B,\G}^{\mathrm{DDH}}(\lambda)$.  \smallskip \smallskip

    

  \end{description}

  In the final game, it is easy to see that $\Pr[W_9]=1/2$ since the challenge ciphertext does not carry any information about $\beta \in \{0,1\}$. 
  Indeed, we have 
  \begin{align*}
    C_1^\star &= g_1^{\theta_1^\star}, & C_2^\star &= g_2^{\theta_1^\star + \theta_1'},    &  C_0^\star &= M_{\beta} \cdot X^{\theta_1^\star} 
                                                                                                          \cdot  {g_2}^{\theta_1' \cdot x_2} , 
  \end{align*} 
  for some random $\theta_1' \in_R \Z_p$, which implies that 		the term ${g_2}^{\theta_1' \cdot x_2}$ perfectly hides $M_\beta$ in the expression of $C_0^\star$. This follows from the fact
  that $x_2 \in \Z_p$ is perfectly independent of the adversary's view.	Indeed, the public key leaves $x_2 \in \Z_p$ completely undetermined as it only reveals $X=g_1^{x_1} g_2^{x_2}$. During the game, decryption queries are guaranteed not to reveal anything about $x_2$ since all 
  NIZK proofs $(\hat{\vec{C}}_{\theta},\pi_1,\pi_2)$ take place on   Groth-Sahai CRSes $(\hat{\vec{u}}_{\com},\hat{\vec{u}}_1)$ which are perfectly sound (as 
  they span the entire vector space $\hat{\G}^2$)
  whenever $\com \neq \com^\star$. This implies that, although the adversary can see a simulated NIZK proof $(\hat{\vec{C}}_\theta^\star,\pi_1^\star,\pi_2^\star)$ for 
  a false statement in the challenge phase, it remains  unable to trick the decryption oracle into accepting a  ciphertext 
  $ \vec{C} = (\SVK, \com, \open, C_0, C_1, C_2, \hat{\vec{C}}_{\theta}, \vec{\pi}, \vec{\sigma})  $ such that $\log_{g_1}(C_1) \neq \log_{g_2}(C_2)$. 
  As a consequence, the adversary does not learn anything about $x_2$ from responses of the decryption oracle. 
  \qed
\end{proof}



\begin{lemma}\label{lemma-game3}
  In Game 3, % of the proof of Theorem~\ref{thm-SXDH}, 
   % equal to the probability of computing an enhanced chosen message target 
  %collison againt 
	there exists an ECM-TCR adversary with  advantage $\epsilon \geq \Pr[F_3]$ against  the  trapdoor commitment scheme of Section~\ref{TC} and which 
	runs in about the same time as $\A$.
%  If there exists an PPT adversary $\Adv_{2,3}$ which can produce the evenement $F_3$ with non-negligible probability $Adv_{F_3}$, then there exists an PPT adversary $\Adv_{ECM-CTR}$ which can win with a non-negligible probability against the enhanced chosen-message target collision-resistance of the underlying trapdoor commitment. 
\end{lemma}
\begin{proof}
  Let $\A$ be an adversary against the SP-CCA encryption scheme as in the proof of
  Theorem~\ref{thm-SXDH} and let the event $F_3$ be defined as in Game 3. Then, we 
  build an adversary $\B_3$ against the ECM-CTR security of the 
  structure-preserving trapdoor commitment defined in Section~\ref{TC}
  which efficiently runs $\A$.
  
  The challenger $\B_3$ is given the public parameter $\PPP_{TC}$ and a 
  commitment key $\ck$ generated as in the trapdoor commitment scheme 
  as well as an access to a commit-open oracle $\mathcal{O}_{\ck}$ as
  defined in Definition~\ref{ecm-tcr-def}. 
  Then, $\B_3$ runs step~\ref{it:SPCCA-key} to step~\ref{it:SPCCA-end}
  of the key generation algorithm of the encryption scheme to get $\PK$
  and $\SK=(x_1,x_2)$ as specified in Game 2 and Game 3.
  
  The adversary $\A$ is given $\PK$ and $\B_3$ is easily able to answer to 
  $\A$'s decryption queries as described in Game 2 and Game 3 thanks to 
  $\SK$. In order to compute the challenge ciphertext given $\{m_0,m_1\}$, 
  $\B_3$ picks $\beta\sample\{0,1\}$, runs all the steps of the encryption
  algorithm with $m_\beta$ except for step~\ref{it:SPCCA-com} for which 
  $\B_3$ queries $\mathcal{O}_{\ck}$ on $\SVK^\star$ to get 
  $(\com^\star, \open^\star)$. The computed ciphertext $\vec{C}^\star$
  is then given to $\A$.
  
  Assuming that $F_3$ occurs, which means that $\A$ makes a post-challenge 
  decryption query for a valid ciphertext $ \vec{C} = (\SVK, \com, \open, 
  C_0, C_1, C_2, \hat{\vec{C}}_{\theta}, \vec{\pi}, \vec{\sigma})  $
  such that $\com=\com^\star$ but $(\SVK,\open) \neq (\SVK^\star,\open^\star)$,
  the challenger simply outputs $(\com^\star,\SVK,\open)$.
  
  Obviously, we have $TC.\Verif(\ck, \com^\star, \SVK, \open) = 1$ since 
  $\vec{C}$ is valid. However, during the ECM-TR experiment $\B_3$ only 
  chose a single message $\SVK^\star$ so that there is only one target in
  $Q=\{(\com^\star,\SVK^\star,\open^\star)\}$. Moreover, since we also have
  $(\com^\star,\SVK,\open)\not\in Q$, we find
  $\Pr[F_3]=\Adv_{TC,\B_3}^{ECM\textrm{-}TCR}(\lambda)$.
  \qed
%  
%   is so that 
%  \\
%  
%  
%  
%  \begin{enumerate}
%  \item $\exists(m^\dagger, \open^\dagger) ~\mathrm{ s.t. }~ (\com^*, m^\dagger, \open^\dagger) \in Q$.
%  \item $(\com^*, m^*, \open^*) \not \in Q$.
%  \item $\Verif(\PP, \com^\star, m^\star, \open^\star) = \True$
%  \end{enumerate}
%
%  We simulate the encryption scheme for $\Adv_{F_3}$.
%  We run the key generation algorithm $SPCCA.\KeyGen(1^\lambda)$ in our construction, 
%  but instead of generate new public parameter for the underlying trapdoor commitment, we use $\PPP_{TC}$.
%  In the encryption algorithm during the challenge phase, 
%  instead of use $\ck$ we require the commit oracle to generate the commitment $(\com^\star, \open^\star)$ of $\SVK^\star$.
%  Thus we have $(\com^\star, \SVK^\star, \open^\star) \in Q$ and it is the only element in $Q$.
%  The decryption algorithm remains the same.
%
%  Since the modification of the ciphertext generation in the challenge phase does not change the view of the adversary $\Adv_{F_3}$,
%  it will still output with non-negligible probability $\vec{C} = (\SVK, \com, \open, C_0, C_1, \vec{\hat{C}}_{\theta}, \vec{\pi}, \vec{\sigma})$ such that
%  $\com = \com^\star$ but $\open \neq \open^\star$.
%  We submit $(\com, \SVK, \open)$ as an challenge text for the enhanced chosen-message target collision-resistance (ECM-TCR) of the commitment scheme.
%  
%  Since $(\com^\star, m^\star, \open^\star)$ is the only element in the hash table $Q$.
%  $\com = \com^\star$ implies that $(\com, \SVK^\star, \open^\star) \in Q$ and 
%  $\open \neq \open^\star$ implies that $(\com, \SVK, \open) \not \in Q$.
%  And as $\vec{C} = (\SVK, \com, \open, C_0, C_1, \vec{\hat{C}}_{\theta}, \vec{\pi}, \vec{\sigma})$ is a valid ciphertext which means $\Verif(\PPP_{TC}, \com, \SVK, \open) = \True$.
%  Thus $(\com, \SVK, \open)$ is a success chanllenge commitment for the ECM-TCR security of the commitment scheme.
\end{proof}



\section{A Randomizable RCCA-Secure Construction} \label{RCCA-scheme}

\input{RCCA-SXDH.tex}

\input{RCCA-SXDH-proof.tex}

\bibliographystyle{abbrv}
\bibliography{main-SPCCA}


\newpage 

\appendix 



%%%%%%%%%%%%%%%%%%%%%%%%%%%%%%%%%%%%%%%%%%%%%%%%%%%%%%%%%%%%%%%%%%%%%%%%%%%%%%%
\section{Groth-Sahai Proofs}\label{GS-proofs}
%%%%%%%%%%%%%%%%%%%%%%%%%%%%%%%%%%%%%%%%%%%%%%%%%%%%%%%%%%%%%%%%%%%%%%%%%%%%%%%

 Our constructions use Groth-Sahai proofs for pairing
product equations (PPE) of the form:
$$\prod_{j=1}^n e(\mathcal{A}_j,\mathcal{Y}_j) \prod_{j=1}^n e(\mathcal{X}_i,\mathcal{B}_i)
\prod_{i=1}^m \prod_{j=1}^n
e(\mathcal{X}_i,\mathcal{Y}_j)^{\gamma_{i,j}} = t_T,$$ where
$\mathcal{X}_i, \mathcal{Y}_j$ are variables in $\G_1$ and $\G_2$,
respectively,
 and $\mathcal{A}_j \in\G,\mathcal{B}_i\in \hat{\G}$ and $t_T\in\G_T$ are constants for $i\in \{1,\ldots,m \}$ and $j\in \{1,\ldots,n \}$.

A non-interactive witness indistinguishable (NIWI) proof system is a tuple of four algorithms
$(\mathsf{Setup},\mathsf{Prove},\mathsf{VerifyProof})$. 
$\mathsf{Setup}$ outputs a common reference string (CRS) $crs$,
$\mathsf{Prove}$ first generates commitments of variables and
constructs proofs that these variables satisfy the statement, and
$\mathsf{VerifyProof}$ verifies the proof. Such a proof system should satisfy correctness,
soundness and witness-indistinguishability. \emph{Correctness}
requires that honestly generated proofs
for  true statements be always accepted by the verifier. \emph{Soundness} guarantees that cheating
provers can only prove true statements with all but negligible probability.
\emph{Witness-indistinguishability} requires the existence of   an efficient
simulator $\mathsf{GSSimSetup}$  that produces a common
reference string (CRS) $crs'$ which is computationally
indistinguishable from a normal $crs$. When commitments are computed
using $crs'$, they are perfectly hiding and the corresponding
proofs are witness indistinguishable:  i.e., so long as a statement as several witnesses, the proof 
 leaks no information on which specific witness is used to generate it.
\emph{Zero-knowledge} additionally requires the existence of an
algorithm $\mathsf{GSSimProve}$ that, given a simulated CRS $crs'$
and some trapdoor information $\tau$, generates a simulated proof of
the statement without using the witnesses and in such a way that the
proof is indistinguishable from a real proof. 


In the perfect soundness setting, a  CRS  $(\vec{u_1},\vec{u_2},\vec{\hat{u}_1},\vec{\hat{u}_2})$ consists of vectors $\vec{u_1}=(g,u_{12})$, $\vec{u_2}=(h,u_{22}) \in \G^2$ and 
$\vec{\hat{u}_1}=(\hat{g},\hat{u}_{12})$, $\vec{\hat{u}_2}=(\hat{h},\hat{u}_{22}) \in \hat{\G}^2$ that are linearly dependent. Namely,  there exist $\zeta,\hat{\zeta} \in \Z_p$ for which $\vec{u_2}=\vec{u_1}^\zeta$ and $\vec{\hat{u}_2} =\vec{\hat{u}_1}^{\hat{\zeta}}$. Moreover, NIWI proofs for pairing product equations are perfectly sound (meaning that proofs for false
statements do not exist)
and the pair $(x,y)=(\log_g(u_{12}),\log_{\hat{g}}(\hat{u}_{12})) \in \Z_p^2$ can 
serve as an extraction trapdoor to extract committed group elements $X \in \G$ and $\hat{X} \in \hat{\G}$ from their commitments 
$\vec{C}_X=(1,X) \cdot \vec{u_1}^{\theta_1} \cdot \vec{u_2}^{\theta_2}$, $\vec{\hat{C}}_X=(1,\hat{X}) \cdot \vec{\hat{u}_1}^{\theta_3} \cdot \vec{\hat{u}_2}^{\theta_4}$.   
In the perfect witness indistinguishability setting, $(\vec{u_1},\vec{u_2})$ are linearly independent vectors, just like $(\vec{\hat{u}_1},\vec{\hat{u}_2})$. In this case,
$\vec{C}_X=(1,X) \cdot \vec{u_1}^{\theta_1} \cdot \vec{u_2}^{\theta_2}$ and $\vec{\hat{C}}_X=(1,\hat{X}) \cdot \vec{\hat{u}_1}^{\theta_3} \cdot \vec{\hat{u}_2}^{\theta_4}$ are 
perfectly hiding commitments to $X$ and $\hat{X}$, respectively, and non-interactive proofs for pairing product equations are perfectly witness indistinguishable. Under the SXDH
assumption, no PPT adversary can distinguish a perfectly sound  CRS from a perfectly hiding CRS.  

Regardless of which kind of CRS is used, linear pairing product equations (i.e., where $\gamma_{ij}=0$ for all $i,j$) have proofs in $\G^2 \times \hat{\G}^2$ when they involve witnesses in both $\G$ and $\hat{\G}$. When 
all witnesses are in $\G$, proofs live in $\hat{\G}^2$. 
For quadratic statement (i.e., where $\gamma_{ij}\neq 0$ for some $i,j$)
the proof are in $\G^4 \times \hat{\G}^4$. 
 




%%%%%%%%%%%%%%%%%%%%%%%%%%%%%%%%%%%%%%%%%%%%%%%%%%%%%%%%%%%%%%%%%%%%%%%%%%%%%%%
\section{Definitions for Involved Primitives} 
%%%%%%%%%%%%%%%%%%%%%%%%%%%%%%%%%%%%%%%%%%%%%%%%%%%%%%%%%%%%%%%%%%%%%%%%%%%%%%%

%------------------------------------------------------------------------------
\subsection{Definitions for Linearly Homomorphic Structure-Preserving
  Signatures}\label{hom-sig-sec-defs}    
%------------------------------------------------------------------------------

Let $(\G,\G_T)$ be groups of prime order $p$ such that a bilinear map
$e:\G \times {\G} \rightarrow \G_T$ can be efficiently computed.

A signature scheme is \emph{structure-preserving} \cite{DBLP:journals/iacr/AbeHO10,DBLP:conf/crypto/AbeFGHO10}
if messages, signatures and public keys all live in the group $\G$.
In linearly homomorphic structure-preserving signatures, the message
space $\mathcal{M}$ consists of pairs $\mathcal{M} := \mathcal{T}
\times \G^n$, for some $n \in \mathbb{N}$, where $\mathcal{T}$ is a
tag space.  Depending on the application, one may want the tags to be
group elements or not. In this paper, they can be arbitrary strings.
    
\begin{definition}\label{hncs-def}  
  A \emph{linearly homomorphic structure-preserving signature scheme
    over $(\G,\G_T)$} is a tuple of efficient algorithms
  $\Sigma=(\mathsf{Keygen},\mathsf{Sign},\mathsf{SignDerive},\mathsf{Verify})$
  for which the message space consists of $\mathcal{M} := \mathcal{T}
  \times \G^n$, for some integer $n \in \mathsf{poly}(\lambda)$ and
  some set $\mathcal{T}$, and with the following specifications.
  \begin{description}  
  \item[\boldmath$\mathsf{Keygen}(\lambda,n)$] is a randomized  
    algorithm that takes in a security parameter $\lambda \in  
    \mathbb{N}$ and an integer $n \in \mathsf{poly}(\lambda)$ denoting  
    the dimension of vectors to be signed. It outputs a key pair  
    $(\pk,\sk)$, where $\pk$ includes  the description of a tag   
    space $\mathcal{T}$, where each tag serves as a file identifier.   
  \item[\boldmath$\mathsf{Sign}(\mathsf{sk},\tau,\vec{M} )$] takes as
    input a private key $\sk$, a file identifier $\tau \in
    \mathcal{T}$ and a vector of group elements
    $\vec{M}=(M_1,\dotsc,M_n) \in \G^n$. It outputs a signature
    $\sigma \in \G^{n_{s}}$, for some $n_s \in
    \mathsf{poly}(\lambda)$.
  \item[\boldmath$\mathsf{SignDerive}(\mathsf{pk},\tau,\{(\omega_i,
    \sigma^{(i)})\}_{i=1}^\ell )$] is a homomorphic signature 
    derivation algorithm.  It inputs a public key $\mathsf{pk}$, a
    file identifier $\tau$ as well as $\ell$ pairs
    $(\omega_i,\sigma^{(i)}) $, each of which consists of a
    coefficient $\omega_i \in \Z_p$ and a signature $\sigma^{(i)} \in
    \G^{n_{s}}$. It outputs a signature $\sigma \in \G^{n_{s}}$ on the
    vector $\vec{M}=\prod_{i=1}^{\ell} \vec{M}_i^{~ \omega_i }$, where
    $\sigma^{(i)}$ is a signature on $\vec{M}_i$.
  \item[\boldmath$\mathsf{Verify}(\mathsf{pk},\tau,\sigma,\vec{M})$]
    is a  verification algorithm that takes as input a
    public key $\pk$, a file identifier $\tau \in \mathcal{T}$, a
    signature $\sigma$ and a vector $\vec{M}=(M_1,\dotsc,M_n)$. It
    outputs $0$ or $1$. 
		%depending on whether $\sigma$ is deemed valid
    %or not.
  \end{description}  
\end{definition}  
  
In a \emph{one-time} linearly homomorphic SPS, the tag $\tau$ can be
omitted in the specification as a given key pair $(\pk,\sk)$ only
allows signing one linear subspace.

As in all linearly homomorphic signatures, the desired security notion
mandates the adversary's inability to come up with a valid triple
$(\tau^\star,\vec{M}^\star,\sigma^\star)$ for a new file identifier
$\tau^\star$ or, if $\tau^\star$ appeared in signatures generated by
the signing oracle, for a vector $\vec{M}^\star$ outside the linear
span of the vectors that have been legitimately signed for the tag
$\tau^\star$.



%%%%%%%%%%%%%%%%%%%%%%%%%%%%%%%%%%%%%%%%%%%%%%%%%%%%%%%%%%%%%%%%%%%%%%%%%%%%%%%
\subsection{Quasi-Adaptive NIZK Arguments}  \label{QA-NIZK}
%%%%%%%%%%%%%%%%%%%%%%%%%%%%%%%%%%%%%%%%%%%%%%%%%%%%%%%%%%%%%%%%%%%%%%%%%%%%%%%

Quasi-Adaptive NIZK (QA-NIZK) proofs \cite{DBLP:conf/asiacrypt/JutlaR13} are NIZK proofs where the CRS is
allowed to depend on the specific language for which proofs have to be
generated. The CRS is divided into a fixed part $\Gamma$, produced by
an algorithm $\K_0$, and a language-dependent part $\psi$. However,
there should be a single simulator for the entire class of languages.

Let $\lambda$ be a security parameter. For public parameters $\Gamma$
produced by $\K_0$, let $\mathcal{\D}_{\Gamma}$ be a probability
distribution over a collection of relations $\mathcal{R}=\{R_{\rho}\}$
parametrized by a string $\rho$ with an associated language
$$\mathcal{L}_{\rho}=\{x \mid \exists w : R_{\rho}(x,w)=1\}.$$

A tuple
of algorithms $(\K_0,\K_1,\PP,\VV)$ is a QA-NIZK proof system for
$\mathcal{R}$ if there exists a PPT simulator $(\SIM_1,\SIM_2)$ such
that, for any PPT adversaries $\A_1,\A_2$ and $\A_3$, we have the
 properties hereunder. \\
\indent 
We assume that the CRS $\psi$ contains an encoding of $\rho$, which is
thus available to $\VV$.  The definition of Quasi-Adaptive
Zero-Knowledge requires a single simulator for the entire family of
relations $\mathcal{R}$.
	  
\begin{description}  
\item[Quasi-Adaptive Completeness:]   
  \begin{multline*}   
    \Pr[\Gamma \leftarrow \K_0(\lambda); ~\rho \leftarrow
    \D_{\Gamma};~\psi \leftarrow \K_1(\Gamma,\rho);\\ ~(x,w)
    \leftarrow \A_1(\Gamma,\psi);~\pi \leftarrow
    \PP(\psi,x,w): \VV(\psi,x,\pi)=1 ~\textrm{ if }
    R_{\rho}(x,w)=1]=1\enspace.   
  \end{multline*}   
\item[Quasi-Adaptive Soundness:]   
  \begin{multline*}   
    \Pr[\Gamma \leftarrow \K_0(\lambda); ~\rho \leftarrow
    \D_{\Gamma};~\psi \leftarrow \K_1(\Gamma,\rho); ~(x,\pi)
    \leftarrow \A_2(\Gamma,\psi):\\ \VV(\psi,x,\pi)=1
    ~\wedge~ \neg (\exists w : R_{\rho}(x,w)=1)] \in
    \mathsf{negl}(\lambda)\enspace.   
  \end{multline*}   
\item[Quasi-Adaptive Zero-Knowledge:]     
  \begin{multline*}   
    \Pr[\Gamma \leftarrow \K_0(\lambda); ~\rho \leftarrow
    \D_{\Gamma};~\psi \leftarrow \K_1(\Gamma,\rho)~:~
    \A_3^{\PP(\psi,.,.)}(\Gamma,\psi) =1] \\ \approx \Pr[\Gamma
    \leftarrow \K_0(\lambda); ~\rho \leftarrow
    \D_{\Gamma};~(\psi,\tau_{sim}) \leftarrow \SIM_1(\Gamma,\rho)~:~
    \A_3^{\SIM(\psi,\tau_{sim},.,.)}(\Gamma,\psi) =1]\thinspace,
  \end{multline*}  
  where  
  \begin{itemize}  
  \item $\PP(\psi,.,.)$ % is an oracle that
    emulates the actual prover. It takes as input a pair $(x,w) $ 
    and outputs a proof $\pi$ if $(x,w) \in R_{\rho}$. Otherwise, it
    outputs $\perp$.
  \item $\SIM(\psi,\tau_{sim},.,.)$ is an oracle that takes as input
    $(x,w)$. It outputs a simulated proof
    $\SIM_2(\psi,\tau_{sim},x)$ if $(x,w) \in R_{\rho}$ and
    $\perp$ if $(x,w) \not\in R_{\rho}$.
  \end{itemize}  
\end{description}   


\subsection{Definitions of Re-Randomizable Encryption and RCCA Security } \label{RCCA-def}

A public-key encryption (PKE) scheme is defined as follows.

  \begin{definition}{(PKE)} A public-key encryption (PKE) scheme is tuple of algorithms $\mathcal{E} = (\Setup, \KeyGen, \Enc, \Dec)$ that:
    \begin{description}
    \item[$\boldmath{\Setup(\lambda)  }:$] This is a setup algorithm that takes the security parameter $\lambda$ and generates the public parameter $\PPP$.
    \item[$\boldmath{\KeyGen(\PPP) }:$] is key generation algorithm which takes in $\PPP$ and outputs the public key $\PK$ as well as a secret key $\SK$.
    \item[$\boldmath{\Enc(\PPP, \PK, m)}:$] is a randomized algorithm that takes as input a pair $(\PK, m)$ made of a public key and a plaintext. It outputs a ciphertext $c$.
    \item[$\boldmath{\Dec(\PPP, \SK, c) }:$]  takes in a secret key $\SK$ and a ciphertext $C$. It outputs either a plaintext $m \in \mathcal{M}$ or $\bot$.
    \end{description}

We assume that valid ciphertexts (i.e., which are in the range of the encryption algorithm) are publicly recognizable.

    The correctness  and public verifiability are defined as follows:
    \begin{enumerate}
    \item (Correctness) For any ciphertext $c$ computed by $c \gets \Enc(\SK, m)$, we have always $m=\Dec(\PK, c) $.
  %  \item (Structure-preserving) The message, the ciphertext and the public keys produced by the encryption scheme are all group elements.
  %    And there is not hash function in the scheme.
    \item (Public Verifiability) There exists a  PPT algorithm $\Verif(\PPP, \PK, c) \to \{0,1\}$ which returns false if and only if $\Dec(\PK, c)$ outputs $\bot$. 
    \end{enumerate}
  \end{definition}


\begin{definition} A PKE scheme $\mathcal{E} = (\Setup, \KeyGen, \Enc, \Dec)$ is secure against \emph{adaptive chosen-ciphertext attack (IND-CCA2)} if   no PPT adversary $\A$ has non-negligible advantage in the experiment below:

  \begin{align*}
    \Adv_{\A, \mathcal{E}}^{IND\textrm{-}CCA2} (\lambda) := \Pr \left[
    \beta = \beta'
    \middle|
    \begin{array}{l}
      \PPP \gets \Setup(\lambda)\\
      (\PK, \SK) \gets \KeyGen(\PPP)\\
      (m_0, m_1) \gets \A^{\Dec^1(\SK, \cdot)}(\PPP, \PK)\\
      \beta \sample \{0,1\}\\
      c^\star \gets \Enc(\PK, m_\beta  )\\
      \beta ' \gets \A^{\Dec^2_{c^\star}(\SK, \cdot)}(\PPP, \PK, c^\star)
    \end{array}
    \right] - \frac{1}{2}
  \end{align*}
    In the above experiment, $\Dec^1(\SK, \cdot)$ is an oracle which   decrypts any arbitrary ciphertext and $\Dec^2_{c^\star }(\SK, \cdot)$ is a restricted oracle which allows the adversary to decrypt any ciphertext except $c^\star$.
   
  \end{definition}
   
  We also recall the notion of Replayable Chosen-Ciphertext Security (RCCA) \cite{DBLP:conf/crypto/CanettiKN03}, which is defined via a similar security game   except that, 
	at each decryption query occurring after the challenge phase,   if the ciphertext decrypts to one the messages $\{m_0,m_1\}$, the oracle
	returns $Replay$.

  \begin{definition} A \emph{re-randomizable encryption} scheme is tuple of efficient algorithms $\mathcal{E} = (\Setup, \KeyGen, \Enc, \Dec, \Rerand)$ that:
    \begin{description}
    \item[$\boldmath{\Setup(\lambda)  }:$] The setup algorithm takes the security parameter $\lambda$ and generates the public parameter $\PPP$.
    \item[$\boldmath{\KeyGen(\PPP)  }:$] The key generation algorithm takes $\PPP$, outputs the public key $\PK$ and the secret key $\SK$.
    \item[$\boldmath{\Enc(\PPP, \PK, m)  }:$] is a randomized  encryption algorithm that   inputs $(\PPP, \PK, m)$. It generates a ciphertext $c$.
    \item[$\boldmath{\Dec(\PPP, \SK, c)  }:$] The decryption algorithm takes $(\PK, C)$, it tries to decrypt the ciphertext if it can not then outputs $\bot$, 
      otherwise it outputs the decryption result $m$ which is in the message space $\mathcal{M}$.
    \item[$\boldmath{\Rerand(\PPP, \PK, c)  }:$] is a probabilistic  re-randomization algorithm. It takes as input $(\PK,c)$   and outputs a new ciphertext $c'$.
    \end{description}


    Ccorrectness is generalized as follows.   For all $\PPP \gets \Setup(\lambda)$, $(\PK, \SK) \gets \KeyGen(\PPP)$,  the following two conditions must  hold: 
    \begin{enumerate}
    \item[ ]  $~$ 
		%\smallskip \smallskip 
      \begin{enumerate}
      \item For any message $m$, $m=\Dec(\PPP, \SK, \Enc(\PPP, \PK, m ))  $.
      \item For any    $m$, $\Dec(\PPP, \SK, \Rerand(\PPP, \PK, c)) = \Dec(\PPP, \SK, c)$.
      \end{enumerate}
   % \item (Structure-preserving) The message, the ciphertext and the public keys produced by the encryption scheme are all group elements.
   %   And there is not hash function in the scheme.
     \end{enumerate}
  \end{definition}
\smallskip \smallskip 

  \begin{definition} \label{def:RCCA}
    We say that a \emph{re-randomizable encryption} PKE scheme $\mathcal{E} = (\Setup, \KeyGen, \Enc, \Dec, \Rerand)$ is secure against the \emph{replayable chosen-ciphertext attack} (RCCA) if no PPT adversary has non-negligible advantage in the experiment below
    
    \begin{align*}
      \Adv_{\A, \mathcal{E}}^{RCCA} (\lambda) := \Pr \left[
      \beta = \beta'
      \middle|
      \begin{array}{l}
        \PPP \gets \Setup(\lambda)\\
        (\PK, \SK) \gets \KeyGen(\PPP)\\
        (m_0, m_1) \gets \A^{\Dec^1(\SK, \cdot)}(\PPP, \PK)\\
        \beta \sample \{0,1\}\\
        c^\star \gets \Enc(\PK, m_b )\\
        \beta ' \gets \A^{\Dec^2_{m_0, m_1}(\SK, \cdot)}(\PPP, \PK, c^\star)
      \end{array}
      \right] - \frac{1}{2}
    \end{align*}
      In the experiment, $\Dec^1(\SK, \cdot)$ is an oracle which can decrypt any ciphertext and $\Dec^2_{m_0, m_1}(\SK, \cdot)$ is a restricted oracle which allows the adversary to decrypt any ciphertext $c$, except when $\Dec(\SK, c) = m_0$ or $\Dec(\SK, c) = m_1$. In these two cases, the oracle returns \emph{Replay}.
  \end{definition}
  
  We also define the unlinkability of the re-randomizable encryption scheme which was first proposed by Prabhakaran and Rosulek~\cite{DBLP:conf/crypto/PrabhakaranR07} and re-used by Chase {\it et al.} \cite{DBLP:conf/eurocrypt/ChaseKLM12}. It intuitively captures that, for any ciphertext 
	$c$ in the support of the encryption algorithm, a re-randomization of $c$ is identically distributed to a fresh encryption of $\Dec(\PPP,\SK,c)$.
	
  \begin{definition}\label{def:unlikability} We say that a \emph{re-randomizable encryption} PKE scheme $\mathcal{E} = (\Setup, \KeyGen, \Enc, \Dec, \Rerand)$ is \emph{unlinkable} if no PPT adversary $\A$ has noticeable  advantage in the experiment below.
    \begin{multline*}
      \Adv_{\A, \mathcal{E}}^{\mathrm{unlink}} (\lambda) \\ := \Pr \left[
      \beta = \beta '
      \middle|
      \begin{array}{l}
        \PPP \gets \Setup(\lambda)\\
        (\PK, \SK) \gets \KeyGen(\PPP)\\
        c \gets \A (\PPP, \PK)\\
				\beta \sample \{0,1\}\\
				if 	~	\Dec(\PPP, \SK, c)=\perp\\ 
				\qquad \quad ~then~output~(\beta,\beta')~with~\beta' \sample \{0,1\} \\
        if~ \beta = 1\\ \qquad \quad ~ then~ c^\star \leftarrow \Enc(\PPP, \PK, \Dec(\PPP, \SK, c))\\
        else~c^\star \leftarrow \Rerand(\PPP,\PK, c )\\
        \beta' \gets \A (\PPP, \PK, c^\star) \\
				output~(\beta,\beta')
      \end{array}
      \right] - \frac{1}{2}
    \end{multline*}
    %is negligible against all PPT adversary $\A$.  %Here $\Dec^1(\SK, \cdot)$ is an oracle which can decrypt any ciphertext.
  \end{definition}
  
In the above definition, if the adversary chooses an invalid ciphertext $c$, we replace its output $\beta'$ by a random bit so as to annihilate 
its advantage: namely, re-randomized ciphertexts are only required to be satistically indistinguishable from fresh ciphertexts when the 
re-randomization is applied to  valid ciphertexts.

Like  Prabhakaran and Rosulek~\cite{DBLP:conf/crypto/PrabhakaranR07} and   Chase {\it et al.} \cite{DBLP:conf/eurocrypt/ChaseKLM12}, 
we will consider statistical unlikability, i.e. statistical re-randomizability
of adversarially computed valid ciphetext. 




\end{document}




